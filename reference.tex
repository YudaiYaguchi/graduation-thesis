\newpage
\addcontentsline{toc}{section}{参考文献}
\begin{thebibliography}{99}

  \bibitem{meti_robot} 経済産業省,自動配送ロボットの将来像を取りまとめました,
  \url{https://www.meti.go.jp/press/2024/02/20250226002/20250226002.html}
  %
  \bibitem{idou}青柳 誠司, 佐藤 伸仁, 山本 恭輝, 高橋 智一, 鈴木 昌人, "移動ロボットの移動
  障害物回避に関するファジィルールの学習 ポテンシャル法, 強化学習法との比較,"
  システム制御情報学会論文誌,Vol.34,No.8,pp.209–218,2021.
  
  \bibitem{neural-network}Ngangbam Herojit Singh and Khelchandra Thongam,
  "Neural network-based approaches for mobile robot navigation in static and moving obstacles environments,"
  Intelligent Service Robotics 12.1 (2019): 55–67. Web. \url{https://doi.org/10.1007/s11370-018-0260-2}.
  
  \bibitem{potential}Xing. Yang, Wei. Yang, Huijuan. Zhang, Hao. Chang, Chin-Yin. Chen, and Shuangchi. Zhang, "A new method for robot path planning based artificial potential field," 
  2016 IEEE 11th Conference on Industrial Electronics and Applications (ICIEA),
  pp. 1294-1299, 2016.
  
  \bibitem{kasuyama}粕山 剛輝,”壁沿い走行を組み込んだ仮想ゴールポイントに基づくポテンシャル法による停留と障害物回避,”東京都市大学2021年度卒業論文(2021)

  \bibitem{hiraoka} 平岡 翔,”ポテンシャル法における停留問題の回避,および効率的な経路生成手法,”東京都市大学2022年度卒業論文(2022)
  %

  \bibitem{koike} 小池 基也,”A*探索法を組み込んだ静止障害物に対するポテンシャル法の回避方法,”東京都市大学2023年度卒業論文(2023)
  %------%
  %
  \bibitem{enokida} 榎田 日和,”速度推定機能を組み込んだポテンシャル法による動的障害物の回避,”東京都市大学2024年度卒業論文(2024)

  \bibitem{potential-field}瑠城 祐亮,江口 和樹,岩崎 聡,山内 由章,中田 昌宏,
  ”ポテンシャル法によるロボット製品の障害物回避技術の開発,”
  新製品・新技術特集 三菱重工技法,Vol.51, No.1, pp.40–45, 2014.

\end{thebibliography}
