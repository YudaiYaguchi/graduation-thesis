\newpage
\section{シミュレーション}
加速度推定機能を組み込んだ提案手法,従来研究\cite{enokida}の速度推定機能を用いた手法,
およびローカルセンサポテンシャル法の3つの手法についてシミュレーションを実施し,
それぞれの性能を比較評価する.
ローカルセンサポテンシャル法では,ロボットに搭載されたセンサがその時点で検知した障害物の
位置情報のみを用いてポテンシャル場を生成し,そのポテンシャル場の勾配に基づいてロボットを移動させる.
シミュレーションでは,動的障害物として縦・横2[m]の大きさを有する障害物を設定した.
また,ロボットの初期位置は青色の円,目標位置は赤色の円,ロボットの軌跡を青の実線で示す.

\subsection{シミュレーション結果}
5つの異なる場面におけるシミュレーション結果を示す.
各場面について,提案手法,従来研究\cite{enokida},およびローカルセンサポテンシャル法の3手法の性能を比較する.
評価指標には,目的地到達までに要したステップ数(到達時間[step])およびロボットが移動した総距離(経路長[m])を用いる.
また,\textcircled{1}~\textcircled{4}は時間経過を示している.
  
\subsubsection{横方向に等加速で進行する障害物の回避}
図\ref{fig:scene1}に横方向に等加速で進行する障害物の回避シミュレーション結果を示し,
表\ref{tab:scene1}に各手法の到達時間と経路長の比較結果を示す.
本場面では,初期速度は $(0.035, 0.000)$[m/step],加速度は $(0.004, 0.000)$[m/step$^2$]に設定した.


\begin{figure}[t]
  \centering
 
  \begin{subfigure}[b]{0.45\textwidth}
    \centering
    \includegraphics[width=\textwidth]{figure/scene1/suggestion.png}
    \caption{提案手法}
    \label{fig:scene1/suggestion}
  \end{subfigure}
  \hfill
  \begin{subfigure}[b]{0.45\textwidth}
    \centering
    \includegraphics[width=\textwidth]{figure/scene1/enokida.png}
    \caption{従来研究\cite{enokida}}
    \label{fig:scene1/enokida}
  \end{subfigure}
  \hfill
  \begin{subfigure}[b]{0.45\textwidth}
    \centering
    \includegraphics[width=\textwidth]{figure/scene1/potential.png}
    \caption{ローカルセンサポテンシャル法}
    \label{fig:c}
  \end{subfigure}

  \caption{横方向に等加速で進行する障害物の回避}
  \label{fig:scene1}
\end{figure}


\begin{table}[!ht]
\centering
\caption{横方向に等加速で進行する障害物の回避性能比較}
\label{tab:scene1}
\begin{tabular}{|l|c|c|}
\hline
手法 & 到達時間[step] & 経路長[m] \\
\hline
提案手法 & 113 & 13.56 \\
\hline
従来研究\cite{enokida} & 144 & 17.28 \\
\hline
ローカルセンサポテンシャル法 & 120 & 14.40 \\
\hline
\end{tabular}
\end{table}


提案手法は到達時間が113[step],経路長が13.56[m]となり,従来研究\cite{enokida}と比較して到達時間が31[step]短縮され,経路長も3.72[m]短くなった.
ローカルセンサポテンシャル法と比較しても,到達時間が7[step],経路長が0.84[m]それぞれ改善されている.



\subsubsection{斜め方向に等加速で進行する障害物の回避}
図\ref{fig:scene2}に斜め方向に等加速で進行する障害物の回避シミュレーション結果を示し,
表\ref{tab:scene2}に各手法の到達時間と経路長の比較結果を示す.
本場面では,障害物の初期速度は$(-0.035, -0.035)$[m/step],
加速度は$(-0.0030, -0.0015)$[m/step$^2$]に設定した.


\begin{figure}[t]
    \centering

  \begin{subfigure}[b]{0.45\textwidth}
    \centering
    \includegraphics[width=\textwidth]{figure/scene2/suggestion.png}
    \caption{提案手法}
    \label{fig:scene2/suggestion}
  \end{subfigure}
  \hfill
  \begin{subfigure}[b]{0.45\textwidth}
    \centering
    \includegraphics[width=\textwidth]{figure/scene2/enokida.png}
    \caption{従来研究\cite{enokida}}
    \label{fig:scene2/enokida}
  \end{subfigure}
  \hfill
  \begin{subfigure}[b]{0.45\textwidth}
    \centering
    \includegraphics[width=\textwidth]{figure/scene2/potential.png}
    \caption{ローカルセンサポテンシャル法}
    \label{fig:scene2/potential}
  \end{subfigure}
  \caption{斜め方向に等加速で進行する障害物の回避}
  \label{fig:scene2}
\end{figure}


\begin{table}[!ht]
\centering
\caption{斜め方向に等加速で進行する障害物の回避性能比較}
\label{tab:scene2}
\begin{tabular}{|l|c|c|}
\hline
手法 & 到達時間[step] & 経路長[m] \\
\hline
提案手法 & 106 & 12.72 \\
\hline
従来研究\cite{enokida} & 139 & 16.68 \\
\hline
ローカルセンサポテンシャル法 & 115 & 13.80 \\
\hline
\end{tabular}
\end{table}

提案手法は到達時間が106[step],経路長が12.72[m]となり,従来研究\cite{enokida}と比較して到達時間が33[step]短縮され,経路長も3.96[m]短くなった.
ローカルセンサポテンシャル法と比較しても,到達時間が9[step],経路長が1.08[m]それぞれ改善されている.



\subsubsection{横方向の往復移動をする障害物の回避}
図\ref{fig:scene3}に横方向の往復移動をする障害物の回避シミュレーション結果を示し,
表\ref{tab:scene3}に各手法の到達時間と経路長の比較結果を示す.
本場面では,障害物の初期速度は$(0.35, 0.00)$[m/step],加速度は$(-0.01, 0.00)$[m/step$^2$]に設定した.

\begin{figure}[t]
  \centering

  \begin{subfigure}[b]{0.45\textwidth}
    \centering
    \includegraphics[width=\textwidth]{figure/scene3/suggestion.png}
    \caption{提案手法}
    \label{fig:scene3/suggestion}
  \end{subfigure}
  \hfill
  \begin{subfigure}[b]{0.45\textwidth}
    \centering
    \includegraphics[width=\textwidth]{figure/scene3/enokida.png}
    \caption{従来研究\cite{enokida}}
    \label{fig:scene3/enokida}
  \end{subfigure}
  \hfill
  \begin{subfigure}[b]{0.45\textwidth}
    \centering
    \includegraphics[width=\textwidth]{figure/scene3/potential.png}
    \caption{ローカルセンサポテンシャル法}
    \label{fig:scene3/potential}
  \end{subfigure}
  \caption{横方向の往復移動をする障害物の回避}
  \label{fig:scene3}
\end{figure}


\begin{table}[!ht]
\centering
\caption{横方向の往復移動をする障害物の回避性能比較}
\label{tab:scene3}
\begin{tabular}{|l|c|c|}
\hline
手法 & 到達時間[step] & 経路長[m] \\
\hline
提案手法 & 110 & 13.20 \\
\hline
従来研究\cite{enokida} & 136 & 16.32 \\
\hline
ローカルセンサポテンシャル法 & 121 & 14.52 \\
\hline
\end{tabular}
\end{table}

提案手法は到達時間が110[step],経路長が13.20[m]となり,従来研究\cite{enokida}と比較して到達時間が26[step]短縮され,経路長も3.12[m]短くなった.
ローカルセンサポテンシャル法と比較しても,到達時間が11[step],経路長が1.32[m]それぞれ改善されている.







\subsubsection{横方向に等加速で進行する障害物と静的障害物の回避}
図\ref{fig:scene4}に横方向に移動する動的障害物と静的障害物の壁が存在する場面における回避シミュレーション結果を示し,
表\ref{tab:scene4}に各手法の到達時間と経路長の比較結果を示す.
本場面では,障害物の初期速度は$(0.035, 0.000)$[m/step],加速度は$(0.004, 0.000)$[m/step$^2$]に設定した.



\begin{figure}[t]
  \centering

  \begin{subfigure}[b]{0.45\textwidth}
    \centering
    \includegraphics[width=\textwidth]{figure/scene4/suggestion.png}
    \caption{提案手法}
    \label{fig:scene4/suggestion}
  \end{subfigure}
  \hfill
  \begin{subfigure}[b]{0.45\textwidth}
    \centering
    \includegraphics[width=\textwidth]{figure/scene4/enokida.png}
    \caption{従来研究\cite{enokida}}
    \label{fig:scene4/enokida}
  \end{subfigure}
  \hfill
  \begin{subfigure}[b]{0.85\textwidth}
    \centering
    \includegraphics[width=\textwidth]{figure/scene4/potential.png}
    \caption{ローカルセンサポテンシャル法}
    \label{fig:scene4/potential}
  \end{subfigure}
  \caption{横方向に等加速で進行する障害物と静的障害物の回避}
  \label{fig:scene4}
\end{figure}


\begin{table}[!ht]
\centering
\caption{横方向に等加速で進行する障害物と静的障害物の回避性能比較}
\label{tab:scene4}
\begin{tabular}{|l|c|c|}
\hline
手法 & 到達時間[step] & 経路長[m] \\
\hline
提案手法 & 145 & 17.40 \\
\hline
従来研究\cite{enokida} & 171 & 20.52 \\
\hline
ローカルセンサポテンシャル法 & 到達できない & --- \\
\hline
\end{tabular}
\end{table}

提案手法は到達時間が145[step],経路長が17.40[m]となり,従来研究\cite{enokida}と比較して到達時間が26[step]短縮され,経路長も3.12[m]短くなった.
一方,ローカルセンサポテンシャル法では目的地への到達が不可能であった.
この結果から,加速度推定機能を組み込むことで,複雑な環境下においても適切な経路生成が可能となることが確認された.






\subsubsection{横方向に等速で進行する障害物の回避}
図\ref{fig:scene5}に横方向に等速で進行する障害物の回避シミュレーション結果を示し,
表\ref{tab:scene5}に各手法の到達時間と経路長の比較結果を示す.
本場面では,障害物の初期速度は$(0.035, 0.000)$[m/step],加速度は$(0, 0)$[m/step$^2$]であり,等速運動を行う.


\begin{figure}[t]
  \centering

  \begin{subfigure}[b]{0.45\textwidth}
    \centering
    \includegraphics[width=\textwidth]{figure/scene5/suggestion.png}
    \caption{提案手法}
    \label{fig:scene5/suggestion}
  \end{subfigure}
  \hfill
  \begin{subfigure}[b]{0.45\textwidth}
    \centering
    \includegraphics[width=\textwidth]{figure/scene5/enokida.png}
    \caption{従来研究\cite{enokida}}
    \label{fig:scene5/enokida}
  \end{subfigure}
  \hfill
  \begin{subfigure}[b]{0.45\textwidth}
    \centering
    \includegraphics[width=\textwidth]{figure/scene5/potential.png}
    \caption{ローカルセンサポテンシャル法}
    \label{fig:scene5/potential}
  \end{subfigure}
  \caption{横方向に等速で進行する障害物}
  \label{fig:scene5}
\end{figure}


\begin{table}[!ht]
\centering
\caption{横方向に等速で進行する障害物の回避性能比較}
\label{tab:scene5}
\begin{tabular}{|l|c|c|}
\hline
手法 & 到達時間[step] & 経路長[m] \\
\hline
提案手法 & 122 & 14.64 \\
\hline
従来研究\cite{enokida} & 133 & 15.96 \\
\hline
ローカルセンサポテンシャル法 & 159 & 19.08 \\
\hline
\end{tabular}
\end{table}

提案手法は到達時間が122[step],経路長が14.64[m]となり,従来研究\cite{enokida}と比較して到達時間が11[step]短縮され,経路長も1.32[m]短くなった.
ローカルセンサポテンシャル法と比較しても,到達時間が37[step],経路長が4.44[m]それぞれ改善されている.

\subsection{考察}
5つの場面におけるシミュレーション結果を総合すると,提案手法は全ての場面において,従来研究\cite{enokida}およびローカルセンサポテンシャル法と比較して,より短い到達時間と経路長を実現した.
これは,加速度推定機能を導入することで,動的障害物の運動状態をより正確に予測でき,ロボットが余分な回避行動を取ることなく効率的な経路を選択できたためであると考えられる.
図\ref{fig:scene3}に示すように,ロボットが動的障害物をセンサで検知した時点では,障害物は$x$方向に正の速度を有していたが,その後速度が減少して0となり,最終的には負方向へと反転している.
このように運動方向が反転する動的障害物に対しても,提案手法は最も短い到達時間および経路長を示しており,運動状態が大きく変化する障害物に対しても有効に機能する手法であることが確認できる.
また,図\ref{fig:scene4}のように動的障害物と静的障害物である壁が同時に存在する複雑な環境下において,ローカルセンサポテンシャル法が目的地への到達に失敗したのに対し,提案手法および従来研究\cite{enokida}はいずれも目的地への到達に成功した.
さらに,提案手法では動的障害物に対して加速度推定に基づく回避を行いながら,静的障害物に対しては従来研究\cite{hiraoka} の静的障害物回避手法へと適切に切り替えながら走行していることが確認できる.
このことから,提案手法が動的および静的環境が混在する状況においても安定した回避行動を実現できることを示している.
図\ref{fig:scene5} では加速度を持たない等速で移動する障害物を対象としている.提案手法は他の手法より短い到達時間と経路長を達成した.
加速度を持たない障害物に対しても性能が向上していることから,等速で移動する障害物に対しても適切な回避行動を行い有効な手法であると言える.
以上の結果より,提案手法は加速度を持つ障害物に対する性能向上だけでなく,等速で移動する障害物に対しても効率的な回避を実現できることが示された.
