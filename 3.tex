\clearpage
\section{提案手法}
従来研究\cite{kasuyama}~\cite{koike}では,静的障害物のみを対象としており,動的に移動する障害物については十分に考慮されていない.
また,従来研究\cite{enokida}では動的障害物の回避を扱っているものの,対象としているのは等速運動を行う障害物であり,
加速度を有する障害物に対しては適切な経路生成が困難であるという課題が残っている.
そこで本研究では,センサにより検知した動的障害物の情報からその速度および加速度を推定し,
複数秒先の将来位置を予測する手法を提案する.
具体的には,予測した将来位置をポテンシャル場に反映させることで,加速度を持つ動的障害物に対しても,
ポテンシャル法に基づくスムーズで効率的な回避挙動を実現することを目指す.
加えて,本研究の手法に従来研究\cite{hiraoka}の静的障害物回避アルゴリズムを組み込むことにより,
動的障害物と静的障害物の双方に対応可能な回避手法を構築する.
また,動的障害物をセンサが検知した場合には,その四隅の位置情報を取得できるものとする.


\subsection{ロボットのセンサ距離の設定}
従来研究\cite{enokida}では,動的障害物に対する検知距離を200[cm]に設定している.
しかし,想定している障害物は等速運動を前提としており,
速度変化を伴う動的障害物には十分に対応できないという問題がある.
加速する障害物は短時間で速度が大きく変化するため,200[cm]程度の検知距離では回避動作の開始が遅れる可能性が高い.
本研究ではこれらの課題を踏まえ,ロボットに搭載するセンサとして
SICK社製のレーザスキャナ TiM1xx シリーズ(型番:TiM150-3010300)を想定する.
本センサは測定範囲0.05~10[m]を有するセンサである.
なお,本研究はシミュレーションによる評価を対象としているため,
実機センサの特性を完全に再現するものではなく,
検知距離などの主要な仕様を参考にした簡易的なモデルを用いる.
このセンサの性能を踏まえ,検知距離を300[cm]に設定した.
検知範囲を拡大することで障害物をより早期に捉え,
その速度および加速度を推定するために必要な複数回の観測データを安定して取得できる.
これにより,障害物が加速しながら接近する場合でも将来位置の予測精度を維持でき,
衝突リスクを低減しつつ適切な回避経路を生成するための十分な時間的余裕を確保する.

% センサ 公式ページ:https://www.sick.com/jp/ja/catalog/products/lidar-and-radar-sensors/lidar-sensors/tim/tim150-3010300/p/p595144?tab=detail#technical-details
% 14.5 Hz = 1/14.5 = 0.069 秒周期


\subsection{動的障害物の加速度推定と将来位置予測}
本研究では,動的障害物 $ob_m$ の将来位置を推定するために,
センサで取得した障害物の四隅の位置情報から重心を算出し,その重心を当該時刻における障害物の代表位置として用いる.
続いて,連続する観測データから速度および加速度を推定し,
等加速度運動の運動方程式に基づいて複数秒先の位置を予測する.
想定するレーザスキャナのスキャン周波数は 14.5 [Hz] であるが,
実環境でのセンサノイズを考慮し,
速度および加速度推定の安定化を目的として
サンプリング間隔を $\Delta t = 0.1$ [s] と設定する.
以下に,動的障害物の将来位置予測の手順を示す.



\begin{enumerate}
  \item センサによって動的障害物 $ob_m$ の四隅の座標を取得し,そこから算出した代表位置を次式のように定義する.
        \begin{equation}
            \mathbf{p}_{t} = (x_{1}, y_{1}), \quad
            \mathbf{p}_{t+\Delta t} = (x_{2}, y_{2})
        \end{equation}

  \item 重心位置の変化量から速度ベクトル $\mathbf{v}$ を推定する.
      \begin{equation}
            \mathbf{v} = (x_{v}, y_{v})
            = \frac{\mathbf{p}_{t+\Delta t} - \mathbf{p}_{t}}{\Delta t}
      \end{equation}

  \item さらに速度の変化量に基づき,加速度ベクトル $\mathbf{a}$ を推定する.
      \begin{equation}
            \mathbf{a} = (x_{a}, y_{a})
            = \frac{\mathbf{v}_{t+\Delta t} - \mathbf{v}_{t}}{\Delta t}
      \end{equation}

  \item 推定した速度・加速度と現在位置 $\mathbf{p}_t$ を用いて,
        等加速度運動の運動方程式
        \begin{equation}
          \label{eq:acc_model}
            \mathbf{p}(T)
            = \mathbf{p}_t + \mathbf{v}T + \frac{1}{2}\mathbf{a}T^2
        \end{equation}
        により,予測秒数 $T$ を変化させながら複数秒先の将来位置を逐次算出する.

  \item 推定した将来位置群をポテンシャル場に反映し,  
        ロボットはこれらを基に動的障害物を考慮した回避軌道を生成する.
\end{enumerate}

\vspace{50pt} 
\subsection{提案手法の経路生成アルゴリズム}
提案手法の全体フローを以下に示す.

\begin{algorithm}
\caption{目的地までの経路生成アルゴリズム}
\begin{algorithmic}

\WHILE{目的地に到達していない}
  \STATE センサで周囲の障害物を検出する
  \IF {障害物 $ob_m$ を検出したとき}
    \STATE 四隅の座標から代表位置 $\mathbf{p}_t$ を算出する
    \IF{$ob_m$ が動的障害物である場合}
      \STATE 連続する観測から速度 $\mathbf{v}$ と加速度 $\mathbf{a}$ を推定する
      \FOR{$\tau = 1 \dots T$}
        \STATE 等加速度運動の運動方程式(式\ref{eq:acc_model})により将来位置を算出する
      \ENDFOR
      \STATE 予測位置群を斥力ポテンシャルとしてポテンシャル場へ反映する
    \ELSIF{$ob_m$ が静的障害物である場合}
      \STATE 従来手法\cite{hiraoka}の静的障害物回避アルゴリズムを適用する
    \ENDIF
  \ENDIF

  \STATE ポテンシャル場(目的地からの引力+障害物の斥力(予測位置含む))を構築する
  \STATE ポテンシャル場の勾配に基づき,ロボットを移動させる
\ENDWHILE
\end{algorithmic}
\end{algorithm}

本手法では,ロボットが目的地へ到達するまでポテンシャル法により移動方向を逐次決定する.
まず,ロボットは搭載センサを用いて周囲の障害物を検出する.
検出された障害物に対して,その四隅の座標から代表位置を算出し,対象が動的障害物か静的障害物かを判別する.
動的障害物に対しては,連続した観測から速度および加速度を推定し,
等加速度運動の運動方程式を用いて将来位置を1秒間隔で時刻 $\tau=1\dots T$ まで予測する.
本研究では予測秒数を $T = 20$ と設定し,20秒先までの将来位置を算出する. 
これらの将来位置群を斥力ポテンシャルとしてポテンシャル場に反映させる.
一方,静的障害物に対しては,従来手法\cite{hiraoka}に基づく静的障害物回避アルゴリズムを適用する.
すべての障害物処理が完了すると,目的地に対する引力ポテンシャルと,
静的または動的障害物から生じる斥力ポテンシャルから全体のポテンシャル場を構築する.
ロボットはこのポテンシャル場の勾配方向に沿って逐次移動し,最終的に目的地へ到達する.