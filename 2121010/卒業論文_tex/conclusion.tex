\newpage

\section{結論と今後の課題}
本研究では,除細動直前10秒間のVF波形に対して除細動が有効か否か判定するアルゴリズムを構築し検証を行った.
特に,除細動の"成功"と"失敗"について検討し,失敗については除細動の効果が得られない"失敗"と除細動後に一度自己心拍を再開するものの,再度VFに陥る"再発"の2状態を合わせて定義した.


提案手法としては,除細動直前の10秒の心電図波形に対してウェーブレット変換と疑似微分作用素を用いたウェーブレット変換を適用し,スカログラムを導出することにより特徴量を抽出した.
また,スカログラムから得られる$NSI$,$NTI$,$SDW$,スペクトルエントロピーから特徴量を抽出し,加えて心電図波形から得られるポアンカレプロット解析からも特徴量を抽出した.
ただし,すべての特徴量を考慮し予測を行うことは困難であるため,$\chi^2$検定で検証に有効である特徴量の絞り込みを行い,SVMで3種のカーネル関数と6種の正則化係数を使用して除細動の"成功"と"失敗"の予測を行った.
また,予測システムの有用性を評価するために,感度,特異度,正答率を用いた.
結果としては,感度100\%,特異度93.10\%,正答率96.55\%が得られた.先行研究\cite{SCI}\cite{yosikawa}と比較して,高い正答率を得ることができたため,本研究の手法が有効であると考えられる.

今後の課題としては,本研究では$\chi^2$検定で特徴量の絞り込みを行っているが,絞り込みを行わず,すべての特徴量を考慮し組み合わせを考える必要がある.
予測精度として正答率96.55\%を得ることができたが,実用には不十分である.特徴量すべてを用いることで今回得られた結果を上回る精度が期待される.
%本研究の目的として,高精度な予測を行うことであるため,絞り込みを行わないことでより目的に近づけると考えられる.
また,疑似微分作用素の関数$L(a)$と$H(y)$について検討する必要がある.本研究では,文献\cite{JSCI2022}で提案されている$L(a)=\frac{1}{a}, H(y)=|y|^{\frac{1}{4}}$を採用しており,
これらは暫定的な値として設定されている.よって,VF波形の除細動の効果予測に有効であるパラメータについて検討し,最良の関数の組み合わせを見つけることで,より高精度なシステムの構築につながると考えられる.
%さらに,除細動の失敗については,"失敗"と"再発"の2状態が混合しているため,この2状態の識別を目的とした予測を行い,除細動の"成功","失敗","再発"の3状態について検討する必要があるといえる.
%3状態の予測を行うことで,より患者の状態に合わせた救命措置が可能となり,蘇生率向上に寄与できると思われる.