\makeatletter
\let\my@xfloat\@xfloat
\makeatother

\documentclass[12pt, a4,dvipdfmx]{jarticle}

%%%
%%%		必要ならpackageは増やしてもいい
%%%
\usepackage{graphicx}
%\usepackage[dvipdfmx]{graphicx}
\usepackage[dvipdfmx]{color}
\usepackage{here}
\usepackage{float}
%\usepackage{hyperref}
\usepackage{doublespace}
\usepackage{amsmath,amssymb}
\usepackage{amsfonts}
\usepackage{url}
\usepackage{cases}
\usepackage{fancybox}
\usepackage{ascmac}
\usepackage{geometry}
\usepackage{comment}
\usepackage{diagbox}
\usepackage{xcolor}
\usepackage{colortbl}
\usepackage{multirow}
\usepackage{booktabs}
\counterwithin{figure}{section}%セクションごとの画像番号

\usepackage{pdfpages}
\usepackage[hang,small]{caption}
\usepackage[subrefformat=parens]{subcaption}
\captionsetup{compatibility=false}

\makeatletter
\def\@xfloat#1[#2]{
    \my@xfloat#1[#2]%
    \def\baselinestretch{1}%
    \@normalsize \normalsize
}
\makeatother




%%%
%%%		設定関連(いじらないほうがよい)
%%%
\renewcommand{\baselinestretch}{1.5}
\setlength{\textheight}{24cm}
\setlength{\topmargin}{-1cm}
\setlength{\textwidth}{16.2cm}
\setlength{\oddsidemargin}{-0.2cm}
\setlength{\columnsep}{3zw}


%%%
%%%		指導教員印, 卒業年度, 卒論タイトル 
%%%
\title{
	\vspace{-1cm}
	\begin{flushright}
 		\begin{tabular}{|c|} \hline
 			\small \raisebox{10pt}{指導教員印} \\
 			[-15pt] \hline \\
      [15pt] \hline
 		\end{tabular}
 		\\
 	\end{flushright}
	\vspace{2cm}
	% 卒業年度(変えるところ)
 	{\huge 2024年度卒業論文} \\
 	\vspace{3cm}
 	% タイトル(変えるところ)
 	{\huge 疑似微分作用素を用いたウェーブレット \\
		変換に基づく電気的除細動の効果予測}
    \vspace{2cm}
	{Prediction of the Effect of Electrical Defibrillation based on Wavelet Transform with Pseudo-Differential Operators}
}

%%%
%%%		この2つは無視すること
%%%
\author{}
\date{}


\begin{document}
	%%%
	%%%		所属, 指導教員名, 自分の学籍番号と名前
	%%%
	\begin{singlespace}
		% タイトルページの挿入
		\maketitle
		\large
 		\vspace{2.0cm} \par
 		% 大学名と学部
 		\hspace{9.0cm}東京都市大学 情報工学部 \\ 					
 		\vspace{-0.8cm} \par
 		% 学科名
 		\hspace{9.0cm}情報科学科 \\
 		\vspace{-0.8cm} \par
 		% 研究室名
 		\hspace{9.0cm}自動制御研究室 \\
 		\vspace{-0.8cm} \par
 		% 指導教員名, 役職
 		\hspace{9.0cm}指導教員 大屋\hspace{0.5em}英稔 教授 \\
 		\vspace{-0.8cm} \par
 		\hspace{9.0cm}\hspace{5em}星\hspace{0.5em}義克 講師 \\
		\vspace{-0.8cm} \par
		% 自分の学籍番号と名前
 		\hspace{9.0cm}2121010\hspace{1.5em}市場\hspace{0.5em}裕己
	\end{singlespace}

	%\includepdf[pages={1}]{front.pdf}
	%\includepdf[pages={1}]{abstract.pdf}
	
	%\thispagestyle{empty}		% タイトルページにページ番号を振らないようにするため
  %  \begin{abstract}
  本研究は,自動走行ロボットの社会実装に向けて重要となる,動的障害物を考慮した安全かつ効率的な経路生成手法の確立を目的とする。
  近年,物流分野を中心に自動配送ロボットの実用化が進められており,人手不足の緩和や労働環境の改善が期待されている。
  しかし,未知環境や動的障害物が存在する実環境においては,ロボットが障害物の挙動を適切に予測しながら移動する必要があり,
  高い安全性とリアルタイム性を両立した経路生成が求められる。

  移動ロボットの経路生成には,大局的経路計画,ニューラルネットワーク,時空間RRT,ポテンシャル法などが用いられてきた。
  ポテンシャル法は計算量が少なくリアルタイム処理に優れる一方で,障害物と目的地の影響が釣り合うことで移動不能に陥る停留問題を有する。
  これに対し,仮ゴールの設定や壁沿い走行,回り込み走行を組み合わせることで停留問題を解決する手法が提案されてきた。
  また,動的障害物に対しては,障害物の速度推定を用いた回避手法が報告されているが,
  速度情報のみでは加速や減速を伴う運動を十分に表現できず,将来位置の予測誤差により経路が冗長化するという課題が残されている。
  
  そこで本研究では,従来のポテンシャル法を基盤とし,
  障害物の位置変化から速度および加速度を推定する機能を新たに組み込んだ動的障害物回避手法を提案する。提案手法では,
  障害物の将来位置を複数秒先まで予測し,それらを斥力ポテンシャル場に反映させることで,障害物の進行経路への侵入を回避しつつ,安全かつ効率的な経路生成を実現する。さらに,従来手法および通常のポテンシャル法との比較を通じて,到達時間および経路長の観点から提案手法の有効性を検証する。

  \vspace{10pt} 
  
  In recent years, research on autonomous mobile robots has been actively conducted.
  In particular, in the logistics industry, the development and legal reforms toward the practical use of autonomous delivery robots have been progressing, which are expected to help reduce the long working hours of truck drivers and alleviate labor shortages.
  
  For the social implementation of autonomous delivery robots, safe path generation in unknown environments is essential.
  Although methods such as global path planning and neural networks have been proposed, global path planning tends to cause robots to approach obstacles too closely at corners, and neural networks require a large amount of training data and time.
  Therefore, these methods are not suitable for real-time path generation in unknown environments.
  On the other hand, the potential field method is a lightweight algorithm with low computational cost and excellent real-time performance.
  
  This study aims to construct a dynamic obstacle avoidance method based on the potential field approach incorporating acceleration estimation.
  By introducing acceleration estimation, the robot can predict the future motion of obstacles more accurately, enabling smoother and safer path generation in dynamic environments.
  Finally, the effectiveness of the proposed method is verified through simulation.
\end{abstract}
	%\thispagestyle{empty}		% Abstractにページ番号を振らないようにするため
	
	%%%
	%%%		目次の設定
	%%%
	\newpage 					% 表紙の次のページに目次を出力するため
	\pagenumbering{roman}		% ページ番号をローマ数字(i,ii,iii,...)にする	
	\setcounter{page}{1}		% ページ番号をiとする
	\tableofcontents			% 目次の出力


	%%%
	%%%		本文の設定
	%%%
	\newpage 					% 目次の次のページから本文を出力するため
	\pagenumbering{arabic}		% ページ番号をアラビア数字(1,2,3...)にする
	\setcounter{page}{1} 		% ページ番号を1とする
	
	%%%
	%%%		本文の挿入
	%%%	
	%%%		[やり方]
	%%%		\input{xyz.tex}でxyz.texファイルを挿入できる
	%%%		ただし, 挿入するファイルの文頭に \newpage \section{}を入れること

	%% 序章 %%
	\newpage
\section{序論}

\subsection{研究背景}
\begin{figure}[b]
  \centering
  \includegraphics[scale=0.5]{figure/hospital.png}
  \caption{現場到着までの所要時間及び病院収容までの時間の推移 \label{fig:hospital}}
\end{figure}
我が国では日々多くの人が突然の心停止が原因で命を落としている.令和4年度における心肺停止傷病者数は約14万人であり,そのうち心原生かつ心肺機能停止が一般市民によって目撃された数は2万8千人に及ぶ\cite{people}.
主な原因は心室細動(Ventricular Fibrillation:VF)と呼ばれる致死性不整脈である.心室細動に陥ると心臓が無秩序な痙攣によって血液を送り出すことができなくなり,数秒で意識を失い数分以内に脳を含む全身の細胞が死に至る\cite{AED}.
近年の傾向としては,図\ref{fig:hospital}より救急隊の到着及び病院収容時間は年々増加していることがわかる.また,図\ref{fig:recovry},\ref{fig:survival}より社会復帰率,生存率が減少傾向にあることが見てとれる.
心室細動からの救命には迅速な早期の電気的除細動(以下,除細動)が有効であることが知られており,現場に居合わせた一般人(バイスタンダー)による応急手当が極めて重要である.
実際,図\ref{fig:die}より除細動を実施した傷病者1229人の1ヶ月後の生存率は50.3\%であり,実施しなかった傷病者数11,766人の生存率と比較すると7.6倍もの差があることが明らかである.
このため,除細動を行うための自動体外式除細動器(Automated External Defibrillator:AED)が空港や商業施設等に配備されており,AEDの販売台数としても毎年増加している.
しかし,近年VFによる心肺停止が長期化した場合,除細動が効果的でなく自己心拍を再開しない場合があることが報告されており,
この場合には心肺蘇生法(CardioPulmonary Resuscitation:CPR)や投薬などの措置を施すことで蘇生率が向上する可能性がある.
これらのことから,除細動を実施する前の心電図波形(ElectroCardioGram:ECG)を解析し,解析結果から除細動が有効に機能するか否かを予測する研究が進められている.

\begin{figure}[t]
  \centering
  \includegraphics[scale=0.7]{figure/recovry.png}
  \caption{社会復帰率の推移\cite{people} \label{fig:recovry}}
\end{figure}

\begin{figure}[t]
  \centering
  \includegraphics[scale=0.7]{figure/survival.png}
  \caption{生存率の推移\cite{people} \label{fig:survival}}
\end{figure}

\begin{figure}[t]
  \centering
  \includegraphics[scale=0.6]{figure/die.png}
  \caption{応急手当の実施及び救命効果\cite{people} \label{fig:die}}
\end{figure}
  

\subsection{従来研究}
これらを踏まえて,心電図波形解析に関する研究が多く行われている.
1.1節で述べた通り,VFに対して除細動は効果的である.しかし,除細動は身体に大きな負荷がかかるため,回数が増加すると救命率が低下することが示されている\cite{count}.
そのため,患者の状態に応じて除細動を適用することが望ましく,除細動を必要とする心電図波形と除細動を適用してはいけない波形を識別する研究が行われている\cite{JSCI2022}\cite{nature2023}.
解析対象の心電図波形は,除細動を必要としない正常洞調律(Sinus Rhythm:SR),無脈性電気活動(Pulseless Electorical Activity:PEA),および必要である心室頻拍(Ventricular Tachycardia:VT),心室細動(Ventricular Fibrillation:VF)
であり,これらの分類を行う手法が提案されている.
文献\cite{JSCI2022}では疑似微分作用素を用いたウェーブレット変換による波形解析手法を提案しており,通常のウェーブレット変換では得られない特徴を捉えられることが示されている.
%ただし,疑似微分作用素にはいくつかの設定すべきパラメータがあり,除細動の識別における最良の組み合わせを示されているものの,除細動の効果予測に関するパラメータの検討はされていない.
また文献\cite{nature2023}では文献\cite{JSCI2022}の特徴量に加え,ウェーブレット変換によって導出できる時間-周波数領域のエネルギー分布であるスカログラムより,時間軸に着目した新しい特徴量としてNTI (Normalized Time Index)を導入している.
さらに除細動の効果を予測する研究では,除細動直前のVF波形に着目し,抽出する特徴量の時々刻々の変化に着目した新しい特徴量を用いた予測手法が提案されている\cite{SCI}.
文献\cite{SCI}の手法では感度88.24\%,特異度93.10\%,正答率91.30\%という結果が得られている.
また,文献\cite{yosikawa}でウェーブレット変換を用いた時間周波数解析に加え,高速フーリエ変換(Fast Fourier Transform:FFT)を用いた手法が提案されており,
感度96.55\%,特異度93.10\%,正答率94.83\%という成果が示されている.ただし依然として人命に関する精度としては不十分である.

\subsection{研究目的と研究意義}
本研究では,除細動の効果予測手法を提案する.具体的には,まず心室細動における除細動直前の波形に対して,ウェーブレット変換と疑似微分作用素を用いたウェーブレット変換を適用し,
各状態の特徴を抽出するためにそれぞれの変換結果からスカログラムを導出する.また,スカログラムから導出できるNSI(Normalized Spectrum Index),
NTI(Normalized Time Index),SDW(Scale Distribution Width),さらにスペクトルエントロピーとしてシャノンエントロピー,レニーエントロピーを採用する.
これらに加え心電図波形に対してポアンカレプロット解析を行い,基本統計量を用いて特徴量を抽出する.
ただし,得られたすべての特徴量を用いて有用性の検証を行うことは,計算量の面で困難である.
よって,$\chi^2$検定を行い予測に有効である特徴量の選択を行う.最後に検定によって選定された特徴量を組み合わせ,除細動の"成功"と"失敗"の2クラス分類をサポートベクターマシン(Support Vector Machine:SVM)を用いて行う.
この際,3種のカーネル関数(線形,RBF,多項式(2$\sim$5次))と6種の正則化係数を用いて,感度,特異度,そして正答率で評価することにより,
提案手法の本研究の有用性を示す.本研究の手法により,予測精度が向上することで傷病者に対して適切な処置に繋がり,救命率の向上に寄与することができる.
また,除細動を行う前に除細動の効果の有無がわかることにより,救急隊員がより迅速に患者の状態に適した処置を実施することが可能になる.

%疑似微分作用素にはいくつかの設定すべきパラメータが存在するため,パラメータを変更することで変化させた特徴を得ることができる.よってより予測に寄与する組み合わせを見つけることで有用な結果を得られるため,
%パラメータの組み合わせの検討も行う.

\subsection{本論文の構成と用いる表記}
本論文では,第2章で心電図波形,第3章では予測に使用する特徴量と抽出方法について説明する.そして第4章では特徴量の選択方法,選択された特徴量を述べる.
さらに,第5章で提案するサポートベクターマシンを用いた予測システムの検証方法,第6章で予測結果を示す.最後に第7章で今後の課題についてまとめる.

なお,本論文で使用する表記を以下に示す.
\begin{align}
  \bar{x} = \frac{1}{N}\sum_{i=1}^{N}x_i
\end{align}
ベクトル$x_i=(i=1,2\dots,N)$に対し,$\bar{x}$はベクトルの平均を表す.また,複素関数$f(t)$に対して,$\overline{f(t)}$は複素共役を表す.


  
	%% 第2章 %%
	\newpage
\section{心電図波形}
本章では心電図波形\cite{ECG},解析に用いた心電図波形のデータベース,および信号の前処理について示す.
また,本論文における心電図波形は横軸に時間[sec],そして縦軸に電位[mV]をとる.

\begin{figure}[t]
  \centering
  \includegraphics[scale=0.7]{figure/QRS.png}
  \caption{正常洞調律\cite{QRS} \label{fig:QRS}}
\end{figure}

\subsection{正常洞調律(Sinus Rhythm:SR)}
正常洞調律とは電気的興奮が正しく反復され,心臓の拍動が一定のリズムを保つ状態である.SRの図解を図\ref{fig:QRS},SR波形の一例を図\ref{fig:SR}に示す.
SRでは最初に小さな波P波が起こり次にQ波,R波,S波が続き,これらをまとめてQRS波と呼ぶ.心臓の動きとしては,P波で心房の収縮が起こり,QRS波で心室の収縮が行われる.
それぞれ心電図波形を読みとるのに重要な波形であり,T波が診断の決め手になる心電図波形は多くない.また,SRは心臓が正常に機能している状態を示すため除細動適用すべきでない波形である.

\begin{figure}[t]
  \begin{tabular}{cc}
    \begin{minipage}[t]{0.45\hsize}
      \centering
      \includegraphics[width=1.0\columnwidth]{figure/SR.png}
      \caption{正常洞調律(SR)}
      \label{fig:SR}
    \end{minipage} &
    \begin{minipage}[t]{0.45\hsize}
      \centering
      \includegraphics[width=1.0\columnwidth]{figure/PEA.png}
      \caption{無脈性電気活動(PEA)}
      \label{fig:PEA}
    \end{minipage} \\
 
    \begin{minipage}[t]{0.45\hsize}
      \centering
      \includegraphics[width=1.0\columnwidth]{figure/VT.png}
      \caption{心室頻拍(VT)}
      \label{fig:VT}
    \end{minipage} &
    \begin{minipage}[t]{0.45\hsize}
      \centering
      \includegraphics[width=1.0\columnwidth]{figure/ECG_success_10.png}
      \caption{心室細動(VF)}
      \label{fig:VF}
    \end{minipage} 
  \end{tabular}
\end{figure}

\subsection{無脈性電気活動(Pulseless Electorical Activity:PEA)}
無脈性電気活動は心電図波形が確認できるが脈は触れない状態である.PEA波形の一例を図\ref{fig:PEA}に示す.
PEAは除細動適用外波形であり,心肺蘇生法が有効である.除細動は心筋を脱分極させ,心静止にさせるのが目的であり,自発的な命令が回復し正常な波形に戻ることを期待する.
そのため,除細動は心静止にする必要がある場合のみ実施するため,無脈性電気活動は適用外である.

\subsection{心室頻拍(Ventricular Tachycardia:VT)}
心臓の動きが正常な場合,心臓を収縮させる命令は洞結線という場所から出されるが,心室頻拍では心室から命令が出ている状態である.
心室が非常に速く収縮し,心拍が1分間に200回を越える場合がある.心拍数が多すぎると空打ちの状態になり全身に血液を送れず意識を失う.図\ref{fig:VT}はVT波形の一例である.
波形に関しては心室が収縮するため,QRS波が多く確認できる.一定のリズムで同じ場所から命令が出されるため,波の形や高さが同じになる.
心室頻拍のうち,心拍が速く脈が触れず意識がないものを脈なしVTと呼び,これは除細動を必要とする波形である.

\subsection{心室細動(Ventricular Fibrillation:VF)}
心室細動とは心室全体で無秩序な痙攣が発生し,その結果心室が適当に細かく震えている状態である.すなわち,収縮することができず全身に血液を送ることができないため
意識は欠落し脈が触れなくなる.VF波形の一例を図\ref{fig:VF}に示す.心室が震えているため,波と波の間隔,波の幅や波の高さ,すべてに規則性がない.
心室細動は除細動を必要とする波形であるが,心肺停止状態長期化した場合,効果が得られない場合が報告されている.詳しくは2.5章で説明する.

\subsection{心室細動に対する除細動の効果}
\begin{figure}[t]
  \centering
  \includegraphics[scale=0.45]{figure/VF_1.png}
  \caption{除細動成功 \label{fig:VF_success}}
\end{figure}
\begin{figure}[t]
  \centering
  \includegraphics[scale=0.45]{figure/VF_2.png}
  \caption{VF再発 \label{fig:VF_saihatu}}
\end{figure}
\begin{figure}[t]
  \centering
  \includegraphics[scale=0.45]{figure/VF_3.png}
  \caption{除細動失敗 \label{fig:VF_sippai}}
\end{figure}
心室細動の状態によっては除細動の効果が得られない場合がある.除細動後の心電図波形3例を図\ref{fig:VF_success}$\sim$\ref{fig:VF_sippai}に示す.
図\ref{fig:VF_success}では心室細動に対して除細動を適用した後,SR波形が確認できる(成功).しかし,図\ref{fig:VF_saihatu}では除細動適用後に一時的にSRが確認できるが心室細動が再発している(VF再発).
また,図\ref{fig:VF_sippai}では除細動適用後も心室細動が続いており,効果が得られていない(失敗).本研究では図\ref{fig:VF_saihatu}と図\ref{fig:VF_sippai}のような状態を合わせて
除細動の効果が得られなかった失敗データとして扱う.

\subsection{本論文で用いる心電図波形}
\sloppy
本研究では,解析に用いる心電図波形として"Creighton University Ventricular Tachyarrthymia Database(CU)"を使用する\cite{ECG_data}.
CUはクレイトン大学心臓センター(Creighton University Cardiac Center)のF. M. Nolleによって収集されたものであり,
このデータベースには,持続性心室頻拍,心室粗動および,心室細動の発作を経験した被験者の記録が35件含まれている.
ただし,本研究で用いるデータの関係上,1人の心電図波形の記録から複数のデータを取得することに注意されたい.
また,信号については8分間のサンプリング周波数250[Hz]の記録である.表\ref{tab:VFdata}に本研究で使用する心電図波形のデータ数を示す.
\begin{table}[h]
  \centering
  \caption{VF波形データの分類と総数}
  \label{tab:VFdata}
  \begin{tabular}{|c|c|c|}
    \hline
    除細動 & データの詳細 & 総数 \\ \hline \hline
    効果あり & 成功 & 17 \\ \hline
    効果なし & VF再発: 11, 失敗: 18 & 29 \\ \hline
  \end{tabular}
\end{table}

\subsection{心電図波形の前処理}
本研究では,CUのVF波形のうち,解析対象を除細動直前の10秒間とする.また,解析するために前処理を行う.
前処理として,まず波形に対してスプライン補間を適用し,隣接するデータ点を基に曲線を用いて欠損部分を補完する.
次に,目的とするデータ成分を強調するためにハイパスフィルタとローパスフィルタを用いてノイズや不要な信号成分を除去する.
そして,オフセットを除去し,直流成分をカットし平均値を0にする.最後に,最小二乗法を用いて原信号から減算してトレンド成分を除去し,外れ値に対して中央値置換を行う.
本研究では,外れ値を中央値からの距離が中央絶対偏差(Mean Absolute Deviation:MAD)の3倍を越える値とする.MADは次式によって算出される.
\begin{equation}
  MAD = median(|x_i|-\bar{x})
\end{equation}
前処理の手順と数学的表現を要約すると次のようになる.ここで$x[i]$を処理前の信号,$y[i]$を処理後の信号とする.
\begin{enumerate}
  \item 欠損値に対しスプライン補間 \\
  $y[i]$はスプライン関数$S(x)$によって補完される.
  \begin{eqnarray}
    y[i] = 
    \begin{cases}
      x[i] & \text{if $x[i]$ is not missing} \\
      S(x[i]) & \text{if $x[i]$ is missing}
    \end{cases}
  \end{eqnarray}
  \item ベースライン除去
  \begin{itemize}
    \item ハイパスフィルタ(カットオフ周波数0.5[Hz])
    \item ローパスフィルタ(カットオフ周波数50[Hz])
  \end{itemize}
  \item 直流成分カット
  \begin{equation}
    y[i] = x[i] - \bar{x}
  \end{equation}
  \item トレンド除去 \\
  $a$,$b$は,最小二乗法によって求められた線形回帰の係数である.$a$は傾き,$b$は切片を表す.
  \begin{equation}
    y[i] = x[i] - (a\cdot i + b)
  \end{equation}
  \item 外れ値置換(中央値置換) \\
  ここで$\tilde{x}$は$x[i]$の中央値である.
  \begin{eqnarray}
    y[i] = 
    \begin{cases}
      x[i] & \text{if $x[i]$ is not an outlier} \\
      \tilde{x} & \text{if $x[i]$ is an outlier}
    \end{cases}
  \end{eqnarray}
\end{enumerate}



	%% 第3章 %%
	\newpage

\section{心電図波形の解析手法}
本章では,解析手法と特徴量抽出について説明する.

\subsection{連続ウェーブレット変換(Continuous Wavelet transform)}
ウェーブレット変換(Wavelet Transform:WT)は,関数をある形式に変換する手法であり,原信号の特徴を調べるのに適した形式に変換したり,もとのデータセットを簡潔に
記述することができるような形式に変換できる.ウェーブレット変換を行うには,局在した波形であるウェーブレット関数$\phi$が必要である.この関数を平行移動(トランスレーション)や
伸張(ダイレーション)という変換の過程を通じて,時間やスケールで関数を広げる操作を行い,別の形式に変換される.
ウェーブレット変換には連続ウェーブレット変換(Continuous Wavelet Transform:CWT)と離散ウェーブレット変換(Discrete Wavelet Transform:DWT)が存在するが,本研究では信号の時間変化を観測したいため,
次式のように定義される連続ウェーブレット変換を採用する\cite{SCI}.
\begin{align}
  W_{\phi}f(a,b) &\triangleq \frac{1}{\sqrt{a}}\int_{-\infty}^{\infty}f(t)\overline{\phi\bigg(\frac{t-b}{a}\bigg)}dt \label{f:wavelet}\\
  \phi(t) &\triangleq \frac{1}{\sqrt{2\pi\sigma^2}}e^{-\frac{t^2}{2\sigma^2}e^{jwt}} \label{f:m_wavelet}
\end{align}
ここで$f(t)$は原信号,$a$はスケールパラメータ(scale parameter),$b$はシフトパラメータ(shift parameter)と呼ばれる.
本研究では$a$は1-20[Hz]に対して、0.1刻みの200段階,bをサンプリング周波数250[Hz]$\times$10[sec]の計2500段階変化させ計算を行う。
また,$\phi(t)$はマザーウェーブレットであり,本研究では(\ref{f:m_wavelet})式のガボールウェーブレットを用いる.ここで$w$は基底角周波数,$\sigma$は正規分布の分散を表す.
ウェーブレット変換には時間の測定を正確にするほどスペクトルの計測の正確さが損なわれるハイゼンベルクの不確定性原理が存在するが,
ガウス関数の窓を持つ複素正弦波は,時間-周波数平面で最も小さい面積のハイゼンベルクの箱を持ち不確定性原理の影響を最小化することが知られている.
すなわち最適な時間-周波数のエネルギー分布を得ることが可能となる\cite{wavelet-fu}\cite{wevelet}.
なお,本研究では$w$,$\sigma$は$w = 4\pi$,$\sigma = 0.5$と実験的に決定した.

\subsection{スカログラム}
信号$x(t)$に含まれる全エネルギーはその信号の振幅の二乗積分として定義される.
\begin{eqnarray}
  E = \int_{-\infty}^{\infty}|x(t)|^2dt < \infty \label{f:enegy}
 \end{eqnarray}

(\ref{f:enegy})式は,信号は有限のエネルギーを持つことを意味する\cite{wevelet}.これより,あるスケール$a$と位置$b$での信号のエネルギーの相対値は,2次元ウェーブレットのエネルギー密度関数
によって与えられる.
\begin{eqnarray}
  E(a,b) = |W_{\phi}f(a,b)|^2
\end{eqnarray}
$E(a,b)$のプロットをスカログラム(Scalogram)と定義され,信号の主要なエネルギー的特徴を位置やスケールによって強調するものである.
VF波形(図\ref{fig:VF})のスカログラムを図\ref{fig:scalo}に示す.ここで横軸に縦軸に周波数[Hz],時間[sec]とし,正規化したエネルギーをカラーバーで示す.

% pseudo differential operator
\subsection{疑似微分作用素(Pseudo-Differential Operator)}
疑似微分作用素は主にフーリエ変換で使用され,次式で定義される\cite{JSCI2022}.
\begin{eqnarray}
  (\mathcal{F}f')(\epsilon) \triangleq -i\epsilon\hat{f}(\epsilon) \label{f:pseudo-1}
\end{eqnarray}
ここで$\mathcal{F}$をフーリエ変換,また$f'(t)$は$f(t)$の微分を表す.$\hat{f}(\epsilon)$はフーリエ変換後の信号である.
信号$f(t)$に疑似微分作用素を用いると式(\ref{f:pseudo-2})で表すことができる.
\begin{eqnarray}
  \Big(-\frac{d^2}{dt^2}+1\Big)^{\alpha}f(t) \label{f:pseudo-2}
\end{eqnarray}
疑似微分作用素は,二階微分と定数項1の組み合わせであり,$\alpha$は作用素の指数であり,次数を制御する.
この疑似微分作用素のフーリエ変換を
\begin{eqnarray}
  \hat{f}(\epsilon)\to(\epsilon^2+1)^\alpha\hat{f}(\epsilon) \label{f:pseudo-3}
\end{eqnarray}
と表すことができる.このように,疑似微分作用素をフーリエ変換の形に拡張することができ,次式の二つの関数を用いて計算することができる.
\begin{align}
  L\colon{a}\in{R}_+&\rightarrow{L(a)}\in{C} \label{f:pseudo-4-1} \\
  H\colon{y}\in{C}_+&\rightarrow{H(y)}\in{C} \label{f:pseudo-4-2}
\end{align}

$L$がウェーブレット変換における疑似微分作用素であり,$H$は非線形関数である.この二つの関数を用いて,疑似微分作用素を用いたウェーブレット変換における
エネルギー分布であるスカログラムを次式で導出することができる.
\begin{eqnarray}
  E(a,b)_{pdo} = H(L(a)\cdot{W_{\phi}f}(a,b)) \label{f:pseudo-5}
\end{eqnarray}
本研究では,文献\cite{JSCI2022}で提案されている$L(a)=\frac{1}{a}\text{,}H(y) = |y|^\frac{1}{4}$を採用する.
実際の疑似微分作用素を用いたウェーブレット変換のスカログラムの例を図\ref{fig:scalo_pdo}に示す.図\ref{fig:scalo},\ref{fig:scalo_pdo}を比較すると
通常のウェーブレット変換では得られない高周波帯のエネルギー(6Hz$\sim$20Hz)の特徴を得ることが可能となる.

\begin{figure}[t]
\centering
\begin{minipage}[b]{0.49\columnwidth}
    \centering
    \includegraphics[width=1.0\columnwidth]{figure/scalo_success_10.png}
    \caption{スカログラム$E(a,b)$ \label{fig:scalo}}
\end{minipage}
\begin{minipage}[b]{0.49\columnwidth}
    \centering
    \includegraphics[width=1.0\columnwidth]{figure/scalo_pdo_success_10.png}
    \caption{$L(a)=\frac{1}{a}, H(y)=|y|^\frac{1}{4}$におけるスカログラム$E_{pdo}(a,b)$ \label{fig:scalo_pdo}}
\end{minipage}
\end{figure}

%% NSI
\subsection{NSI(Normalized Spectrum Index)}
\begin{figure}[t]
  \centering
  \begin{minipage}[b]{0.49\columnwidth}
      \centering
      \includegraphics[width=1.0\columnwidth]{figure/NSI_success_10.png}
      \caption{図\ref{fig:scalo}のスカログラムによって得られた$NSI$ \label{fig:NSI}}
  \end{minipage}
  \begin{minipage}[b]{0.49\columnwidth}
      \centering
      \includegraphics[width=1.0\columnwidth]{figure/NSI_pdo_success_10.png}
      \caption{図\ref{fig:scalo_pdo}のスカログラムによって得られた$NSI_{pdo}$ \label{fig:NSI_pdo}}
  \end{minipage}
\end{figure}

NSI(Normalized Spectrum Index)はスカログラムにおける周波数に対するエネルギーの重心の変化を表したものであり次式で定義される.
\begin{equation}
  NSI(b) = \frac{\sum_{a}^{}E(a,b)f(a)}{\sum_{a}^{}E(a,b)} \label{f:NSI}
\end{equation}
$E(a,b)$はスカログラム,$f(a)$は$a$サンプル目の周波数を示す.本研究では,通常のウェーブレット変換と疑似微分作用素を用いたウェーブレット変換を用いるため
それぞれに対するスカログラムが導出できる.本研究では,次式のように疑似微分作用素を用いたスカログラムに対するNSIを導出する.
\begin{equation}
  NSI_{pdo}(b) = \frac{\sum_{a}^{}E_{pdo}(a,b)f(a)}{\sum_{a}^{}E_{pdo}(a,b)} \label{f:NSI_pdo}
\end{equation}
ここで,図\ref{fig:scalo},\ref{fig:scalo_pdo}のスカログラムに対するNSIを図\ref{fig:NSI},\ref{fig:NSI_pdo}に示す.

%% NTI
\subsection{NTI(Normalized Time Index)}
\begin{figure}[t]
  \centering
  \begin{minipage}[b]{0.49\columnwidth}
      \centering
      \includegraphics[width=1.0\columnwidth]{figure/NTI_success_10.png}
      \caption{図\ref{fig:scalo}のスカログラムによって得られた$NTI$ \label{fig:NTI}}
  \end{minipage}
  \begin{minipage}[b]{0.49\columnwidth}
      \centering
      \includegraphics[width=1.0\columnwidth]{figure/NTI_pdo_success_10.png}
      \caption{図\ref{fig:scalo_pdo}のスカログラムによって得られた$NTI_{pdo}$ \label{fig:NTI_pdo}}
  \end{minipage}
\end{figure}

NTI(Normalized Time Index)はスカログラムの時間軸におけるエネルギーの重心の動きを表したものであり,次式で求めることができる\cite{nature2023}.
\begin{align}
  NTI(a) &= \frac{\sum_{b}^{}E(a,b)T(b)}{\sum_{b}^{}E(a,b)} \label{f:NTI} \\
  NTI_{pdo}(a) &= \frac{\sum_{b}^{}E_{pdo}(a,b)T(b)}{\sum_{b}^{}E_{pdo}(a,b)} \label{f:NTI_pdo}
\end{align}
$E(a,b),E_{pdo}(a,b)$はスカログラム,$T(b)$は$b$番目の時刻を示す.NSIと同様にスカログラムから求められるNTIを図\ref{fig:NTI},\ref{fig:NTI_pdo}に示す. 
これまではNSIを用いて,スカログラムの周波数軸に対する特徴のみを抽出していた.NTIを使用することでこれまでに無かった時間軸に対する特徴を抽出することが可能になる.

%% SDW
\subsection{SDW(Scale Distribution Width)}
スカログラムのある時刻に着目することで,周波数とエネルギーの関連性を把握することが可能である.このとき,エネルギーの最大値の60\%を占めるピーク幅を
$SDW$(Scale Distribution Width)と定義する\cite{SDW}.$SDW$により,ある時刻においてエネルギー分布がどれだけ広がっているか,また集中しているか測ることが可能となる.図\ref{fig:SDW},\ref{fig:SDW_pdo}
にスカログラムから得られた$SDW$を示す.

\begin{figure}[t]
  \centering
  \begin{minipage}[b]{0.49\columnwidth}
      \centering
      \includegraphics[width=1.0\columnwidth]{figure/SDW_success_10.png}
      \caption{図\ref{fig:scalo}のスカログラムによって得られた$SDW$ \label{fig:SDW}}
  \end{minipage}
  \begin{minipage}[b]{0.49\columnwidth}
      \centering
      \includegraphics[width=1.0\columnwidth]{figure/P_SDW_success_10.png}
      \caption{図\ref{fig:scalo_pdo}のスカログラムによって得られた$SDW_{pdo}$ \label{fig:SDW_pdo}}
  \end{minipage}
\end{figure}

%% ポアンカレプロット解析
\subsection{ポアンカレプロット解析}
心電図波形より,時系列信号の変化を視覚的に表し,信号の複雑さと自己相似性を測ることができるポアンカレプロット解析を行う.
ポアンカレプロットの例を図\ref{fig:PP2}に示し,横軸に$i$番目の信号$x_i$,そして縦軸が$i+1$番目の信号$x_{i+1}$を表し,各点を$p_i(x_i,x_{i+1})$とする.
さらに,本研究では$p_i$,$p_{i+1}$の連続する点からユークリッド距離$l_i$を次式を用いて算出する.
\begin{equation}
  l_i = d(p_i,p_{i+1}) = \sqrt{(x_{i+1}-x_i)^2 + (x_{i+2}-x_{i+1})^2} \label{f:PP}
\end{equation}
図\ref{fig:PP}に実際に求められたユークリッド距離$l_i$を示す.

\begin{figure}[t]
  \centering
  \begin{minipage}[b]{0.49\columnwidth}
      \centering
      \includegraphics[width=1.0\columnwidth]{figure/PP2_success_10.png}
      \caption{図\ref{fig:VF_success}に対するポアンカレプロット \label{fig:PP2}}
  \end{minipage}
  \begin{minipage}[b]{0.49\columnwidth}
      \centering
      \includegraphics[width=1.0\columnwidth]{figure/PP_success_10.png}
      \caption{図\ref{fig:PP2}に対する$l_{i}$ \label{fig:PP}}
  \end{minipage}
\end{figure}

%% SHANON RENYI
\subsection{スペクトルエントロピー}
シャノンエントロピー(Shannon Entropy:$\mathcal{S}_e$)とレニーエントロピー(Renyi Entropy:$\mathcal{R}_e$)はてんかん患者の識別など脳波の分析で使用されている\cite{entoropy}.
\begin{figure}[t]
  \centering
  \begin{minipage}[b]{0.49\columnwidth}
      \centering
      \includegraphics[width=1.0\columnwidth]{figure/ShEn_success_10.png}
      \caption{図\ref{fig:scalo}のスカログラムによって得られた$\mathcal{S}_e$ \label{fig:shen}}
  \end{minipage}
  \begin{minipage}[b]{0.49\columnwidth}
      \centering
      \includegraphics[width=1.0\columnwidth]{figure/P_ShEn_success_10.png} 
      \caption{図\ref{fig:scalo_pdo}のスカログラムによって得られた${\mathcal{S}_e}_{pdo}$ \label{fig:shen_pdo}}
  \end{minipage}
\end{figure}
$\mathcal{S}_e$,$\mathcal{R}_e$は次式を用いて算出できる.
\begin{align}
  P(k,j) &= \frac{E(k,j)}{\sum_{j}^{}E(k,j)} \label{f:P(k,l)} \\
  P(k,j)_{pdo} &= \frac{E(k,j)_{pdo}}{\sum_{j}^{}E(k,j)_{pdo}} \label{f:P(k,j)_pdo} 
\end{align}
$E(k,j)$は時刻$k$,$j$サンプル目のエネルギー,$P(k,j)$は時刻$k$サンプル目のエネルギーの総和と,周波数$j$サンプル目の比を示す.
$\mathcal{S}_e$の算出方法を式(\ref{f:shen}),(\ref{f:shen_pdo}),$\mathcal{R}_e$を式(\ref{f:reen}),(\ref{f:reen_pdo})に示す.
\begin{align}
  \mathcal{S}_e &= -P(k,j)\sum_{j}^{}\text{log}_2P(k,j) \label{f:shen} \\
  {\mathcal{S}_e}_{pdo} &= -P(k,j)_{pdo}\sum_{j}^{}\text{log}_2P(k,j)_{pdo} \label{f:shen_pdo} \\
  \mathcal{R}_e &= \frac{1}{1-\alpha}\text{log}\sum_{j}^{}P(k,j)^\alpha \label{f:reen} \\
  {\mathcal{R}_e}_{pdo} &= \frac{1}{1-\alpha}\text{log}\sum_{j}^{}P(k,j)_{pdo}^\alpha \label{f:reen_pdo}
\end{align}
このとき,シャノンエントロピーを1パラメータ拡張したものがレニーエントロピーであり,本研究では先行研究\cite{SCI}と同様に$\alpha = 2$
を使用する.シャノンエントロピーを図\ref{fig:shen},\ref{fig:shen_pdo},レニーエントロピーを図\ref{fig:reen},\ref{fig:reen_pdo}に示す.

\begin{figure}[t]
  \centering
  \begin{minipage}[b]{0.49\columnwidth}
      \centering
      \includegraphics[width=1.0\columnwidth]{figure/ReEn_success_10.png}
      \caption{図\ref{fig:scalo}のスカログラムによって得られた$\mathcal{R}_e$ \label{fig:reen}}
  \end{minipage}
  \begin{minipage}[b]{0.49\columnwidth}
      \centering
      \includegraphics[width=1.0\columnwidth]{figure/P_ReEn_success_10.png}
      \caption{図\ref{fig:scalo_pdo}のスカログラムによって得られた${\mathcal{R}_e}_{pdo}$ \label{fig:reen_pdo}}
  \end{minipage}
\end{figure}

\subsection{スカログラムに基づく特徴量の抽出}
スカログラムからは4種類の特徴量抽出し検討する.以下にこれらの抽出方法を示す.
ここで,$j$は周波数サンプル番号,$k$は周波数サンプル番号,$E_{(k,j)}$は時間サンプル数$k$と周波数サンプル数$j$のときのエネルギーを示す.
\begin{itemize}
  \item エネルギー比率
\begin{eqnarray}
  \mathcal{H}_{\alpha,\beta} = \frac{P_{[\beta-20]}^F}{P_{[\alpha-20]}^F}
\end{eqnarray}
スカログラムのすべてのエネルギーに対して,特定の範囲の周波数を用いて比をとる.このとき,$P_{[\alpha-20]}^F$は$\alpha{\sim}20[Hz]$までの周波数の総エネルギーであり,
同様に$P_{[\beta-20]}^F$は$\beta{\sim}20{Hz}$を示す.本研究では$\alpha=0$,$\beta=5.8$とする.
\end{itemize}

\begin{itemize}
  \item NSIとピーク周波数との差の二乗平均
\begin{eqnarray}
  \epsilon^r = \frac{1}{N}\sum_{k=1}^{N}(NSI(k)-f_r(k))^2
\end{eqnarray}
$f_r(k)$は時刻$k$においてエネルギーが最大となる周波数を表す.このピークの周波数とNSI差を取り,二乗して時間平均をとったものを特徴量として抽出する.
\end{itemize}

\begin{itemize}
  \item 周波数による重みづけ
\begin{align}
  \mathcal{S}^w &= \sum_{k=1}^{N} \sum_{j=1}^{200} E^*_{(k,j)} f^{(k)}_j \\
  E^*_{(k,j)} &=
  \begin{cases}
    E^*_{(k,j)} - T^* & \text{if } E^*_{(k,j)} - T^* > 0, \\
    0 & \text{if } E^*_{(k,j)} - T^* \leq 0.
  \end{cases}
\end{align}
\end{itemize}
ここで,$f^{(k)}_j$は時間サンプル$k$における$j$サンプル目の周波数であり,
$E^*_{(k,j)}$は正規化されたスカログラムのエネルギーを示している.また,$T^*$は閾値であり,本研究では$T^*=0.6$とする.


\begin{itemize}
  \item スカログラムの時刻$k$におけるエネルギーの総和の最大値
\begin{eqnarray}
  P_{NNSM} = \max_{k}\{\sum_{j=1}^{200}E_{(k,j)},k = 1,2,\dots,N\}
\end{eqnarray}
\end{itemize}
この特徴量は各時間$k$におけるエネルギーのうち最大値を特徴量としたものである.

\begin{itemize}
  \item $SPP$(Spectral Pole Power)
\end{itemize}
$SPP$は,スカログラムのパワースペクトル密度(Power Spectral Density:PSD)によって求めることができる\cite{spp}.
本研究では,AR(Auto Regressive)モデルで波形を近似し,Yule Walker法を用いて信号の周波数成分におけるパワー分布を求める.
フーリエ変換から求める手法も存在するが,Yule Walker法を使用することでグラフ形状が平滑でスペクトルの可読性が高く好まれる.
$SPP$は次式によって算出することができる.
\begin{eqnarray}
  SPP = \sum_{l=1}^{N}f_{pole}(l){\cdot}P_{pole}(l) \label{f:SPP}
\end{eqnarray}
すなわち$SPP$は,ARモデルより求められる固有周波数$f_{pole}$と固有エネルギー$P_{pole}$によって求めることができる.

\subsection{特徴量抽出}
本研究では,心電図波形から得られるポアンカレプロット,また通常のウェーブレット変換と疑似微分作用を用いたウェーブレット変換を使用し,
得られるスカログラムから導出した$NSI$,$NTI$,$SDW$,$\mathcal{S}_e$,$\mathcal{R}_e$といった時系列信号に対して,基本統計量(平均,分散,標準偏差等)
を特徴量として採用した.以下にそれぞれの特徴量の抽出方法を示す.ここで時系列信号を$s(k)$とし,$k$を時間サンプル,$N$を時間サンプル数の総数とする.

\begin{itemize}
  \item 平均
\end{itemize}
\begin{eqnarray}
  \bar{s} = \frac{1}{N}\sum_{k=1}^{N}s(k) \label{f:mean}
\end{eqnarray}

\begin{itemize}
  \item 分散
\end{itemize}
\begin{eqnarray}
  \mathcal{V} = \frac{1}{N}\sum_{k=1}^{N}(s(k)-\bar{s})^2 \label{f:V}
\end{eqnarray}

\begin{itemize}
  \item 標準偏差
\end{itemize}
\begin{eqnarray}
  SD_s = \sqrt{\mathcal{V}} \label{f:sd}
\end{eqnarray}

\begin{itemize}
  \item エネルギー
\end{itemize}
\begin{eqnarray}
  \mathcal{P}_s = \sum_{k=1}^{N}s(k)^2 \label{f:P}
\end{eqnarray}

\begin{itemize}
  \item 累積の傾き
\end{itemize}
\begin{eqnarray}
  \mathcal{A}_s = \frac{1}{N-1}\sum_{k=1}^{N}|s(k+1)-s(k)| \label{f:A}
\end{eqnarray}

\newpage
\begin{itemize}
  \item 尖度
\end{itemize}
\begin{eqnarray}
  \mathcal{K}_s = \frac{1}{(\mathcal{V}_s)^3}\frac{1}{N}\sum_{k=1}^{N}(s(k)-\bar{s})^2 \label{f:K}
\end{eqnarray}

\begin{itemize}
  \item 歪度
\end{itemize}
\begin{eqnarray}
  \mathcal{SQ}_s = \frac{1}{(\mathcal{V}_s)^2}\frac{1}{N}\sum_{k=1}^{N}(s(k)-\bar{s})^4 \label{f:sq}
\end{eqnarray}

\begin{itemize}
  \item Entropy-Based Index
\end{itemize}
\begin{eqnarray}
  EBI_s = -\sum_{k=1}^{N}|s(k)|{\cdot}\text{log}_2|s(k)| \label{f:ebi}
\end{eqnarray}

\begin{itemize}
  \item 最頻値
\end{itemize}
$s(k)$の中で最も出現頻度が高い値を最頻値とする.

\begin{itemize}
  \item 中央値
\end{itemize}
$s'(k)$を$s(k)$を昇順に並べ替えた信号とするとき,次式で定義される .
\begin{eqnarray}
  \mathcal{Q}_s = 
  \begin{cases}
    \frac{1}{2}{s'\big(\frac{N}{2}\big)+s'\big(\frac{N}{2}+1\big)} & \text{if $N \equiv 0 \pmod 2$} \\
    s'\big(\frac{N+1}{2}\big) & \text{if $N \equiv 1 \pmod 2$}
  \end{cases}
\end{eqnarray}

加えて,各時系列信号$s(k)$の振幅を250段階に分割してヒストグラム$H_s(m),m = 1,2,...,250$を生成し,以下の3つの特徴量を抽出する.
\begin{itemize}
  \item ヒストグラムの平均
\end{itemize}
\begin{eqnarray}
  \bar{h_s} = \frac{1}{250}\sum_{m=1}^{250}H_s(m)
\end{eqnarray}

\newpage
\begin{itemize}
  \item ヒストグラムの分散
\end{itemize}
\begin{eqnarray}
  \mathcal{V_s}^h = \frac{1}{250}\sum_{m=1}^{250}(H_s(m)-\bar{h_s})^2
\end{eqnarray}

\begin{itemize}
  \item ヒストグラムの累積の傾き
\end{itemize}
\begin{eqnarray}
  \mathcal{A_s}^h = \frac{1}{250-1}\sum_{m=1}^{250-1}|H_s(m+1)-H_s(m)|
\end{eqnarray}

\subsection{抽出された特徴量}
以下に抽出された特徴量152種類とその表記を示す.
\begin{table}[h]
  \begin{center} \caption{ECGから得られる特徴量}
    \begin{tabular}{|l|c|c|} \hline
      No. & 特徴量 & 記号 \\ \hline \hline
      1 & ECGの平均 & $\overline{ECG}$ \\ \hline
      2 & ECGの分散 & $\mathcal{V}_{ECG}$ \\ \hline
      3 & ECGの累積の傾き & $\mathcal{A}_{ECG}$ \\ \hline
      4 & ECGの標準偏差 & $SD_{ECG}$ \\ \hline
      5 & ECGの尖度 & $\mathcal{SQ}_{ECG}$ \\ \hline
      6 & ECGの歪度 & $\mathcal{K}_{ECG}$ \\ \hline
      7 & ECGのエネルギー & $\mathcal{P}_{ECG}$ \\ \hline
      8 & ECGのEntoropy Based Index & $EBI_{ECG}$ \\ \hline
      9 & ECGの最頻値 & $\mathcal{M}_{ECG}$ \\ \hline
      10 & ECGのヒストグラムの分散 & ${\mathcal{V}_{ECG}}^h$ \\ \hline
      11 & ECGのヒストグラムの累積の傾き & ${\mathcal{A}}^h_{ECG}$ \\ \hline  
    \end{tabular}
  \end{center}
\end{table}

\begin{table}[p]
  \begin{center} \caption{NSIから得られる特徴量}
    \begin{tabular}{|l|c|c|} \hline
      No. & 特徴量 & 記号 \\ \hline \hline
      12 & NSIの平均 & $\overline{NSI}$ \\ \hline
      13 & NSIの分散 & $\mathcal{V}_{NSI}$ \\ \hline
      14 & NSIの標準偏差 & $SD_{NSI}$ \\ \hline
      15 & NSIの累積の傾き & $\mathcal{A}_{NSI}$ \\ \hline
      16 & NSIの尖度 & $\mathcal{SQ}_{NSI}$ \\ \hline
      17 & NSIの歪度 & $\mathcal{K}_{NSI}$ \\ \hline
      18 & NSIのエネルギー & $\mathcal{P}_{NSI}$ \\ \hline
      19 & NSIのEntoropy Based Index & $EBI_{NSI}$ \\ \hline
      20 & NSIの最頻値 & $\mathcal{M}_{NSI}$ \\ \hline
      21 & NSIの中央値 & $\mathcal{Q}_{NSI}$ \\ \hline
      22 & NSIのヒストグラムの分散 & ${\mathcal{V}}^h_{NSI}$ \\ \hline
      23 & NSIのヒストグラムの累積の傾き & ${\mathcal{A}}^h_{NSI}$ \\ \hline  
    \end{tabular}
  \end{center}
\end{table}

\begin{table}[p]
  \begin{center} \caption{NTIから得られる特徴量}
    \begin{tabular}{|l|c|c|} \hline
      No. & 特徴量 & 記号 \\ \hline \hline
      24 & NTIの平均 & $\overline{NTI}$ \\ \hline
      25 & NTIの分散 & $\mathcal{V}_{NTI}$ \\ \hline
      26 & NTIの標準偏差 & $SD_{NTI}$ \\ \hline
      27 & NTIの累積の傾き & $\mathcal{A}_{NTI}$ \\ \hline
      28 & NTIの尖度 & $\mathcal{SQ}_{NTI}$ \\ \hline
      29 & NTIの歪度 & $\mathcal{K}_{NTI}$ \\ \hline
      30 & NTIのエネルギー & $\mathcal{P}_{NTI}$ \\ \hline
      31 & NTIのEntoropy Based Index & $EBI_{NTI}$ \\ \hline
      32 & NTIの最頻値 & $\mathcal{M}_{NTI}$ \\ \hline
      33 & NTIの中央値 & $\mathcal{Q}_{NTI}$ \\ \hline
      34 & NTIのヒストグラムの分散 & ${\mathcal{V}}^h_{NTI}h$ \\ \hline
      35 & NTIのヒストグラムの累積の傾き & ${\mathcal{A}}^h_{NTI}$ \\ \hline  
    \end{tabular}
  \end{center}
\end{table}

\begin{table}[p]
  \begin{center} \caption{SDWから得られる特徴量}
    \begin{tabular}{|l|c|c|} \hline
      No. & 特徴量 & 記号 \\ \hline \hline
      36 & SDWの平均 & $\overline{SDW}$ \\ \hline
      37 & SDWの分散 & $\mathcal{V}_{SDW}$ \\ \hline
      38 & SDWの標準偏差 & $SD_{SDW}$ \\ \hline
      39 & SDWの累積の傾き & $\mathcal{A}_{SDW}$ \\ \hline
      40 & SDWの尖度 & $\mathcal{SQ}_{SDW}$ \\ \hline
      41 & SDWの歪度 & $\mathcal{K}_{SDW}$ \\ \hline
      42 & SDWのエネルギー & $\mathcal{P}_{SDW}$ \\ \hline
      43 & SDWのEntoropy Based Index & $EBI_{SDW}$ \\ \hline
      44 & SDWの最頻値 & $\mathcal{M}_{SDW}$ \\ \hline
      45 & SDWの中央値 & $\mathcal{Q}_{SDW}$ \\ \hline
      46 & SDWのヒストグラムの分散 & ${\mathcal{V}}^h_{SDW}$ \\ \hline
      47 & SDWのヒストグラムの累積の傾き & ${\mathcal{A}}^h_{SDW}$ \\ \hline  
    \end{tabular}
  \end{center}
\end{table}

\begin{table}[p]
  \begin{center} \caption{スカログラムから得られる特徴量}
    \begin{tabular}{|l|c|c|} \hline
      No. & 特徴量 & 記号 \\ \hline \hline
      48 & エネルギー比率$(\alpha = 0, \beta = 5.8)$ & $\mathcal{H}_{0,5.8}$ \\ \hline
      49 & NSIとのピーク周波数との差の二乗平均 & $\epsilon^r$ \\ \hline
      50 & 周波数による重みづけ & $\mathcal{S}^w$ \\ \hline
      51 & スカログラムの時間$k$におけるエネルギーの総和の最大値 & $P_{NNSM}$ \\ \hline
      52 & スカログラムにARモデルを用いたSPP & $SPP$ \\ \hline
    \end{tabular}
  \end{center}
\end{table}

\begin{table}[p]
  \begin{center} \caption{ポアンカレプロットによる$l_i$から得られる特徴量}
    \begin{tabular}{|l|c|c|} \hline
      No. & 特徴量 & 記号 \\ \hline \hline
      53 & $l_i$の平均 & $\overline{l_i}$ \\ \hline
      54 & $l_i$の分散 & $\mathcal{V}_{l_i}$ \\ \hline
      55 & $l_i$の累積の傾き & $\mathcal{A}_{l_i}$ \\ \hline
      56 & $l_i$の標準偏差 & $SD_{l_i}$ \\ \hline
      57 & $l_i$の尖度 & $\mathcal{SQ}_{l_i}$ \\ \hline
      58 & $l_i$の歪度 & $\mathcal{K}_{l_i}$ \\ \hline
      59 & $l_i$のエネルギー & $\mathcal{P}_{l_i}$ \\ \hline
      60 & $l_i$のEntoropy Based Index & $EBI_{l_i}$ \\ \hline
      61 & $l_i$の最頻値 & $\mathcal{M}_{l_i}$ \\ \hline
      62 & $l_i$の中央値 & $\mathcal{Q}_{l_i}$ \\ \hline
      63 & $l_i$のヒストグラムの分散 & ${\mathcal{V}}^h_{l_i}$ \\ \hline
      64 & $l_i$のヒストグラムの累積の傾き & ${\mathcal{A}}^h_{l_i}$ \\ \hline  
    \end{tabular}
  \end{center}
\end{table}

\begin{table}[p]
  \begin{center} \caption{$\mathcal{S}_e$から得られる特徴量}
    \begin{tabular}{|l|c|c|} \hline
      No. & 特徴量 & 記号 \\ \hline \hline
      65 & $\mathcal{S}_e$の平均 & $\overline{\mathcal{S}_e}$ \\ \hline
      66 & $\mathcal{S}_e$の分散 & $\mathcal{V}_{\mathcal{S}_e}$ \\ \hline
      67 & $\mathcal{S}_e$の標準偏差 & $SD_{\mathcal{S}_e}$ \\ \hline
      68 & $\mathcal{S}_e$の累積の傾き & $\mathcal{A}_{\mathcal{S}_e}$ \\ \hline
      69 & $\mathcal{S}_e$の尖度 & $\mathcal{SQ}_{\mathcal{S}_e}$ \\ \hline
      70 & $\mathcal{S}_e$の歪度 & $\mathcal{K}_{\mathcal{S}_e}$ \\ \hline
      71 & $\mathcal{S}_e$のエネルギー & $\mathcal{P}_{\mathcal{S}_e}$ \\ \hline
      72 & $\mathcal{S}_e$のEntoropy Based Index & $EBI_{\mathcal{S}_e}$ \\ \hline
      73 & $\mathcal{S}_e$の最頻値 & $\mathcal{M}_{\mathcal{S}_e}$ \\ \hline
      74 & $\mathcal{S}_e$の中央値 & $\mathcal{Q}_{\mathcal{S}_e}$ \\ \hline
      75 & $\mathcal{S}_e$のヒストグラムの分散 & ${\mathcal{V}}^h_{\mathcal{S}_e}$ \\ \hline
      76 & $\mathcal{S}_e$のヒストグラムの累積の傾き & ${\mathcal{A}}^h_{\mathcal{S}_e}$ \\ \hline  
    \end{tabular}
  \end{center}
\end{table}

\begin{table}[p]
  \begin{center} \caption{$\mathcal{R}_e$から得られる特徴量}
    \begin{tabular}{|l|c|c|} \hline
      No. & 特徴量 & 記号 \\ \hline \hline
      77 & $\mathcal{R}_e$の平均 & $\overline{\mathcal{R}_e}$ \\ \hline
      78 & $\mathcal{R}_e$の分散 & $\mathcal{V}_{\mathcal{R}_e}$ \\ \hline
      79 & $\mathcal{R}_e$の標準偏差 & $SD_{\mathcal{R}_e}$ \\ \hline
      80 & $\mathcal{R}_e$の累積の傾き & $\mathcal{A}_{\mathcal{R}_e}$ \\ \hline
      81 & $\mathcal{R}_e$の尖度 & $\mathcal{SQ}_{\mathcal{R}_e}$ \\ \hline
      82 & $\mathcal{R}_e$の歪度 & $\mathcal{K}_{\mathcal{R}_e}$ \\ \hline
      83 & $\mathcal{R}_e$のエネルギー & $\mathcal{P}_{\mathcal{R}_e}$ \\ \hline
      84 & $\mathcal{R}_e$のEntoropy Based Index & $EBI_{\mathcal{R}_e}$ \\ \hline
      85 & $\mathcal{R}_e$の最頻値 & $\mathcal{M}_{\mathcal{R}_e}$ \\ \hline
      86 & $\mathcal{R}_e$の中央値 & $\mathcal{Q}_{\mathcal{R}_e}$ \\ \hline
      87 & $\mathcal{R}_e$のヒストグラムの分散 & ${\mathcal{V}}^h_{\mathcal{R}_e}$ \\ \hline
      88 & $\mathcal{R}_e$のヒストグラムの累積の傾き & ${\mathcal{A}}^h_{\mathcal{R}_e}$ \\ \hline  
    \end{tabular}
  \end{center}
\end{table}

\begin{table}[p]
  \begin{center} \caption{$NSI_{pdo}$から得られる特徴量}
    \begin{tabular}{|l|c|c|} \hline
      No. & 特徴量 & 記号 \\ \hline \hline
      89 & $NSI_{pdo}$の平均 & $\overline{NSI_{pdo}}$ \\ \hline
      90 & $NSI_{pdo}$の分散 & $\mathcal{V}_{NSI_{pdo}}$ \\ \hline
      91 & $NSI_{pdo}$の標準偏差 & $SD_{NSI_{pdo}}$ \\ \hline
      92 & $NSI_{pdo}$の累積の傾き & $\mathcal{A}_{NSI_{pdo}}$ \\ \hline
      93 & $NSI_{pdo}$の尖度 & $\mathcal{SQ}_{NSI_{pdo}}$ \\ \hline
      94 & $NSI_{pdo}$の歪度 & $\mathcal{K}_{NSI_{pdo}}$ \\ \hline
      95 & $NSI_{pdo}$のエネルギー & $\mathcal{P}_{NSI_{pdo}}$ \\ \hline
      96 & $NSI_{pdo}$のEntoropy Based Index & $EBI_{NSI_{pdo}}$ \\ \hline
      97 & $NSI_{pdo}$の最頻値 & $\mathcal{M}_{NSI_{pdo}}$ \\ \hline
      98 & $NSI_{pdo}$の中央値 & $\mathcal{Q}_{NSI_{pdo}}$ \\ \hline
      99 & $NSI_{pdo}$のヒストグラムの分散 & ${\mathcal{V}}^h_{NSI_{pdo}}$ \\ \hline
      100 & $NSI_{pdo}$のヒストグラムの累積の傾き & ${\mathcal{A}}^h_{NSI_{pdo}}$ \\ \hline  
    \end{tabular}
  \end{center}
\end{table}

\begin{table}[p]
  \begin{center} \caption{$NTI_{pdo}$から得られる特徴量}
    \begin{tabular}{|l|c|c|} \hline
      No. & 特徴量 & 記号 \\ \hline \hline
      101 & $NTI_{pdo}$の平均 & $\overline{NTI_{pdo}}$ \\ \hline
      102 & $NTI_{pdo}$の分散 & $\mathcal{V}_{NTI_{pdo}}$ \\ \hline
      103 & $NTI_{pdo}$の標準偏差 & $SD_{NTI_{pdo}}$ \\ \hline
      104 & $NTI_{pdo}$の累積の傾き & $\mathcal{A}_{NTI_{pdo}}$ \\ \hline
      105 & $NTI_{pdo}$の尖度 & $\mathcal{SQ}_{NTI_{pdo}}$ \\ \hline
      106 & $NTI_{pdo}$の歪度 & $\mathcal{K}_{NTI_{pdo}}$ \\ \hline
      107 & $NTI_{pdo}$のエネルギー & $\mathcal{P}_{NTI_{pdo}}$ \\ \hline
      108 & $NTI_{pdo}$のEntoropy Based Index & $EBI_{NTI_{pdo}}$ \\ \hline
      109 & $NTI_{pdo}$の最頻値 & $\mathcal{M}_{NTI_{pdo}}$ \\ \hline
      110 & $NTI_{pdo}$の中央値 & $\mathcal{Q}_{NTI_{pdo}}$ \\ \hline
      111 & $NTI_{pdo}$のヒストグラムの分散 & ${\mathcal{V}}^h_{NTI_{pdo}}$ \\ \hline
      112 & $NTI_{pdo}$のヒストグラムの累積の傾き & ${\mathcal{A}}^h_{NTI_{pdo}}$ \\ \hline  
    \end{tabular}
  \end{center}
\end{table}

\begin{table}[p]
  \begin{center} \caption{$SDW_{pdo}$から得られる特徴量}
    \begin{tabular}{|l|c|c|} \hline
      No. & 特徴量 & 記号 \\ \hline \hline
      113 & $SDW_{pdo}$の平均 & $\overline{SDW_{pdo}}$ \\ \hline
      114 & $SDW_{pdo}$の分散 & $\mathcal{V}_{SDW_{pdo}}$ \\ \hline
      115 & $SDW_{pdo}$の標準偏差 & $SD_{SDW_{pdo}}$ \\ \hline
      116 & $SDW_{pdo}$の累積の傾き & $\mathcal{A}_{SDW_{pdo}}$ \\ \hline
      117 & $SDW_{pdo}$の尖度 & $\mathcal{SQ}_{SDW_{pdo}}$ \\ \hline
      118 & $SDW_{pdo}$の歪度 & $\mathcal{K}_{SDW_{pdo}}$ \\ \hline
      119 & $SDW_{pdo}$のエネルギー & $\mathcal{P}_{SDW_{pdo}}$ \\ \hline
      120 & $SDW_{pdo}$のEntoropy Based Index & $EBI_{SDW_{pdo}}$ \\ \hline
      121 & $SDW_{pdo}$の最頻値 & $\mathcal{M}_{SDW_{pdo}}$ \\ \hline
      122 & $SDW_{pdo}$の中央値 & $\mathcal{Q}_{SDW_{pdo}}$ \\ \hline
      123 & $SDW_{pdo}$のヒストグラムの分散 & ${\mathcal{V}}^h_{SDW_{pdo}}$ \\ \hline
      124 & $SDW_{pdo}$のヒストグラムの累積の傾き & ${\mathcal{A}}^h_{SDW_{pdo}}$ \\ \hline  
    \end{tabular}
  \end{center}
\end{table}

\begin{table}[p]
  \begin{center} \caption{疑似微分作用素のスカログラムから得られる特徴量}
    \begin{tabular}{|l|c|c|} \hline
      No. & 特徴量 & 記号 \\ \hline \hline
      125 & エネルギー比率$(\alpha = 0, \beta = 5.8)$ & ${\mathcal{H}_{pdo}}_{0,5.8}$ \\ \hline
      126 & NSIとのピーク周波数との差の二乗平均 & $\epsilon_{pdo}^r$ \\ \hline
      127 & 周波数による重みづけ & $\mathcal{S}_{pdo}^w$ \\ \hline
      128 & スカログラムの時間$k$におけるエネルギーの総和の最大値 & ${P_{pdo}}_{NNSM}$ \\ \hline
    \end{tabular}
  \end{center}
\end{table}

\newpage
\begin{table}[p]
  \begin{center} \caption{${\mathcal{S}_e}_{pdo}$から得られる特徴量}
    \begin{tabular}{|l|c|c|} \hline
      No. & 特徴量 & 記号 \\ \hline \hline
      129 & ${\mathcal{S}_e}_{pdo}$の平均 & $\overline{{\mathcal{S}_e}_{pdo}}$ \\ \hline
      130 & ${\mathcal{S}_e}_{pdo}$の分散 & $\mathcal{V}_{{\mathcal{S}_e}_{pdo}}$ \\ \hline
      131 & ${\mathcal{S}_e}_{pdo}$の標準偏差 & $SD_{{\mathcal{S}_e}_{pdo}}$ \\ \hline
      132 & ${\mathcal{S}_e}_{pdo}$の累積の傾き & $\mathcal{A}_{{\mathcal{S}_e}_{pdo}}$ \\ \hline
      133 & ${\mathcal{S}_e}_{pdo}$の尖度 & $\mathcal{SQ}_{{\mathcal{S}_e}_{pdo}}$ \\ \hline
      134 & ${\mathcal{S}_e}_{pdo}$の歪度 & $\mathcal{K}_{{\mathcal{S}_e}_{pdo}}$ \\ \hline
      135 & ${\mathcal{S}_e}_{pdo}$のエネルギー & $\mathcal{P}_{{\mathcal{S}_e}_{pdo}}$ \\ \hline
      136 & ${\mathcal{S}_e}_{pdo}$のEntoropy Based Index & $EBI_{{\mathcal{S}_e}_{pdo}}$ \\ \hline
      137 & ${\mathcal{S}_e}_{pdo}$の最頻値 & $\mathcal{M}_{{\mathcal{S}_e}_{pdo}}$ \\ \hline
      138 & ${\mathcal{S}_e}_{pdo}$の中央値 & $\mathcal{Q}_{{\mathcal{S}_e}_{pdo}}$ \\ \hline
      139 & ${\mathcal{S}_e}_{pdo}$のヒストグラムの分散 & ${\mathcal{V}}^h_{{\mathcal{S}_e}_{pdo}}$ \\ \hline
      140 & ${\mathcal{S}_e}_{pdo}$のヒストグラムの累積の傾き & ${\mathcal{A}}^h_{{\mathcal{S}_e}_{pdo}}$ \\ \hline  
    \end{tabular}
  \end{center}
\end{table}


\begin{table}[!t]
  \begin{center} \caption{${\mathcal{R}_e}_{pdo}$から得られる特徴量}
    \begin{tabular}{|l|c|c|} \hline
      No. & 特徴量 & 記号 \\ \hline \hline
      141 & ${\mathcal{R}_e}_{pdo}$の平均 & $\overline{{\mathcal{R}_e}_{pdo}}$ \\ \hline
      142 & ${\mathcal{R}_e}_{pdo}$の分散 & $\mathcal{V}_{{\mathcal{R}_e}_{pdo}}$ \\ \hline
      143 & ${\mathcal{R}_e}_{pdo}$の標準偏差 & $SD_{{\mathcal{R}_e}_{pdo}}$ \\ \hline
      144 & ${\mathcal{R}_e}_{pdo}$の累積の傾き & $\mathcal{A}_{{\mathcal{R}_e}_{pdo}}$ \\ \hline
      145 & ${\mathcal{R}_e}_{pdo}$の尖度 & $\mathcal{SQ}_{{\mathcal{R}_e}_{pdo}}$ \\ \hline
      146 & ${\mathcal{R}_e}_{pdo}$の歪度 & $\mathcal{K}_{{\mathcal{R}_e}_{pdo}}$ \\ \hline
      147 & ${\mathcal{R}_e}_{pdo}$のエネルギー & $\mathcal{P}_{{\mathcal{R}_e}_{pdo}}$ \\ \hline
      148 & ${\mathcal{R}_e}_{pdo}$のEntoropy Based Index & $EBI_{{\mathcal{R}_e}_{pdo}}$ \\ \hline
      149 & ${\mathcal{R}_e}_{pdo}$の最頻値 & $\mathcal{M}_{{\mathcal{R}_e}_{pdo}}$ \\ \hline
      150 & ${\mathcal{R}_e}_{pdo}$の中央値 & $\mathcal{Q}_{{\mathcal{R}_e}_{pdo}}$ \\ \hline
      151 & ${\mathcal{R}_e}_{pdo}$のヒストグラムの分散 & ${\mathcal{V}}^h_{{\mathcal{R}_e}_{pdo}}$ \\ \hline
      152 & ${\mathcal{R}_e}_{pdo}$のヒストグラムの累積の傾き & ${\mathcal{A}}^h_{{\mathcal{R}_e}_{pdo}}$ \\ \hline  
    \end{tabular}
  \end{center}
\end{table}

\clearpage
	%% 第4章 %%
	\newpage

\section{予測に有効な特徴量の選択}
3章で抽出された心電図波形やポアンカレプロット解析から得られる24種,また通常のウェーブレット変換から得られる特徴量64種に加え,疑似微分作用を用いたウェーブレット変換によって得られる特徴量64種,
合計152種の特徴量すべてを考慮して組み合わせを考えることは,計算量が指数関数的に増加するため非現実的である.そのため,本研究では予測に有効な特徴量を絞り込むために
ピアソンの$\chi^2$検定(Pearson's chi-square test)を用いる.

\subsection{$\chi^2$検定}
本章では,ピアソンの$\chi^2$検定について説明する.
$\chi^2$検定とは独立性の検定であり,2つのクラス間に関連があるか否かを判断するためのものである.
本来はカテゴリカルデータに対して行うものであるが,連続的な特徴量についてもビン化(離散化)することで有効となる.
帰無仮説$H_0$と対立仮説$H_1$は以下の通りとなり,帰無仮説を棄却することで,特徴量が除細動"成功"と"失敗"に関係性があるといえる\cite{chi}.
表\ref{tab:chi}に特徴量とクラスの分割表を示す.

\[
\left\{
\begin{array}{l}
  H_0 : \text{特徴量間は独立している(分割表の行と列は独立している)} \\
  H_1 : \text{特徴量間は関連している(分割表の行と列は独立していない)}
\end{array}
\right.
\]

\begin{table}[h]
  \begin{center} \caption{分割表} 
    \label{tab:chi}
    \begin{tabular}{|c|c|c|c|c|} \hline
      & 1$\sim$51 & 52$\sim$102 & 103$\sim$152 & 行和[個] \\ \hline \hline
      成功 & $C_{1,1}$ & $C_{1,2}$ & $C_{1,3}$ & $C_{.1}$\\ \hline
      失敗 & $C_{2,1}$ & $C_{2,2}$ & $C_{2,3}$ & $C_{.2}$\\ \hline
      列和 & $C_{1,.}$ & $C_{2,.}$ & $C_{3,.}$ & $C_{.,.}$\\ \hline
    \end{tabular}
  \end{center}
\end{table}

ここで$C_{i,j}$は観測値であり,表\ref{tab:chi}の行がクラス,列は特徴量の分割数を表す.
この表を基に$\chi^2$検定を行う.本研究における推定値は次式で求める.
\begin{eqnarray}
  E_{ij} = n\Big(\frac{C_i}{n}\Big)\Big(\frac{C_j}{n}\Big) = \frac{C_iC_j}{n} \label{f:E}
\end{eqnarray}
ここで,$n$は総データ数を示し観測値に対する$\chi^2$検定は次式で定義される.
\begin{eqnarray}
  {\chi}^2 \triangleq \sum_{i=1}^{r}\sum_{j=1}^{c}\frac{(C_{ij}-E{ij})^2}{E_{ij}} \label{f:chi}
\end{eqnarray}

データの平方和における$r$を総行数,$c$を総列数とすると,自由度$(r-1)\cdot(c-1)$の$\chi^2$分布に従うことが知られており,$\chi^2$検定で検定統計量を導出できる.この$\chi^2$検定の結果より,
帰無仮説が棄却される有意水準である$p$値$(0{\leq}p\leq1)$を求め,$p$値が事前に定められた有意水準より下回る場合,帰無仮説は棄却される.
本研究では次式でスコアを定義し,このとき,$p$値と自然対数(底$e$)を用いて評価することによって特徴量の選択を行う.
\begin{equation}
  Score = -\text{log}(p\text{-value}) \label{f:p-val}
\end{equation}

本研究では,$p$値を0.05と設定し$Score=3.00$を上回る特徴量を採用する.

\subsection{選択された特徴量}
図\ref{fig:chi}に$\chi^2$検定の結果を示す.ここで横軸は表2$\sim$15の各特徴量,縦軸は$Score$を表す.
また表\ref{tab:re-chi}に選択された特徴量とその記号を示す.
\begin{figure}[h]
  \centering
  \includegraphics[scale=0.7]{figure/Kai_result.png}
  \caption{${\chi}^2$検定結果 \label{fig:chi}}
\end{figure}

\begin{table}[h]
  \begin{center} \caption{${\chi}^2$検定を用いて選択した特徴量 \label{tab:re-chi}}
    \begin{tabular}{|l|c|c|c|} \hline
      No. & 特徴量 & 記号 & $Score$ \\ \hline \hline
      127 & 周波数による重みづけ& $\mathcal{S}_{pdo}^p$ & 6.54 \\ \hline
      92 & $NSI_{pdo}$の累積の傾き & $\mathcal{A}_{NSI_{pdo}}$ & 6.32\\ \hline
      31 & NTIのEntoropy Based Index & $EBI_{NTI}$ & 5.25\\ \hline
      24 & NTIの平均 & $\overline{NTI}$ & 5.22 \\ \hline
      30 & NTIのエネルギー & $\mathcal{P}_{NTI}$ & 5.22 \\ \hline
      34 & NTIのヒストグラムの分散 & ${\mathcal{V}}^h_{NTI}$ &5.22 \\ \hline
      22 & NSIのヒストグラムの分散 & ${\mathcal{V}}^h_{NSI}$ & 4.93 \\ \hline
      35 & NTIのヒストグラムの累積の傾き & ${\mathcal{A}}^h_{NTI}$ & 4.82 \\ \hline
      86 & ReEnの中央値 & $\mathcal{Q}_{\mathcal{R}_e}$ & 4.32 \\ \hline
      28 & NTIの尖度 & $\mathcal{SQ}_{NTI}$ & 4.07 \\ \hline
      33 & NTIの中央値 & $\mathcal{Q}_{NTI}$ & 4.07 \\ \hline
      48 & エネルギー比率 & $\mathcal{H}_{0,5.8}$ & 3.79 \\ \hline
      123 & $SDW_{pdo}$のヒストグラムの分散 & ${\mathcal{V}}^h_{SDW_{pdo}}$ & 3.50 \\ \hline
      52 & スカルグラムにARモデルを用いたSPP & $SPP$ & 3.50 \\ \hline
      94 & $NSI_{pdo}$の歪度& $\mathcal{K}_{NSI_{pdo}}$ & 3.50 \\ \hline
      16 & NSIの尖度 & $\mathcal{SQ}_{NSI}$ & 3.22 \\ \hline
      88 & ReEnのヒストグラムの累積の傾き & ${\mathcal{A}}^h_{\mathcal{R}_e}$ & 3.22 \\ \hline
      82 & ReEnの歪度 &  $\mathcal{K}_{\mathcal{R}_e}$ & 3.10\\ \hline
    \end{tabular}
  \end{center}
\end{table}

\clearpage
	%% 第5章 %%
	\newpage

\section{除細動効果予測システム}
\subsection{予測前処理}
\subsubsection{SMOTE法}
本論文の2.6節の表1で使用するデータ数を示したが,除細動失敗データ数が成功のデータ数より多く不均衡であるため,予測結果が偏る可能性が考えられる.
そのため,本研究ではSMOTE(Synthetic Minority Over-sampling Technique)法を使用し,クラス間のデータ数の差を無くし,検証を行う\cite{smote}.
SMOTEとはオーバーサンプリング手法の一つであり,少数クラスのデータに対してKNN(K-Nearest Neighbor Algorithm)の考えを使用し内挿する.
本研究では除細動成功のデータ数を失敗のデータ数である29個に揃えた.

\subsubsection{標準化}
本研究では,予測の前に標準化を行う.標準化はデータに対して分散を1,平均を0にする手法であり,次式で定義する.
\begin{eqnarray}
  {x^*}_i \triangleq \frac{{x_i}-\bar{x}}{\sigma} \label{f:stad5}
\end{eqnarray}
ここで$i$がサンプル数,$x_i$が$i$番目の特徴量,${x^*}_i$を$i$番目の標準化された特徴量,$\bar{x}$をデータの平均,$\sigma$を標準偏差とする.
標準化を行うことで,スケールの異なるデータが統一され比較可能になり,機械学習のモデルがよりデータを効率的に処理することが可能となる.

\subsection{サポートベクターマシン}
本研究では除細動の"成功"と"失敗"の予測を行うために,サポートベクターマシン(Support Vector Machine:SVM)を用いて2クラス分類を行う.
SVMは機械学習手法の一つで教師あり学習にあたり,分離超平面を使用しマージンを最大化することでデータを線形分離を行う.分離超平面とは$n$次元のデータを分類する$n-1$次元の平面であり,分離超平面
に対して最も距離が近いデータをサポートベクトル(Support Vector),距離のことをマージン(Margin)と呼ぶ.図\ref{fig:svm}にSVMのクラス間のマージンの概念図を示す.
ただし,現実のデータでは単純な直線や平面で分離できない場面が多くあり,この場合にはカーネルトリックと呼ばれる手法を用いて,元の特徴空間から高次元の特徴空間へとデータを写像し,
その高次元空間で線形分離を行う.この写像にカーネル関数と呼ばれる関数が使用され,本研究では多項式カーネル($2\sim5$次),線形カーネルおよびRBFカーネル(ガウシアンカーネル)を使用し,
マージンが最大になるような線形分離を行う.なお,高次元に写像しても線形分離不可能なデータに対しては正則化係数を導入することで柔軟に対応することができる.
本研究では,正則化係数$C(0.01, 0,1, 1, 10, 100, 1000)$の6種について検証を行う\footnote{正則化係数が小さい場合を誤分類を許容するソフトマージン,反対に誤差に対して厳しいものをハードマージンという.}.
正則化係数はモデルの学習の前に手動で設定するハイパーパラメータであり,値が大きいとマージンが狭くなり,訓練データに対して過度に適合し過学習が起こりやすくなる.

\begin{figure}[t]
  \centering
  \includegraphics[scale=0.5]{figure/svm.png}
  \caption{サポートベクターマシン\cite{svm} \label{fig:svm}}
\end{figure}

\subsection{Leave-One-Out Cross-Validation}
本研究では,予測に使用するデータの総数が58個と少ないことから,LOOCV(Leave-One-Out Cross-Validation)を使用する.
全データ$n$個のうち1つをテストデータ,他の$n-1$個を学習データとし,すべてのデータが一度はテストデータとして使用されるように,データ数の$n$回テストを行う手法である.
この手法の利点は,ほぼすべてのデータを学習データに使用できる点であり,ランダム性が排除され最終的な評価が常に同じになる.

\subsection{評価方法}
本研究の有用性を評価するために,感度(Sensitibity:$\mathcal{S}_e$),特異度(Specificity:$\mathcal{S}_p$)および正答率(Accuracy:$\mathcal{A}_c$)を評価指標として採用する.
\begin{table}[h]
  \centering
  \caption{混同行列}
  \begin{tabular}{|c||c|c|}
      \hline
      & \multicolumn{2}{c|}{予測結果} \\ \hline
      {実際の結果} & 陽性 & 陰性 \\ \hline\hline
      陽性 & TP (真陽性) & FN (偽陰性) \\ \hline
      陰性 & FP (偽陽性) & TN (真陰性) \\ \hline
  \end{tabular}
  \label{tab:confusion_matrix}
\end{table}

感度,特異度,正答率は次式で求めることができる.
\begin{align}
  \text{感度}(\mathcal{S}_e) &= \frac{TP}{TP+FN} \times 100 \\
  \text{特異度}(\mathcal{S}_p) &= \frac{TN}{FP+TN} \times 100 \\
  \text{正答率}(\mathcal{A}_c) &= \frac{TP+TN}{TP+FN+FP+TN} \times 100
\end{align}

感度と特異度は,臨床検査で使用される重要な指標であり\cite{kanndo},本研究では除細動が必要である場合を陽性,除細動を必要としない場合を陰性とする.
また,医療分野に関する研究では感度の精度が重要である.これは除細動が必要な波形に対して実際に適用した割合であり,
除細動が必要な波形を見過ごしてしまうことで重大な事故を引き起こす可能性があるためである.
	%% 第6章 %%
	%\input{result.tex}
	\newpage

\section{実験結果}
\subsection{除細動の効果予測}
本研究では,4.2節の$\chi^2$検定で選ばれた18個の特徴量の組み合わせと6種のカーネル関数,そして6種の正則化係数を用いて予測を行う.
予測する組み合わせの総数は次式のようになる.
\begin{eqnarray}
  SUM = 6\cdot6\cdot\sum_{n=1}^{18}{}_{18}C_n = 9,437,148
\end{eqnarray}

すべての特徴量の組み合わせのうち,精度上位3個を表\ref{tab:re}に示す.また,最良の精度を満たす特徴量と正則化係数の組み合わせが8通りあったため,
それらの組み合わせの結果を表\ref{tab:result}に示す.本研究における最良の結果としては感度100\%,特異度93.10\%,正答率96.55\%が得られた.
この結果は先行研究\cite{SCI}\cite{yosikawa}よりも高い正答率が得られた.
先行研究と比較すると,これまで最良の結果に対するサポートベクターマシンのカーネルは2次多項式が大半であり特徴量の組み合わせの数も5,6個であった.
一方で本研究では,2次$\sim$5次多項式のカーネルが使用され,特徴量の組み合わせとしてもこれまでになかった7$\sim$9個による精度が最良のものとなった.
また,選ばれた特徴量の組み合わせに関しては,本研究で新たに除細動の効果予測に加えた特徴量であるNTIがすべての組み合わせに含まれているため,
NTIが予測において有用な特徴量であるといえる.ただし,疑似微分作用素を用いたスカログラムから導出したNTIは$\chi^2$検定によって選定されなかったため,ウェーブレット変換に基づくNTIが予測に有効であると考えられる.
また,すべての組み合わせに疑似微分作用素から導出した特徴量が含まれているため,ウェーブレット変換に対する疑似微分作用素の有用性も示されたと考えられる.


\begin{table}[h]
  \centering
  \caption{予測精度}
  \begin{tabular}{|c||c|c|c|} \hline 
    順位 & 感度 $S_e$ & 特異度 $S_p$ & 正答率 $A_c$\\ \hline
    1 & 100 & 93.10 & 96.55 \\ \hline
    2 & 96.55 & 96.55 & 96.55 \\ \hline
    3 & 96.55 & 93.10 & 94.82 \\ \hline
  \end{tabular}
  \label{tab:re}
\end{table}

\begin{table}[!h]
  \centering
  \caption{予測結果}
  \begin{tabular}{|c|c|c|c|c|c|} 
      \hline
      No. & \multicolumn{3}{|c|}{特徴量} & カーネル & C  \\ \hline \hline
      \multirow{2}{*}{1} & $\mathcal{S}_{pdo}^p$ & ${\mathcal{V}}^h_{NSI}$ & ${\mathcal{A}}^h_{NTI}$ & \multirow{2}{*}{4次多項式} & \multirow{2}{*}{$1_{\text{(a)}}, 10_{\text{(b)}}, 100_{\text{(c)}}, 1000_{\text{(d)}}$}  \\ \cline{2-4}
      & $\mathcal{Q}_{NTI}$ & $\mathcal{K}_{NSI_{pdo}}$ & $\mathcal{K}_{\mathcal{R}_e}$ & & \\ \hline \hline

      \multirow{3}{*}{2} & $\mathcal{S}_{pdo}^p$ & ${\mathcal{V}}^h_{NSI}$ & ${\mathcal{A}}^h_{NTI}$ & \multirow{3}{*}{4次多項式} & \multirow{3}{*}{$1_{\text{(a)}}, 10_{\text{(b)}}, 100_{\text{(c)}}, 1000_{\text{(d)}}$}  \\ \cline{2-4}
      & $\mathcal{Q}_{NTI}$ & $\mathcal{K}_{NSI_{pdo}}$ & $\mathcal{SQ}_{NSI}$ & & \\ \cline{2-4}
      & $\mathcal{K}_{\mathcal{R}_e}$ & & & & \\ \hline \hline

      \multirow{3}{*}{3} & $\mathcal{S}_{pdo}^p$ & $EBI_{NTI}$ & ${\mathcal{V}}^h_{NTI}$ & \multirow{3}{*}{3次多項式} & \multirow{3}{*}{$1_{\text{(a)}}$}  \\ \cline{2-4}
      & ${\mathcal{V}}^h_{NSI}$ & ${\mathcal{A}}^h_{NTI}$ &  $\mathcal{SQ}_{NTI}$ & & \\ \cline{2-4}
      & $\mathcal{H}_{0,5.8}$ & $\mathcal{K}_{NSI_{pdo}}$ & & & \\ \hline \hline

      \multirow{3}{*}{4} &$\mathcal{S}_{pdo}^p$ & $EBI_{NTI}$ & ${\mathcal{V}}^h_{NSI}$ & \multirow{3}{*}{3次多項式} & \multirow{3}{*}{$10_{\text{(a)}}, 100_{\text{(b)}}, 1000_{\text{(c)}}$}  \\ \cline{2-4}
      & ${\mathcal{A}}^h_{NTI}$ & $\mathcal{SQ}_{NTI}$ & $\mathcal{H}_{0,5.8}$ & & \\ \cline{2-4}
      & $\mathcal{K}_{NSI_{pdo}}$ & $\mathcal{K}_{\mathcal{R}_e}$ & & & \\ \hline \hline

      \multirow{3}{*}{5} & $\mathcal{S}_{pdo}^p$ & $EBI_{NTI}$ & $\mathcal{P}_{NTI}$ & \multirow{3}{*}{5次多項式} & \multirow{3}{*}{$1_{\text{(a)}}, 10_{\text{(b)}}, 100_{\text{(c)}}, 1000_{\text{(d)}}$}  \\ \cline{2-4}
      & ${\mathcal{V}}^h_{NTI}$ & ${\mathcal{A}}^h_{NTI}$ & $\mathcal{SQ}_{NTI}$ & & \\ \cline{2-4}
      & $SPP$ & $\mathcal{K}_{NSI_{pdo}}$ & & & \\ \hline \hline

      \multirow{3}{*}{6} &$EBI_{NTI}$ & $\overline{NTI}$ & ${\mathcal{V}}^h_{NSI}$ & \multirow{3}{*}{2次多項式} & \multirow{3}{*}{$10_{\text{(a)}}, 100_{\text{(b)}}, 1000_{\text{(c)}}$}  \\ \cline{2-4}
      & $\mathcal{SQ}_{NTI}$ & $\mathcal{Q}_{NTI}$ & ${\mathcal{V}}^h_{SDW_{pdo}}$ & & \\ \cline{2-4}
      & $\mathcal{K}_{NSI_{pdo}}$ & $\mathcal{SQ}_{NSI}$ & $\mathcal{K}_{\mathcal{R}_e}$ & & \\ \hline \hline

      \multirow{3}{*}{7} & $\mathcal{S}_{pdo}^p$ & $EBI_{NTI}$ & ${\mathcal{V}}^h_{NSI}$ & \multirow{3}{*}{3次多項式} & \multirow{3}{*}{$10_{\text{(a)}}, 100_{\text{(b)}}, 1000_{\text{(c)}}$}  \\ \cline{2-4}
      & $\mathcal{Q}_{NTI}$ & $\mathcal{H}_{0,5.8}$ & ${\mathcal{V}}^h_{SDW_{pdo}}$ & & \\ \cline{2-4}
      & $\mathcal{K}_{NSI_{pdo}}$ & $\mathcal{SQ}_{NSI}$ & $\mathcal{K}_{\mathcal{R}_e}$ & & \\ \hline \hline

      \multirow{3}{*}{8} & $\mathcal{S}_{pdo}^p$ & $EBI_{NTI}$ & $\overline{NTI}$ & \multirow{3}{*}{4次多項式} & \multirow{3}{*}{$1_{\text{(a)}}, 10_{\text{(b)}}, 100_{\text{(c)}}, 1000_{\text{(d)}}$}  \\ \cline{2-4}
      & ${\mathcal{V}}^h_{NSI}$ & $\mathcal{Q}_{NTI}$ & $\mathcal{H}_{0,5.8}$ & & \\ \cline{2-4}
      & ${\mathcal{V}}^h_{SDW_{pdo}}$ & $\mathcal{K}_{NSI_{pdo}}$ & $\mathcal{K}_{\mathcal{R}_e}$ & & \\ \hline
  \end{tabular}
  \label{tab:result}
\end{table}

\clearpage

\subsection{予測処理時間の測定}
\begin{table}[h]
  \centering
  \caption{マシンスペック}
  \begin{tabular}{|c|c|} \hline 
    OS & Windows 11\\ \hline
    CPU & 11th Gen Intel(R) Core(TM) i5-1135G7 @ 2.40GHz   2.42 GHz\\ \hline
    GPU & Intel(R) lris(R) Xe Graphics\\ \hline
    RAM & 8.00 GB \\ \hline
    ソフトウェア & MATLAB R2024a,Python 3.10.7\\ \hline
  \end{tabular}
  \label{tab:machine-spe}
\end{table}



\begin{table}[h]
\centering
\caption{予測時間}
\begin{minipage}{0.45\textwidth}\label{tab:pre-time}
\centering
  \begin{tabular}{|c|c|c|} \hline 
    No. & C & time[sec]\\ \hline \hline
    \multirow{4}{*}{1} & (a) & 0.5320 \\ \cline{2-3}
    & (b) & 0.5037 \\ \cline{2-3}
    & (c) & 0.5335 \\ \cline{2-3}
    & (d) & 0.5375 \\ \hline \hline

    \multirow{4}{*}{2} & (a) & 0.5865 \\ \cline{2-3}
    & (b) & 0.5804 \\ \cline{2-3}
    & (c) & 0.5524 \\ \cline{2-3}
    & (d) & 0.4460 \\ \hline \hline

    3 & (a) & 0.4793 \\ \hline \hline

    \multirow{3}{*}{4} & (a) & 0.6283 \\ \cline{2-3}
    & (b) & 0.6231 \\ \cline{2-3}
    & (c) & 0.6508 \\ \hline
  \end{tabular}
\end{minipage}
\begin{minipage}[t]{0.45\textwidth}
\centering
  \begin{tabular}{|c|c|c|} \hline
    \multirow{4}{*}{5} & (a) & 1.6446 \\ \cline{2-3}
    & (b) & 1.6344 \\ \cline{2-3}
    & (c) & 1.4788 \\ \cline{2-3}
    & (d) & 1.5233 \\ \hline \hline

    \multirow{3}{*}{6} & (a) & 0.5803 \\ \cline{2-3}
    & (b) & 0.4974 \\ \cline{2-3}
    & (c) & 0.6320 \\ \hline \hline

    \multirow{3}{*}{7} & (a) & 0.5869 \\ \cline{2-3}
    & (b) & 0.4433 \\ \cline{2-3}
    & (c) & 0.4741 \\ \hline \hline

    \multirow{4}{*}{8} & (a) & 0.5795 \\ \cline{2-3}
    & (b) & 0.6412 \\ \cline{2-3}
    & (c) & 0.6444 \\ \cline{2-3}
    & (d) & 0.5193 \\ \hline
  \end{tabular}
\end{minipage}
\end{table}

表\ref{tab:result}の特徴量の組み合わせを用いて,データの前処理から特徴量を抽出し,予測結果を得るまでに,予測に使用する特徴量の組み合わせによって
計算時間に差があるか検証を行う.検証方法としては,除細動の"成功"と"失敗"を合わせた46個のデータからランダムに1データを抜き出し,選ばれたデータに対して波形の前処理を行い特徴量を抽出した後,
学習させたモデルを用いて予測を行う.この前処理から予測が完了するまでに要した時間を計測する.
計測に使用したマシンスペックを表\ref{tab:machine-spe},それぞれの組み合わせに対する計測結果を表\ref{tab:pre-time}に示す.

表\ref{tab:pre-time}より,5(a)$\sim$5(d)を除いたすべてが$0.5\sim0.6$[s]付近であり,差が生じなかった.
また,致死性不整脈であるVF波形の読み取りから2[s]かからず判定結果を出力することが行えているため,迅速に心臓の状態に適した処置
が可能であり,本研究の有用性が確認されたと考える.
	%% 第7章 %%
	\newpage

\section{結論と今後の課題}
本研究では,除細動直前10秒間のVF波形に対して除細動が有効か否か判定するアルゴリズムを構築し検証を行った.
特に,除細動の"成功"と"失敗"について検討し,失敗については除細動の効果が得られない"失敗"と除細動後に一度自己心拍を再開するものの,再度VFに陥る"再発"の2状態を合わせて定義した.


提案手法としては,除細動直前の10秒の心電図波形に対してウェーブレット変換と疑似微分作用素を用いたウェーブレット変換を適用し,スカログラムを導出することにより特徴量を抽出した.
また,スカログラムから得られる$NSI$,$NTI$,$SDW$,スペクトルエントロピーから特徴量を抽出し,加えて心電図波形から得られるポアンカレプロット解析からも特徴量を抽出した.
ただし,すべての特徴量を考慮し予測を行うことは困難であるため,$\chi^2$検定で検証に有効である特徴量の絞り込みを行い,SVMで3種のカーネル関数と6種の正則化係数を使用して除細動の"成功"と"失敗"の予測を行った.
また,予測システムの有用性を評価するために,感度,特異度,正答率を用いた.
結果としては,感度100\%,特異度93.10\%,正答率96.55\%が得られた.先行研究\cite{SCI}\cite{yosikawa}と比較して,高い正答率を得ることができたため,本研究の手法が有効であると考えられる.

今後の課題としては,本研究では$\chi^2$検定で特徴量の絞り込みを行っているが,絞り込みを行わず,すべての特徴量を考慮し組み合わせを考える必要がある.
予測精度として正答率96.55\%を得ることができたが,実用には不十分である.特徴量すべてを用いることで今回得られた結果を上回る精度が期待される.
%本研究の目的として,高精度な予測を行うことであるため,絞り込みを行わないことでより目的に近づけると考えられる.
また,疑似微分作用素の関数$L(a)$と$H(y)$について検討する必要がある.本研究では,文献\cite{JSCI2022}で提案されている$L(a)=\frac{1}{a}, H(y)=|y|^{\frac{1}{4}}$を採用しており,
これらは暫定的な値として設定されている.よって,VF波形の除細動の効果予測に有効であるパラメータについて検討し,最良の関数の組み合わせを見つけることで,より高精度なシステムの構築につながると考えられる.
%さらに,除細動の失敗については,"失敗"と"再発"の2状態が混合しているため,この2状態の識別を目的とした予測を行い,除細動の"成功","失敗","再発"の3状態について検討する必要があるといえる.
%3状態の予測を行うことで,より患者の状態に合わせた救命措置が可能となり,蘇生率向上に寄与できると思われる.
	%% 謝辞 %%
	\newpage

\section*{謝辞}
本研究を進めるにあたり,終始丁寧なご指導を賜りました大屋 英稔 教授および星 義克 講師に深謝の意を表します.
また,日頃より多方面にわたり貴重なご助言と温かいご協力をいただきました大学院生の野口瞬太さん,ならびに自動制御研究室の皆様に心より御礼申し上げます.
さらに,研究室の同期には,多くの刺激を受け,濃密な研究室生活を過ごすことができました.この場を借りて深く感謝申し上げるとともに,皆さまの今後のご活躍を心よりお祈り申し上げます.	
	%% 参考文献 %%
	\newpage

\begin{thebibliography}{9}
  \bibitem{people} 総務省消防庁, "令和5年版 救急救助の現状," https://www.fdma.go.jp/publication/rescue/post-5.html, (参照 2024-11-2)

  \bibitem{AED} 公共財団法人 日本AED財団, "心臓突然死の現状," \\
   https://aed-zaidan.jp/knowledge/index.html

  \bibitem{count} M.Hasegawa, T.Abe, T.Nagata, D.Onozauka, and A.Hagihara, "The Number of Prehospital Defibrillation Shocks and 1-month Survival in Patients 
  with out-of-Hospital Cardiac Arrest," Sucandinavian Journal of Trauma, Resuscitation and Emeergency Medicine, Vol.23, No.1, pp.1-12, 2015

  \bibitem{JSCI2022} Md. Masudur Rahman, Toshinao Kagawa, Shuji Kawasaki, Shunya Nagai, Takayuki Okai, Hidetoshi Oya, Yumi Yahagi and Minoru W. Yoshida, 
  ``Various scalographic representation of  electrocardiograms  through wavelet transform with pseudo-differential operators like operators,"  
  Journal of Advanced Simulation in Science and Engineering, Vol.9, No.1, pp.96-112, 2022
      
  \bibitem{nature2023} Md. Masudur Rahman, Sergio Albeverio, Toshinao Kagawa, Shuji Kawasaki, Takayuki Okai, Hidetoshi Oya, Yumi Yahagi and Minoru W. Yoshida, 
  ``High accuracy distinction of shockable and non-shockable arrhythmias in abnormal classes through wavelet transform with pseudo differential like operators," 
  Scientific Reports, Vol.13, 2023, https://doi.org/10.1038/s41598-023-36463-z

  \bibitem{SCI} 野口,吉川,岡井,大屋,星, "特徴量の時間遷移に着目した要除細動波形の特徴解析と電気的除細動の効果予測," 
  第68回 システム制御情報学会研究発表講演会講演論文集, pp.1051-1056, 2024

  \bibitem{yosikawa} Yuta Yoshikawa, Yoshihiro Ogino, Takayuki Okai, Hidetoshi Oya, Yoshikatsu Hoshi and Kazushi Nakano, 
  "Prediction of the effect of electrical defibrillation by using spectral feature parameters," Computers in Biology and Medicine, Vol.182, 109123, 2024

  \bibitem{QRS} 布施 淳, "救急心電図," 中外医学社, 2018

  \bibitem{ECG} 佐藤弘明, "レジデントのためのこれだけ心電図," 日本医事新報社, 第1版, 2018

  \bibitem{ECG_data} F. M. Nolle, F. K. Badura, J. M. Catlett, R. W. Bowser, M. H. Sketch, 
  "CREI-GARD a new concept in computerized arrhythmia monitoring systems," Computers in Cardiology Vol.13, pp.515-518, 1986

  \bibitem{wavelet-fu} 新井 仁之, "フーリエ解析とウェーブレット," 朝倉書店, 2022

  \bibitem{wevelet} 新 誠一, 中野和司, "図説ウェーブレット変換ハンドブック," 朝倉書店, 新装版, 2022

  \bibitem{SDW} K. Umapathy, S. Krishnan, S. Masse, X. Hu, P. Dorian and K. Nanthakumar, "Optimizing Cardiac Resuscitation Outcomes using Wavelet Analysis," 
  31st Annual International Conference of the IEEE EMBS Minneapolis, pp.6761-6764, 2009

  \bibitem{PP} R. Jain and R. A. Ganeasn, "Pointcaré plot analysis for sleep-Wake Classification of Unseen Patients Using a Single EEG channel," 
  17th IEEE International Symposium On Medical Measurements and Applications pp.1-6, 2022

  \bibitem{entoropy} M. N. Tibdewal, H. R. Dey, M. Manjunatha, A. Ray and M. Malokar, "Multiple entropies performance measure for detection and localization of multi-channel epileptic EEG,"  
  Elsevier, Biomedical Signal Processing and Control Vol.38, pp.158-167, 2017

  \bibitem{spp} CN Nowak, G Fischer, A Neurauter, L Wieser, B Tilg, HU Strohmenger, "Comparion of Countershock Predicion Features based on Autoregressive and Fourier transformed Spectral Analysis,"
  Proc. of IEEE COmputers in Cardiolody, pp.485-488, 2008

  \bibitem{chi} 栗原 伸一, "入門統計学 第2版," オーム社, 2021

  \bibitem{kanndo} 矢冨 裕, 横田 浩充, "臨床検査医学総論," 医学書院, 第一版, 2012

  \bibitem{smote} N. V. Chawla, K. W. bowyer, L. O. Hall and W. P. Kegelmeyer, "SMOTE: Synthetic Minority Over-sampling Technique,"  
  Journal of Artificial Intelligence Research, Vol.16, pp.321-357, 2002

  \bibitem{svm} サポートベクターマシン (SVM) - MATLAB Simulink - MathWorks, \\
  https://jp.mathworks.com/discovery/support-vector-machine.html, (参照:2024-11-2)
\end{thebibliography}
\end{document}
