\newpage

\section{除細動効果予測システム}
\subsection{予測前処理}
\subsubsection{SMOTE法}
本論文の2.6節の表1で使用するデータ数を示したが,除細動失敗データ数が成功のデータ数より多く不均衡であるため,予測結果が偏る可能性が考えられる.
そのため,本研究ではSMOTE(Synthetic Minority Over-sampling Technique)法を使用し,クラス間のデータ数の差を無くし,検証を行う\cite{smote}.
SMOTEとはオーバーサンプリング手法の一つであり,少数クラスのデータに対してKNN(K-Nearest Neighbor Algorithm)の考えを使用し内挿する.
本研究では除細動成功のデータ数を失敗のデータ数である29個に揃えた.

\subsubsection{標準化}
本研究では,予測の前に標準化を行う.標準化はデータに対して分散を1,平均を0にする手法であり,次式で定義する.
\begin{eqnarray}
  {x^*}_i \triangleq \frac{{x_i}-\bar{x}}{\sigma} \label{f:stad5}
\end{eqnarray}
ここで$i$がサンプル数,$x_i$が$i$番目の特徴量,${x^*}_i$を$i$番目の標準化された特徴量,$\bar{x}$をデータの平均,$\sigma$を標準偏差とする.
標準化を行うことで,スケールの異なるデータが統一され比較可能になり,機械学習のモデルがよりデータを効率的に処理することが可能となる.

\subsection{サポートベクターマシン}
本研究では除細動の"成功"と"失敗"の予測を行うために,サポートベクターマシン(Support Vector Machine:SVM)を用いて2クラス分類を行う.
SVMは機械学習手法の一つで教師あり学習にあたり,分離超平面を使用しマージンを最大化することでデータを線形分離を行う.分離超平面とは$n$次元のデータを分類する$n-1$次元の平面であり,分離超平面
に対して最も距離が近いデータをサポートベクトル(Support Vector),距離のことをマージン(Margin)と呼ぶ.図\ref{fig:svm}にSVMのクラス間のマージンの概念図を示す.
ただし,現実のデータでは単純な直線や平面で分離できない場面が多くあり,この場合にはカーネルトリックと呼ばれる手法を用いて,元の特徴空間から高次元の特徴空間へとデータを写像し,
その高次元空間で線形分離を行う.この写像にカーネル関数と呼ばれる関数が使用され,本研究では多項式カーネル($2\sim5$次),線形カーネルおよびRBFカーネル(ガウシアンカーネル)を使用し,
マージンが最大になるような線形分離を行う.なお,高次元に写像しても線形分離不可能なデータに対しては正則化係数を導入することで柔軟に対応することができる.
本研究では,正則化係数$C(0.01, 0,1, 1, 10, 100, 1000)$の6種について検証を行う\footnote{正則化係数が小さい場合を誤分類を許容するソフトマージン,反対に誤差に対して厳しいものをハードマージンという.}.
正則化係数はモデルの学習の前に手動で設定するハイパーパラメータであり,値が大きいとマージンが狭くなり,訓練データに対して過度に適合し過学習が起こりやすくなる.

\begin{figure}[t]
  \centering
  \includegraphics[scale=0.5]{figure/svm.png}
  \caption{サポートベクターマシン\cite{svm} \label{fig:svm}}
\end{figure}

\subsection{Leave-One-Out Cross-Validation}
本研究では,予測に使用するデータの総数が58個と少ないことから,LOOCV(Leave-One-Out Cross-Validation)を使用する.
全データ$n$個のうち1つをテストデータ,他の$n-1$個を学習データとし,すべてのデータが一度はテストデータとして使用されるように,データ数の$n$回テストを行う手法である.
この手法の利点は,ほぼすべてのデータを学習データに使用できる点であり,ランダム性が排除され最終的な評価が常に同じになる.

\subsection{評価方法}
本研究の有用性を評価するために,感度(Sensitibity:$\mathcal{S}_e$),特異度(Specificity:$\mathcal{S}_p$)および正答率(Accuracy:$\mathcal{A}_c$)を評価指標として採用する.
\begin{table}[h]
  \centering
  \caption{混同行列}
  \begin{tabular}{|c||c|c|}
      \hline
      & \multicolumn{2}{c|}{予測結果} \\ \hline
      {実際の結果} & 陽性 & 陰性 \\ \hline\hline
      陽性 & TP (真陽性) & FN (偽陰性) \\ \hline
      陰性 & FP (偽陽性) & TN (真陰性) \\ \hline
  \end{tabular}
  \label{tab:confusion_matrix}
\end{table}

感度,特異度,正答率は次式で求めることができる.
\begin{align}
  \text{感度}(\mathcal{S}_e) &= \frac{TP}{TP+FN} \times 100 \\
  \text{特異度}(\mathcal{S}_p) &= \frac{TN}{FP+TN} \times 100 \\
  \text{正答率}(\mathcal{A}_c) &= \frac{TP+TN}{TP+FN+FP+TN} \times 100
\end{align}

感度と特異度は,臨床検査で使用される重要な指標であり\cite{kanndo},本研究では除細動が必要である場合を陽性,除細動を必要としない場合を陰性とする.
また,医療分野に関する研究では感度の精度が重要である.これは除細動が必要な波形に対して実際に適用した割合であり,
除細動が必要な波形を見過ごしてしまうことで重大な事故を引き起こす可能性があるためである.