\newpage

\section{実験結果}
\subsection{除細動の効果予測}
本研究では,4.2節の$\chi^2$検定で選ばれた18個の特徴量の組み合わせと6種のカーネル関数,そして6種の正則化係数を用いて予測を行う.
予測する組み合わせの総数は次式のようになる.
\begin{eqnarray}
  SUM = 6\cdot6\cdot\sum_{n=1}^{18}{}_{18}C_n = 9,437,148
\end{eqnarray}

すべての特徴量の組み合わせのうち,精度上位3個を表\ref{tab:re}に示す.また,最良の精度を満たす特徴量と正則化係数の組み合わせが8通りあったため,
それらの組み合わせの結果を表\ref{tab:result}に示す.本研究における最良の結果としては感度100\%,特異度93.10\%,正答率96.55\%が得られた.
この結果は先行研究\cite{SCI}\cite{yosikawa}よりも高い正答率が得られた.
先行研究と比較すると,これまで最良の結果に対するサポートベクターマシンのカーネルは2次多項式が大半であり特徴量の組み合わせの数も5,6個であった.
一方で本研究では,2次$\sim$5次多項式のカーネルが使用され,特徴量の組み合わせとしてもこれまでになかった7$\sim$9個による精度が最良のものとなった.
また,選ばれた特徴量の組み合わせに関しては,本研究で新たに除細動の効果予測に加えた特徴量であるNTIがすべての組み合わせに含まれているため,
NTIが予測において有用な特徴量であるといえる.ただし,疑似微分作用素を用いたスカログラムから導出したNTIは$\chi^2$検定によって選定されなかったため,ウェーブレット変換に基づくNTIが予測に有効であると考えられる.
また,すべての組み合わせに疑似微分作用素から導出した特徴量が含まれているため,ウェーブレット変換に対する疑似微分作用素の有用性も示されたと考えられる.


\begin{table}[h]
  \centering
  \caption{予測精度}
  \begin{tabular}{|c||c|c|c|} \hline 
    順位 & 感度 $S_e$ & 特異度 $S_p$ & 正答率 $A_c$\\ \hline
    1 & 100 & 93.10 & 96.55 \\ \hline
    2 & 96.55 & 96.55 & 96.55 \\ \hline
    3 & 96.55 & 93.10 & 94.82 \\ \hline
  \end{tabular}
  \label{tab:re}
\end{table}

\begin{table}[!h]
  \centering
  \caption{予測結果}
  \begin{tabular}{|c|c|c|c|c|c|} 
      \hline
      No. & \multicolumn{3}{|c|}{特徴量} & カーネル & C  \\ \hline \hline
      \multirow{2}{*}{1} & $\mathcal{S}_{pdo}^p$ & ${\mathcal{V}}^h_{NSI}$ & ${\mathcal{A}}^h_{NTI}$ & \multirow{2}{*}{4次多項式} & \multirow{2}{*}{$1_{\text{(a)}}, 10_{\text{(b)}}, 100_{\text{(c)}}, 1000_{\text{(d)}}$}  \\ \cline{2-4}
      & $\mathcal{Q}_{NTI}$ & $\mathcal{K}_{NSI_{pdo}}$ & $\mathcal{K}_{\mathcal{R}_e}$ & & \\ \hline \hline

      \multirow{3}{*}{2} & $\mathcal{S}_{pdo}^p$ & ${\mathcal{V}}^h_{NSI}$ & ${\mathcal{A}}^h_{NTI}$ & \multirow{3}{*}{4次多項式} & \multirow{3}{*}{$1_{\text{(a)}}, 10_{\text{(b)}}, 100_{\text{(c)}}, 1000_{\text{(d)}}$}  \\ \cline{2-4}
      & $\mathcal{Q}_{NTI}$ & $\mathcal{K}_{NSI_{pdo}}$ & $\mathcal{SQ}_{NSI}$ & & \\ \cline{2-4}
      & $\mathcal{K}_{\mathcal{R}_e}$ & & & & \\ \hline \hline

      \multirow{3}{*}{3} & $\mathcal{S}_{pdo}^p$ & $EBI_{NTI}$ & ${\mathcal{V}}^h_{NTI}$ & \multirow{3}{*}{3次多項式} & \multirow{3}{*}{$1_{\text{(a)}}$}  \\ \cline{2-4}
      & ${\mathcal{V}}^h_{NSI}$ & ${\mathcal{A}}^h_{NTI}$ &  $\mathcal{SQ}_{NTI}$ & & \\ \cline{2-4}
      & $\mathcal{H}_{0,5.8}$ & $\mathcal{K}_{NSI_{pdo}}$ & & & \\ \hline \hline

      \multirow{3}{*}{4} &$\mathcal{S}_{pdo}^p$ & $EBI_{NTI}$ & ${\mathcal{V}}^h_{NSI}$ & \multirow{3}{*}{3次多項式} & \multirow{3}{*}{$10_{\text{(a)}}, 100_{\text{(b)}}, 1000_{\text{(c)}}$}  \\ \cline{2-4}
      & ${\mathcal{A}}^h_{NTI}$ & $\mathcal{SQ}_{NTI}$ & $\mathcal{H}_{0,5.8}$ & & \\ \cline{2-4}
      & $\mathcal{K}_{NSI_{pdo}}$ & $\mathcal{K}_{\mathcal{R}_e}$ & & & \\ \hline \hline

      \multirow{3}{*}{5} & $\mathcal{S}_{pdo}^p$ & $EBI_{NTI}$ & $\mathcal{P}_{NTI}$ & \multirow{3}{*}{5次多項式} & \multirow{3}{*}{$1_{\text{(a)}}, 10_{\text{(b)}}, 100_{\text{(c)}}, 1000_{\text{(d)}}$}  \\ \cline{2-4}
      & ${\mathcal{V}}^h_{NTI}$ & ${\mathcal{A}}^h_{NTI}$ & $\mathcal{SQ}_{NTI}$ & & \\ \cline{2-4}
      & $SPP$ & $\mathcal{K}_{NSI_{pdo}}$ & & & \\ \hline \hline

      \multirow{3}{*}{6} &$EBI_{NTI}$ & $\overline{NTI}$ & ${\mathcal{V}}^h_{NSI}$ & \multirow{3}{*}{2次多項式} & \multirow{3}{*}{$10_{\text{(a)}}, 100_{\text{(b)}}, 1000_{\text{(c)}}$}  \\ \cline{2-4}
      & $\mathcal{SQ}_{NTI}$ & $\mathcal{Q}_{NTI}$ & ${\mathcal{V}}^h_{SDW_{pdo}}$ & & \\ \cline{2-4}
      & $\mathcal{K}_{NSI_{pdo}}$ & $\mathcal{SQ}_{NSI}$ & $\mathcal{K}_{\mathcal{R}_e}$ & & \\ \hline \hline

      \multirow{3}{*}{7} & $\mathcal{S}_{pdo}^p$ & $EBI_{NTI}$ & ${\mathcal{V}}^h_{NSI}$ & \multirow{3}{*}{3次多項式} & \multirow{3}{*}{$10_{\text{(a)}}, 100_{\text{(b)}}, 1000_{\text{(c)}}$}  \\ \cline{2-4}
      & $\mathcal{Q}_{NTI}$ & $\mathcal{H}_{0,5.8}$ & ${\mathcal{V}}^h_{SDW_{pdo}}$ & & \\ \cline{2-4}
      & $\mathcal{K}_{NSI_{pdo}}$ & $\mathcal{SQ}_{NSI}$ & $\mathcal{K}_{\mathcal{R}_e}$ & & \\ \hline \hline

      \multirow{3}{*}{8} & $\mathcal{S}_{pdo}^p$ & $EBI_{NTI}$ & $\overline{NTI}$ & \multirow{3}{*}{4次多項式} & \multirow{3}{*}{$1_{\text{(a)}}, 10_{\text{(b)}}, 100_{\text{(c)}}, 1000_{\text{(d)}}$}  \\ \cline{2-4}
      & ${\mathcal{V}}^h_{NSI}$ & $\mathcal{Q}_{NTI}$ & $\mathcal{H}_{0,5.8}$ & & \\ \cline{2-4}
      & ${\mathcal{V}}^h_{SDW_{pdo}}$ & $\mathcal{K}_{NSI_{pdo}}$ & $\mathcal{K}_{\mathcal{R}_e}$ & & \\ \hline
  \end{tabular}
  \label{tab:result}
\end{table}

\clearpage

\subsection{予測処理時間の測定}
\begin{table}[h]
  \centering
  \caption{マシンスペック}
  \begin{tabular}{|c|c|} \hline 
    OS & Windows 11\\ \hline
    CPU & 11th Gen Intel(R) Core(TM) i5-1135G7 @ 2.40GHz   2.42 GHz\\ \hline
    GPU & Intel(R) lris(R) Xe Graphics\\ \hline
    RAM & 8.00 GB \\ \hline
    ソフトウェア & MATLAB R2024a,Python 3.10.7\\ \hline
  \end{tabular}
  \label{tab:machine-spe}
\end{table}



\begin{table}[h]
\centering
\caption{予測時間}
\begin{minipage}{0.45\textwidth}\label{tab:pre-time}
\centering
  \begin{tabular}{|c|c|c|} \hline 
    No. & C & time[sec]\\ \hline \hline
    \multirow{4}{*}{1} & (a) & 0.5320 \\ \cline{2-3}
    & (b) & 0.5037 \\ \cline{2-3}
    & (c) & 0.5335 \\ \cline{2-3}
    & (d) & 0.5375 \\ \hline \hline

    \multirow{4}{*}{2} & (a) & 0.5865 \\ \cline{2-3}
    & (b) & 0.5804 \\ \cline{2-3}
    & (c) & 0.5524 \\ \cline{2-3}
    & (d) & 0.4460 \\ \hline \hline

    3 & (a) & 0.4793 \\ \hline \hline

    \multirow{3}{*}{4} & (a) & 0.6283 \\ \cline{2-3}
    & (b) & 0.6231 \\ \cline{2-3}
    & (c) & 0.6508 \\ \hline
  \end{tabular}
\end{minipage}
\begin{minipage}[t]{0.45\textwidth}
\centering
  \begin{tabular}{|c|c|c|} \hline
    \multirow{4}{*}{5} & (a) & 1.6446 \\ \cline{2-3}
    & (b) & 1.6344 \\ \cline{2-3}
    & (c) & 1.4788 \\ \cline{2-3}
    & (d) & 1.5233 \\ \hline \hline

    \multirow{3}{*}{6} & (a) & 0.5803 \\ \cline{2-3}
    & (b) & 0.4974 \\ \cline{2-3}
    & (c) & 0.6320 \\ \hline \hline

    \multirow{3}{*}{7} & (a) & 0.5869 \\ \cline{2-3}
    & (b) & 0.4433 \\ \cline{2-3}
    & (c) & 0.4741 \\ \hline \hline

    \multirow{4}{*}{8} & (a) & 0.5795 \\ \cline{2-3}
    & (b) & 0.6412 \\ \cline{2-3}
    & (c) & 0.6444 \\ \cline{2-3}
    & (d) & 0.5193 \\ \hline
  \end{tabular}
\end{minipage}
\end{table}

表\ref{tab:result}の特徴量の組み合わせを用いて,データの前処理から特徴量を抽出し,予測結果を得るまでに,予測に使用する特徴量の組み合わせによって
計算時間に差があるか検証を行う.検証方法としては,除細動の"成功"と"失敗"を合わせた46個のデータからランダムに1データを抜き出し,選ばれたデータに対して波形の前処理を行い特徴量を抽出した後,
学習させたモデルを用いて予測を行う.この前処理から予測が完了するまでに要した時間を計測する.
計測に使用したマシンスペックを表\ref{tab:machine-spe},それぞれの組み合わせに対する計測結果を表\ref{tab:pre-time}に示す.

表\ref{tab:pre-time}より,5(a)$\sim$5(d)を除いたすべてが$0.5\sim0.6$[s]付近であり,差が生じなかった.
また,致死性不整脈であるVF波形の読み取りから2[s]かからず判定結果を出力することが行えているため,迅速に心臓の状態に適した処置
が可能であり,本研究の有用性が確認されたと考える.