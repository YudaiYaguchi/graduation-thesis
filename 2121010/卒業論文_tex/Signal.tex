\newpage
\section{心電図波形}
本章では心電図波形\cite{ECG},解析に用いた心電図波形のデータベース,および信号の前処理について示す.
また,本論文における心電図波形は横軸に時間[sec],そして縦軸に電位[mV]をとる.

\begin{figure}[t]
  \centering
  \includegraphics[scale=0.7]{figure/QRS.png}
  \caption{正常洞調律\cite{QRS} \label{fig:QRS}}
\end{figure}

\subsection{正常洞調律(Sinus Rhythm:SR)}
正常洞調律とは電気的興奮が正しく反復され,心臓の拍動が一定のリズムを保つ状態である.SRの図解を図\ref{fig:QRS},SR波形の一例を図\ref{fig:SR}に示す.
SRでは最初に小さな波P波が起こり次にQ波,R波,S波が続き,これらをまとめてQRS波と呼ぶ.心臓の動きとしては,P波で心房の収縮が起こり,QRS波で心室の収縮が行われる.
それぞれ心電図波形を読みとるのに重要な波形であり,T波が診断の決め手になる心電図波形は多くない.また,SRは心臓が正常に機能している状態を示すため除細動適用すべきでない波形である.

\begin{figure}[t]
  \begin{tabular}{cc}
    \begin{minipage}[t]{0.45\hsize}
      \centering
      \includegraphics[width=1.0\columnwidth]{figure/SR.png}
      \caption{正常洞調律(SR)}
      \label{fig:SR}
    \end{minipage} &
    \begin{minipage}[t]{0.45\hsize}
      \centering
      \includegraphics[width=1.0\columnwidth]{figure/PEA.png}
      \caption{無脈性電気活動(PEA)}
      \label{fig:PEA}
    \end{minipage} \\
 
    \begin{minipage}[t]{0.45\hsize}
      \centering
      \includegraphics[width=1.0\columnwidth]{figure/VT.png}
      \caption{心室頻拍(VT)}
      \label{fig:VT}
    \end{minipage} &
    \begin{minipage}[t]{0.45\hsize}
      \centering
      \includegraphics[width=1.0\columnwidth]{figure/ECG_success_10.png}
      \caption{心室細動(VF)}
      \label{fig:VF}
    \end{minipage} 
  \end{tabular}
\end{figure}

\subsection{無脈性電気活動(Pulseless Electorical Activity:PEA)}
無脈性電気活動は心電図波形が確認できるが脈は触れない状態である.PEA波形の一例を図\ref{fig:PEA}に示す.
PEAは除細動適用外波形であり,心肺蘇生法が有効である.除細動は心筋を脱分極させ,心静止にさせるのが目的であり,自発的な命令が回復し正常な波形に戻ることを期待する.
そのため,除細動は心静止にする必要がある場合のみ実施するため,無脈性電気活動は適用外である.

\subsection{心室頻拍(Ventricular Tachycardia:VT)}
心臓の動きが正常な場合,心臓を収縮させる命令は洞結線という場所から出されるが,心室頻拍では心室から命令が出ている状態である.
心室が非常に速く収縮し,心拍が1分間に200回を越える場合がある.心拍数が多すぎると空打ちの状態になり全身に血液を送れず意識を失う.図\ref{fig:VT}はVT波形の一例である.
波形に関しては心室が収縮するため,QRS波が多く確認できる.一定のリズムで同じ場所から命令が出されるため,波の形や高さが同じになる.
心室頻拍のうち,心拍が速く脈が触れず意識がないものを脈なしVTと呼び,これは除細動を必要とする波形である.

\subsection{心室細動(Ventricular Fibrillation:VF)}
心室細動とは心室全体で無秩序な痙攣が発生し,その結果心室が適当に細かく震えている状態である.すなわち,収縮することができず全身に血液を送ることができないため
意識は欠落し脈が触れなくなる.VF波形の一例を図\ref{fig:VF}に示す.心室が震えているため,波と波の間隔,波の幅や波の高さ,すべてに規則性がない.
心室細動は除細動を必要とする波形であるが,心肺停止状態長期化した場合,効果が得られない場合が報告されている.詳しくは2.5章で説明する.

\subsection{心室細動に対する除細動の効果}
\begin{figure}[t]
  \centering
  \includegraphics[scale=0.45]{figure/VF_1.png}
  \caption{除細動成功 \label{fig:VF_success}}
\end{figure}
\begin{figure}[t]
  \centering
  \includegraphics[scale=0.45]{figure/VF_2.png}
  \caption{VF再発 \label{fig:VF_saihatu}}
\end{figure}
\begin{figure}[t]
  \centering
  \includegraphics[scale=0.45]{figure/VF_3.png}
  \caption{除細動失敗 \label{fig:VF_sippai}}
\end{figure}
心室細動の状態によっては除細動の効果が得られない場合がある.除細動後の心電図波形3例を図\ref{fig:VF_success}$\sim$\ref{fig:VF_sippai}に示す.
図\ref{fig:VF_success}では心室細動に対して除細動を適用した後,SR波形が確認できる(成功).しかし,図\ref{fig:VF_saihatu}では除細動適用後に一時的にSRが確認できるが心室細動が再発している(VF再発).
また,図\ref{fig:VF_sippai}では除細動適用後も心室細動が続いており,効果が得られていない(失敗).本研究では図\ref{fig:VF_saihatu}と図\ref{fig:VF_sippai}のような状態を合わせて
除細動の効果が得られなかった失敗データとして扱う.

\subsection{本論文で用いる心電図波形}
\sloppy
本研究では,解析に用いる心電図波形として"Creighton University Ventricular Tachyarrthymia Database(CU)"を使用する\cite{ECG_data}.
CUはクレイトン大学心臓センター(Creighton University Cardiac Center)のF. M. Nolleによって収集されたものであり,
このデータベースには,持続性心室頻拍,心室粗動および,心室細動の発作を経験した被験者の記録が35件含まれている.
ただし,本研究で用いるデータの関係上,1人の心電図波形の記録から複数のデータを取得することに注意されたい.
また,信号については8分間のサンプリング周波数250[Hz]の記録である.表\ref{tab:VFdata}に本研究で使用する心電図波形のデータ数を示す.
\begin{table}[h]
  \centering
  \caption{VF波形データの分類と総数}
  \label{tab:VFdata}
  \begin{tabular}{|c|c|c|}
    \hline
    除細動 & データの詳細 & 総数 \\ \hline \hline
    効果あり & 成功 & 17 \\ \hline
    効果なし & VF再発: 11, 失敗: 18 & 29 \\ \hline
  \end{tabular}
\end{table}

\subsection{心電図波形の前処理}
本研究では,CUのVF波形のうち,解析対象を除細動直前の10秒間とする.また,解析するために前処理を行う.
前処理として,まず波形に対してスプライン補間を適用し,隣接するデータ点を基に曲線を用いて欠損部分を補完する.
次に,目的とするデータ成分を強調するためにハイパスフィルタとローパスフィルタを用いてノイズや不要な信号成分を除去する.
そして,オフセットを除去し,直流成分をカットし平均値を0にする.最後に,最小二乗法を用いて原信号から減算してトレンド成分を除去し,外れ値に対して中央値置換を行う.
本研究では,外れ値を中央値からの距離が中央絶対偏差(Mean Absolute Deviation:MAD)の3倍を越える値とする.MADは次式によって算出される.
\begin{equation}
  MAD = median(|x_i|-\bar{x})
\end{equation}
前処理の手順と数学的表現を要約すると次のようになる.ここで$x[i]$を処理前の信号,$y[i]$を処理後の信号とする.
\begin{enumerate}
  \item 欠損値に対しスプライン補間 \\
  $y[i]$はスプライン関数$S(x)$によって補完される.
  \begin{eqnarray}
    y[i] = 
    \begin{cases}
      x[i] & \text{if $x[i]$ is not missing} \\
      S(x[i]) & \text{if $x[i]$ is missing}
    \end{cases}
  \end{eqnarray}
  \item ベースライン除去
  \begin{itemize}
    \item ハイパスフィルタ(カットオフ周波数0.5[Hz])
    \item ローパスフィルタ(カットオフ周波数50[Hz])
  \end{itemize}
  \item 直流成分カット
  \begin{equation}
    y[i] = x[i] - \bar{x}
  \end{equation}
  \item トレンド除去 \\
  $a$,$b$は,最小二乗法によって求められた線形回帰の係数である.$a$は傾き,$b$は切片を表す.
  \begin{equation}
    y[i] = x[i] - (a\cdot i + b)
  \end{equation}
  \item 外れ値置換(中央値置換) \\
  ここで$\tilde{x}$は$x[i]$の中央値である.
  \begin{eqnarray}
    y[i] = 
    \begin{cases}
      x[i] & \text{if $x[i]$ is not an outlier} \\
      \tilde{x} & \text{if $x[i]$ is an outlier}
    \end{cases}
  \end{eqnarray}
\end{enumerate}


