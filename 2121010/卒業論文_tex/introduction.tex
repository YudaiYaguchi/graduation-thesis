\newpage
\section{序論}

\subsection{研究背景}
\begin{figure}[b]
  \centering
  \includegraphics[scale=0.5]{figure/hospital.png}
  \caption{現場到着までの所要時間及び病院収容までの時間の推移 \label{fig:hospital}}
\end{figure}
我が国では日々多くの人が突然の心停止が原因で命を落としている.令和4年度における心肺停止傷病者数は約14万人であり,そのうち心原生かつ心肺機能停止が一般市民によって目撃された数は2万8千人に及ぶ\cite{people}.
主な原因は心室細動(Ventricular Fibrillation:VF)と呼ばれる致死性不整脈である.心室細動に陥ると心臓が無秩序な痙攣によって血液を送り出すことができなくなり,数秒で意識を失い数分以内に脳を含む全身の細胞が死に至る\cite{AED}.
近年の傾向としては,図\ref{fig:hospital}より救急隊の到着及び病院収容時間は年々増加していることがわかる.また,図\ref{fig:recovry},\ref{fig:survival}より社会復帰率,生存率が減少傾向にあることが見てとれる.
心室細動からの救命には迅速な早期の電気的除細動(以下,除細動)が有効であることが知られており,現場に居合わせた一般人(バイスタンダー)による応急手当が極めて重要である.
実際,図\ref{fig:die}より除細動を実施した傷病者1229人の1ヶ月後の生存率は50.3\%であり,実施しなかった傷病者数11,766人の生存率と比較すると7.6倍もの差があることが明らかである.
このため,除細動を行うための自動体外式除細動器(Automated External Defibrillator:AED)が空港や商業施設等に配備されており,AEDの販売台数としても毎年増加している.
しかし,近年VFによる心肺停止が長期化した場合,除細動が効果的でなく自己心拍を再開しない場合があることが報告されており,
この場合には心肺蘇生法(CardioPulmonary Resuscitation:CPR)や投薬などの措置を施すことで蘇生率が向上する可能性がある.
これらのことから,除細動を実施する前の心電図波形(ElectroCardioGram:ECG)を解析し,解析結果から除細動が有効に機能するか否かを予測する研究が進められている.

\begin{figure}[t]
  \centering
  \includegraphics[scale=0.7]{figure/recovry.png}
  \caption{社会復帰率の推移\cite{people} \label{fig:recovry}}
\end{figure}

\begin{figure}[t]
  \centering
  \includegraphics[scale=0.7]{figure/survival.png}
  \caption{生存率の推移\cite{people} \label{fig:survival}}
\end{figure}

\begin{figure}[t]
  \centering
  \includegraphics[scale=0.6]{figure/die.png}
  \caption{応急手当の実施及び救命効果\cite{people} \label{fig:die}}
\end{figure}
  

\subsection{従来研究}
これらを踏まえて,心電図波形解析に関する研究が多く行われている.
1.1節で述べた通り,VFに対して除細動は効果的である.しかし,除細動は身体に大きな負荷がかかるため,回数が増加すると救命率が低下することが示されている\cite{count}.
そのため,患者の状態に応じて除細動を適用することが望ましく,除細動を必要とする心電図波形と除細動を適用してはいけない波形を識別する研究が行われている\cite{JSCI2022}\cite{nature2023}.
解析対象の心電図波形は,除細動を必要としない正常洞調律(Sinus Rhythm:SR),無脈性電気活動(Pulseless Electorical Activity:PEA),および必要である心室頻拍(Ventricular Tachycardia:VT),心室細動(Ventricular Fibrillation:VF)
であり,これらの分類を行う手法が提案されている.
文献\cite{JSCI2022}では疑似微分作用素を用いたウェーブレット変換による波形解析手法を提案しており,通常のウェーブレット変換では得られない特徴を捉えられることが示されている.
%ただし,疑似微分作用素にはいくつかの設定すべきパラメータがあり,除細動の識別における最良の組み合わせを示されているものの,除細動の効果予測に関するパラメータの検討はされていない.
また文献\cite{nature2023}では文献\cite{JSCI2022}の特徴量に加え,ウェーブレット変換によって導出できる時間-周波数領域のエネルギー分布であるスカログラムより,時間軸に着目した新しい特徴量としてNTI (Normalized Time Index)を導入している.
さらに除細動の効果を予測する研究では,除細動直前のVF波形に着目し,抽出する特徴量の時々刻々の変化に着目した新しい特徴量を用いた予測手法が提案されている\cite{SCI}.
文献\cite{SCI}の手法では感度88.24\%,特異度93.10\%,正答率91.30\%という結果が得られている.
また,文献\cite{yosikawa}でウェーブレット変換を用いた時間周波数解析に加え,高速フーリエ変換(Fast Fourier Transform:FFT)を用いた手法が提案されており,
感度96.55\%,特異度93.10\%,正答率94.83\%という成果が示されている.ただし依然として人命に関する精度としては不十分である.

\subsection{研究目的と研究意義}
本研究では,除細動の効果予測手法を提案する.具体的には,まず心室細動における除細動直前の波形に対して,ウェーブレット変換と疑似微分作用素を用いたウェーブレット変換を適用し,
各状態の特徴を抽出するためにそれぞれの変換結果からスカログラムを導出する.また,スカログラムから導出できるNSI(Normalized Spectrum Index),
NTI(Normalized Time Index),SDW(Scale Distribution Width),さらにスペクトルエントロピーとしてシャノンエントロピー,レニーエントロピーを採用する.
これらに加え心電図波形に対してポアンカレプロット解析を行い,基本統計量を用いて特徴量を抽出する.
ただし,得られたすべての特徴量を用いて有用性の検証を行うことは,計算量の面で困難である.
よって,$\chi^2$検定を行い予測に有効である特徴量の選択を行う.最後に検定によって選定された特徴量を組み合わせ,除細動の"成功"と"失敗"の2クラス分類をサポートベクターマシン(Support Vector Machine:SVM)を用いて行う.
この際,3種のカーネル関数(線形,RBF,多項式(2$\sim$5次))と6種の正則化係数を用いて,感度,特異度,そして正答率で評価することにより,
提案手法の本研究の有用性を示す.本研究の手法により,予測精度が向上することで傷病者に対して適切な処置に繋がり,救命率の向上に寄与することができる.
また,除細動を行う前に除細動の効果の有無がわかることにより,救急隊員がより迅速に患者の状態に適した処置を実施することが可能になる.

%疑似微分作用素にはいくつかの設定すべきパラメータが存在するため,パラメータを変更することで変化させた特徴を得ることができる.よってより予測に寄与する組み合わせを見つけることで有用な結果を得られるため,
%パラメータの組み合わせの検討も行う.

\subsection{本論文の構成と用いる表記}
本論文では,第2章で心電図波形,第3章では予測に使用する特徴量と抽出方法について説明する.そして第4章では特徴量の選択方法,選択された特徴量を述べる.
さらに,第5章で提案するサポートベクターマシンを用いた予測システムの検証方法,第6章で予測結果を示す.最後に第7章で今後の課題についてまとめる.

なお,本論文で使用する表記を以下に示す.
\begin{align}
  \bar{x} = \frac{1}{N}\sum_{i=1}^{N}x_i
\end{align}
ベクトル$x_i=(i=1,2\dots,N)$に対し,$\bar{x}$はベクトルの平均を表す.また,複素関数$f(t)$に対して,$\overline{f(t)}$は複素共役を表す.


