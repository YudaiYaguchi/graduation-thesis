%%%
%%%		卒業論文LaTex用テンプレート
%%%
%%%		[文字コード]
%%%		UTF-8
%%%
%%%		[作成したLaTex環境]
%%%		Cloud Latex
%%%		
%%%		[編集履歴]
%%%		2023/12/07	ver1.0 完成, 動作確認
%%%		
\makeatletter
\let\my@xfloat\@xfloat
\makeatother

\RequirePackage{plautopatch}
\RequirePackage[l2tabu, orthodox]{nag}

\documentclass[a4j, 12pt]{jarticle}

%%%
%%%		必要ならpackageは増やしてもいい
%%%
\usepackage[dvipdfmx]{graphicx}
\usepackage{here}
\usepackage{doublespace}
\usepackage{amsmath,amssymb}
\usepackage{amsfonts}
\usepackage{url}
\usepackage{cases}
\usepackage{fancybox}
\usepackage{ascmac}
\usepackage{geometry}
\usepackage{comment}
\usepackage{graphicx}
\usepackage{bxtexlogo}
\usepackage{amsmath}
\usepackage[]{multicol}
\usepackage{caption}
\usepackage{subcaption}
\usepackage{float}
\usepackage{algorithm}
\usepackage{algorithmic}



\makeatletter
\def\@xfloat#1[#2]{
    \my@xfloat#1[#2]%
    \def\baselinestretch{1}%
    \@normalsize \normalsize
}
\makeatother

%%%
%%%		設定関連(いじらないほうがよい)
%%%
\renewcommand{\baselinestretch}{1.5}
\setlength{\textheight}{24cm}
\setlength{\topmargin}{-1cm}
\setlength{\textwidth}{16.2cm}
\setlength{\oddsidemargin}{-0.2cm}
\setlength{\columnsep}{3zw}

%%%
%%%		指導教員印, 卒業年度, 卒論タイトル 
%%%
\title{
	\vspace{-1cm}
	\begin{flushright}
 		\begin{tabular}{|c|} \hline
 			\small \raisebox{10pt}{指導教員印} \\
 			[-15pt] \hline \\
            [15pt] \hline
 		\end{tabular}
 		\\
 	\end{flushright}
	\vspace{2cm}
	% 卒業年度(変えるところ)
 	{\huge 2025年度卒業論文} \\
 	\vspace{3cm}
 	% タイトル(変えるところ)
 	{\huge 加速度推定機能を組み込んだ\\ポテンシャル法による動的障害物回避手法の構築}\\
    \vspace{0.8cm}
    {\large Dynamic Obstacle Avoidance Using a Potential Field Method \\ Incorporating Acceleration Estimation}
}

%%%
%%%		この2つは無視すること
%%%
% \author{}
% \date{}
\begin{document}
	%%%
	%%%		所属, 指導教員名, 自分の学籍番号と名前
	%%%
	\begin{singlespace}
		% タイトルページの挿入
		\maketitle
		\large
 		\vspace{2.0cm} \par
 		% 大学名と学部
 		\hspace{9.0cm}東京都市大学 情報工学部 \\ 					
 		\vspace{-0.8cm} \par
 		% 学科名
 		\hspace{9.0cm}情報科学科 \\
 		\vspace{-0.8cm} \par
 		% 研究室名
 		\hspace{9.0cm}自動制御研究室 \\
 		\vspace{-0.8cm} \par
 		% 指導教員名, 役職
 		\hspace{9.0cm}\hspace{5em}大屋\hspace{0.5em}英稔 教授 \\
 		\vspace{-0.8cm} \par
 		\hspace{9.0cm}\hspace{5em}星\hspace{0.5em}義克 講師 \\
		\vspace{-0.8cm} \par
		% 自分の学籍番号と名前
 		\hspace{9.0cm}2221103\hspace{1.5em}矢口\hspace{0.5em}雄大	
	\end{singlespace}
	\
	
	\thispagestyle{empty}		% タイトルページにページ番号を振らないようにするため
    \newpage
	\begin{abstract}
  本研究は,自動走行ロボットの社会実装に向けて重要となる,動的障害物を考慮した安全かつ効率的な経路生成手法の確立を目的とする。
  近年,物流分野を中心に自動配送ロボットの実用化が進められており,人手不足の緩和や労働環境の改善が期待されている。
  しかし,未知環境や動的障害物が存在する実環境においては,ロボットが障害物の挙動を適切に予測しながら移動する必要があり,
  高い安全性とリアルタイム性を両立した経路生成が求められる。

  移動ロボットの経路生成には,大局的経路計画,ニューラルネットワーク,時空間RRT,ポテンシャル法などが用いられてきた。
  ポテンシャル法は計算量が少なくリアルタイム処理に優れる一方で,障害物と目的地の影響が釣り合うことで移動不能に陥る停留問題を有する。
  これに対し,仮ゴールの設定や壁沿い走行,回り込み走行を組み合わせることで停留問題を解決する手法が提案されてきた。
  また,動的障害物に対しては,障害物の速度推定を用いた回避手法が報告されているが,
  速度情報のみでは加速や減速を伴う運動を十分に表現できず,将来位置の予測誤差により経路が冗長化するという課題が残されている。
  
  そこで本研究では,従来のポテンシャル法を基盤とし,
  障害物の位置変化から速度および加速度を推定する機能を新たに組み込んだ動的障害物回避手法を提案する。提案手法では,
  障害物の将来位置を複数秒先まで予測し,それらを斥力ポテンシャル場に反映させることで,障害物の進行経路への侵入を回避しつつ,安全かつ効率的な経路生成を実現する。さらに,従来手法および通常のポテンシャル法との比較を通じて,到達時間および経路長の観点から提案手法の有効性を検証する。

  \vspace{10pt} 
  
  In recent years, research on autonomous mobile robots has been actively conducted.
  In particular, in the logistics industry, the development and legal reforms toward the practical use of autonomous delivery robots have been progressing, which are expected to help reduce the long working hours of truck drivers and alleviate labor shortages.
  
  For the social implementation of autonomous delivery robots, safe path generation in unknown environments is essential.
  Although methods such as global path planning and neural networks have been proposed, global path planning tends to cause robots to approach obstacles too closely at corners, and neural networks require a large amount of training data and time.
  Therefore, these methods are not suitable for real-time path generation in unknown environments.
  On the other hand, the potential field method is a lightweight algorithm with low computational cost and excellent real-time performance.
  
  This study aims to construct a dynamic obstacle avoidance method based on the potential field approach incorporating acceleration estimation.
  By introducing acceleration estimation, the robot can predict the future motion of obstacles more accurately, enabling smoother and safer path generation in dynamic environments.
  Finally, the effectiveness of the proposed method is verified through simulation.
\end{abstract}
	\thispagestyle{empty}
   
	
	%%%
	%%%		目次の設定
	%%%
	\newpage 					% 表紙の次のページに目次を出力するため
	\pagenumbering{roman}		% ページ番号をローマ数字(i,ii,iii,...)にする	
	\setcounter{page}{1}		% ページ番号をiとする
	\tableofcontents			% 目次の出力

	%%%
	%%%		本文の設定
	%%%
	\newpage 					% 目次の次のページから本文を出力するため
	\pagenumbering{arabic}		% ページ番号をアラビア数字(1,2,3...)にする
	\setcounter{page}{1} 		% ページ番号を1とする
	%%%
	%%%		本文の挿入
	%%%	
	%%%		[やり方]
	%%%		\input{xyz.tex}でxyz.texファイルを挿入できる
	%%%		ただし, 挿入するファイルの文頭に \newpage \section{}を入れること
    
	\newpage
\section{はじめに}
\subsection{研究背景}

近年,自動走行ロボットの研究が盛んに行われている.
特に物流業界では,自動配送ロボットの実用化に向けた開発や法律の改正が進められている\cite{meti_robot}.これにより,
トラック運転手の長時間労働の是正や物流分野の人手不足の緩和が期待される.
このようなロボットの社会実装を実現するためには,未知環境下や未知障害物の存在下において,安全かつ安定した経路を生成する必要がある.
特に,動的障害物を考慮した経路生成を行う場合は障害物の検出を行い,その移動を適切に予測することが重要となる.

移動ロボットにおける障害物回避のための経路生成手法として,
大局的経路計画やニューラルネットワーク,時空間RRT,ポテンシャル法を用いた方法がある.
大局的経路計画は,ロボットが地図全体などの環境情報をすでに把握している場合において,
スタートからゴールまでの最適な経路をあらかじめ計算して求める手法である.
ただし,この手法では静的環境下では非常に有効だが,動的障害物が存在する場合には
障害物の移動の度に経路を再計算する必要があり,計算コストが高くなるという欠点がある\cite{idou}.
次にニューラルネットワークを用いた方法では,多層パーセプトロン(MLP)などのモデルを使用し,静的および動的障害物のある環境下で,
ロボットの経路と速度を効率的に生成する手法が提案されている\cite{neural-network}.
このアプローチは,一度ネットワークの学習が完了すれば動的な環境においても少ない計算時間で経路を生成できるという利点を持つ.
しかし,ニューラルネットワークの一般的な欠点として,学習に多くのデータと時間を要するという課題がある.
この他にも,RRT(Rapidly-exploring Random Tree)を二次元平面から時間軸を含む空間へ拡張した時空間RRTが提案されている\cite{StRRT}.
これは位置だけでなく時間情報を考慮して探索を行うため,他ロボットとの衝突回避や譲り合いといった動的状況に対応した軌道生成が可能である.
しかし,サンプリングベース手法の一般的な課題として探索空間が高次元になるにつれて計算量が増大し,リアルタイム処理が難しい場面も存在する.
一方,ポテンシャル法\cite{potential}では,目的地からの引力および障害物からの斥力によって構成されるポテンシャル場を生成し,
ロボットがそのポテンシャル場の勾配にしたがって移動することで,目的地への経路を導出する手法である.
図\ref{fig:potential}にポテンシャル場の例を示す.
ポテンシャル法は計算量が少ないシンプルなアルゴリズムのためリアルタイム性が優れている手法である.
しかし,ポテンシャル法には「停留問題」がある.
「停留問題」とは障害物の斥力と目的地の引力が釣り合い,勾配が限りなく0に近い状態に陥り目的地にたどり着けなくなる問題である\cite{potential}.
\begin{figure}[t!]  \centering
  \includegraphics[width=.8\linewidth]{figure/potential.png}
  \setlength{\abovecaptionskip}{5pt} % キャプション上
  \setlength{\belowcaptionskip}{5pt} % キャプション下
  \caption{ポテンシャル場の例\cite{potential-field}}
  \label{fig:potential}
\end{figure}

従来研究\cite{kasuyama}~\cite{enokida}では,ポテンシャル場の勾配ベクトルに沿ってロボットを移動させることが考えられている.
従来研究\cite{kasuyama}では,センサが2点同時に障害物を検出した場合にその2点を結ぶ直線の傾きを算出し,その延長線を障害物の進行方向の予測線分として用いることで,
その予測線分に沿った壁沿い走行を行わせる手法を提案している.
また,予測線分と目的地方向との角度関係を用いて方向決定をしている.
一方,従来研究\cite{hiraoka}では,壁沿い走行後に斥力ポテンシャルの影響で大きく膨らんだ経路が生成される問題に着目し,
障害物の裏側へ回り込む円弧状の走行を組み合わせることでより効率的な経路を得る手法を提案している.
さらに,回避後も障害物を過度に認識し続けることで発生する冗長な経路を抑制するために,
障害物の重要度を示す重みパラメータを状況に応じて調整する手法を提案し,適切な重み設定により安全性を損なわずに経路効率を改善している.
従来研究\cite{koike}では,停留した際に障害物の向こう側に仮ゴールをA*探索法を用いて設定することで,「停留問題」を解決することに成功しており,静的障害物の回避に有効であることが示されている.
加えて,文献\cite{koike}では壁沿い走行に回り込み走行を組み込むことで,より柔軟に停留を回避する手法も提案されている.
一方,従来研究\cite{enokida}では,ポテンシャル法に速度推定機能を組み込むことで,動的障害物の回避に成功している.
この手法では,動的障害物の回避においてその速度情報を基に仮ゴールの設定を行っている.
そのため,障害物の速度や進行方向を考慮した経路生成が可能となり,従来研究\cite{koike}の静的障害物に対する回避手法を組み合わせることによって,
静的または動的環境下における自律移動を実現している.
しかし,障害物が加速や減速を伴って移動する場合には障害物の速度のみを用いた仮ゴール設定では将来位置を正確に推定できない場合がある.
速度情報はその時点におけるサンプリングした運動状態を表すに過ぎず,加速度による変化を考慮していない.
その結果,障害物が減速して向きを変えたり,障害物が加速して接近する状況では経路が冗長になる.
この誤差により,ロボットは障害物の進行方向を誤って解釈し,生成される経路が冗長になり目標地点まで効率的に移動できなくなる場合がある.

\subsection{研究目的}
本研究では,従来研究\cite{hiraoka}を基に,ポテンシャル法に障害物の加速度推定機能を組み込んだ動的障害物回避手法を提案する.
提案手法では,障害物の位置変化から速度および加速度を算出し,障害物の将来位置を予測する.
これにより,障害物の将来位置を考慮したより精度の高い回避行動を実現することを目的とする.
また障害物の将来位置を複数秒先まで予測し,それらを斥力ポテンシャル場に反映させることによって,
ロボットが障害物の進行経路に侵入することを回避し,より安全かつ効率的な経路の生成を実現する.
最後に従来研究\cite{enokida}と通常のポテンシャル法と提案する手法を到達時間および経路長の観点から比較し,その有効性を検証する.
 % 序論
    \newpage
\section{従来研究}
\subsection{ポテンシャル場の勾配}
従来研究\cite{enokida}では,障害物から発生する斥力ポテンシャル場,目的地から発生する引力ポテンシャル場,および全体のポテンシャル場を次式のように定めている.

\begin{flushleft} 
\begin{itemize}
  \item 障害物座標の斥力ポテンシャル関数 $P_{ob}(x_r, y_r)$
  \begin{equation}
    P_{ob}(x_r, y_r) \triangleq \frac{1}{\sqrt{(x_r - x_{ob})^2 + (y_r - y_{ob})^2}}
  \end{equation}

  \item 目的地座標の引力ポテンシャル関数 $P_{ds}(x_r, y_r)$
  \begin{equation}
    P_{ds}(x_r, y_r) \triangleq - \frac{1}{\sqrt{(x_r - x_{ds})^2 + (y_r - y_{ds})^2}}
  \end{equation}

  \item 全体のポテンシャル場 $P(x_r, y_r)$
  \begin{equation}
    P(x_r, y_r) \triangleq \sum \omega_{ob} P_{ob} + \omega_{ds} P_{ds}
  \end{equation}
\end{itemize}
\end{flushleft}

\begin{tabular}{ll}
  $(x_r, y_r)$           & ロボットの座標 \\
  $(x_{ob}, y_{ob})$     & 障害物の座標\\
  $(x_{ds}, y_{ds})$     & 目的地の座標\\
  $P_d(x_r,y_r)$         & 引力ポテンシャル関数の重み \\
  $P_o(x_r,y_r)$         & 斥力ポテンシャル関数の重み \\
\end{tabular}

\vspace{1em} % 段落間のスペースを調整

ここで,$(x_r, y_r)$ はロボットの座標を示し,$(x_{ob}, y_{ob})$,$(x_{ds}, y_{ds})$ はそれぞれ障害物の座標と目的地の座標を表している.
また,(3) 式の $\omega_{ob}$,$\omega_{ds}$ はどちらも正の実数の重み付けパラメータであり,
それぞれ引力ポテンシャル関数 $P_d(x_r, y_r)$ と斥力ポテンシャル関数 $P_o(x_r, y_r)$ の重みとなっている.
ただし,$\omega_{ob}$ を大きくしてしまうと障害物の斥力から受ける影響が強まり,
目的地への到着が難しくなる可能性があるため $\omega_{ob} < \omega_{ds}$ とする.
ロボットの動く方向は,ポテンシャル場の勾配によって決められ,
$x$ 方向,$y$ 方向における勾配は次式のように求められる.

% x方向の勾配
\noindent\textbf{x方向の勾配}\vspace{-20pt}

\begin{align}
\frac{\partial P_{ob}(x_r, y_r)}{\partial x} 
&= - \frac{x_r - x_{ob}}{\{(x_r - x_{ob})^2 + (y_r - y_{ob})^2\}\sqrt{(x_r - x_{ob})^2 + (y_r - y_{ob})^2}} \\[24pt] 
\frac{\partial P_{ds}(x_r, y_r)}{\partial x} 
&= \frac{x_r - x_{ds}}{\{(x_r - x_{ds})^2 + (y_r - y_{ds})^2\}\sqrt{(x_r - x_{ds})^2 + (y_r - y_{ds})^2}} \\[24pt]
\frac{\partial P(x_r, y_r)}{\partial x}
&= \sum \omega_o \frac{\partial P_{ob}(x_r, y_r)}{\partial x} + \omega_d \frac{\partial P_{ds}(x_r, y_r)}{\partial x} 
\end{align}

% y方向の勾配
\noindent\textbf{y方向の勾配}
\begin{align}
\frac{\partial P_{ob}(x_r, y_r)}{\partial y} 
&= - \frac{y_r - y_{ob}}{\{(x_r - x_{ob})^2 + (y_r - y_{ob})^2\}\sqrt{(x_r - x_{ob})^2 + (y_r - y_{ob})^2}}  \\[24pt]
\frac{\partial P_{ds}(x_r, y_r)}{\partial y} 
&= \frac{y_r - y_{ds}}{\{(x_r - x_{ds})^2 + (y_r - y_{ds})^2\}\sqrt{(x_r - x_{ds})^2 + (y_r - y_{ds})^2}}  \\[24pt]
\frac{\partial P(x_r, y_r)}{\partial y}
&= \sum \omega_o \frac{\partial P_{ob}(x_r, y_r)}{\partial y} + \omega_d \frac{\partial P_{ds}(x_r, y_r)}{\partial y}
\end{align}

よって,ポテンシャル場内のロボット位置$(x_r, y_r)$における勾配$\nabla p(x_r, y_r)$は,次式のように表される.

\begin{equation}
  \nabla p(x_r, y_r) \triangleq \left( \frac{\partial P(x_r,y_r)}{\partial x}, \frac{\partial P(x_r,y)}{\partial y} \right)
  \label{eq:gradient}
\end{equation}

地点$(x_r, y_r)$にいるロボットは,式(\ref{eq:gradient})で表される勾配ベクトルに従って移動することで,目的地へと向かう. % 従来研究
    \newpage
\section{提案する手法}
従来研究\cite{kasuyama}~\cite{koike}では,静的障害物のみを対象としており,動的に移動する障害物については十分に考慮されていない.
また,従来研究\cite{enokida}では動的障害物の回避を扱っているものの,対象としているのは等速運動を行う障害物であり,
加速度を伴う障害物に対しては適切な経路生成が困難であるという課題が残っている.
そこで本研究では,センサにより検知した動的障害物の情報からその速度および加速度を推定し,
複数秒先の将来位置を予測する手法を提案する.
具体的には,予測した将来位置をポテンシャル場に反映させることで,加速度を持つ動的障害物に対しても,
ポテンシャル法に基づくスムーズで効率的な回避挙動を実現することを目指す.
加えて,本研究の手法を従来研究\cite{hiraoka}の静的障害物回避アルゴリズムへ組み込むことにより,
動的障害物と静的障害物の双方に対応可能な回避手法を構築する.
また,前述したように300[cm]のセンサを5つ持つロボットを使用し,動的障害物をセンサが検知した場合には,その四隅の位置情報を取得できるものとする.


\subsection{センサ距離の設定}
従来研究\cite{enokida}では,動的障害物に対する検知距離を200[cm]に設定している.
しかし,想定している障害物は等速運動を前提としており,
速度変化を伴う動的障害物には十分に対応できないという問題がある.
特に,加速度を有する障害物は短時間で接近速度が大きく変化するため,200[cm]程度の検知距離では回避動作の開始が遅れる可能性が高い.
本研究ではこの課題を解消するため,センサの検知距離を300[cm]に設定した.
検知範囲を拡大することで障害物をより早期に捉え,
その速度および加速度を推定するために必要な複数回の観測データを安定して取得できる.
これにより,障害物が加速しながら接近する場合でも将来位置の予測精度を維持でき,
衝突リスクを低減しつつ適切な回避経路を生成するための十分な時間的余裕を確保できる.

\subsection{動的障害物の将来位置予測}
本研究では,動的障害物 $ob_m$ の将来位置を推定するために,
センサで取得した障害物の四隅の位置情報を用いて重心位置を
その時刻における代表位置とする.
続いて,連続する観測データから速度および加速度を推定し,
等加速度運動の運動方程式に基づいて複数秒先の位置を予測する.
以下に,動的障害物の将来位置予測の手順を示す.

\begin{enumerate}
  \item センサによって動的障害物 $ob_m$ の四隅の座標を取得し,
        その重心位置 $\mathbf{p}_{t}$ を次式のように求める.
        \begin{equation}
            \mathbf{p}_{t} = (x_{1}, y_{1}), \quad
            \mathbf{p}_{t-\Delta t} = (x_{2}, y_{2})
        \end{equation}

  \item 重心位置の変化量から速度ベクトル $\mathbf{v}$ を推定する.
      \begin{equation}
            \mathbf{v} = (x_{v}, y_{v})
            = \frac{\mathbf{p}_{t} - \mathbf{p}_{t-\Delta t}}{\Delta t}
      \end{equation}

  \item さらに速度の変化量に基づき,加速度ベクトル $\mathbf{a}$ を推定する.
      \begin{equation}
            \mathbf{a} = (x_{a}, y_{a})
            = \frac{\mathbf{v}_{t} - \mathbf{v}_{t-\Delta t}}{\Delta t}
      \end{equation}

  \item 推定した速度・加速度と現在位置 $\mathbf{p}_t$ を用いて,
        等加速度運動の運動方程式
        \begin{equation}
          \label{eq:acc_model}
            \mathbf{p}(T)
            = \mathbf{p}_t + \mathbf{v}T + \frac{1}{2}\mathbf{a}T^2
        \end{equation}
        により,予測ステップ数 $T$ を変化させながら複数秒先の将来位置を逐次算出する.

  \item 推定した将来位置群をポテンシャル場に反映し,  
        ロボットはこれらを基に動的障害物を考慮した回避軌道を生成する.
\end{enumerate}


\subsection{提案手法アルゴリズム}
提案手法の全体フローを以下に示す.

\begin{algorithm}
\caption{目的地までの経路生成アルゴリズム}
\begin{algorithmic}

\WHILE{目的地に到達していない}
  \STATE センサで周囲の障害物を検出する
  \FOR{各検出対象 $ob_m$}
    \STATE 四隅の座標から重心位置 $\mathbf{p}_t$ を算出する
    \IF{$ob_m$ が動的障害物である場合}
      \STATE 連続する観測から速度 $\mathbf{v}$ と加速度 $\mathbf{a}$ を推定する
      \FOR{$\tau = 1 \dots T$}
        \STATE 等加速度運動の運動方程式(式\ref{eq:acc_model})により将来位置を算出する
      \ENDFOR
      \STATE 予測位置群を斥力ポテンシャルとしてポテンシャル場へ反映する
    \ELSIF{$ob_m$ が静的障害物である場合}
      \STATE 従来手法\cite{hiraoka}の静的障害物回避アルゴリズムを適用する
    \ENDIF
  \ENDFOR

  \STATE ポテンシャル場(目的地からの引力+障害物の斥力(予測位置含む))を構築する
  \STATE ポテンシャル場の勾配に基づき,ロボットを移動させる
\ENDWHILE
\end{algorithmic}
\end{algorithm}

本手法では,ロボットが目的地へ到達するまでの各ステップで,センサ情報に基づいてポテンシャル法により移動方向を逐次決定する.
まず,ロボットは搭載センサを用いて周囲の障害物を検出する.
検出された各障害物に対して,その四隅の座標から重心位置を算出し,対象が動的障害物か静的障害物かを判別する.
動的障害物に対しては,連続した観測から速度および加速度を推定し,
等加速度運動の運動方程式を用いて将来位置を時刻 $\tau=1 \dots T$ まで予測する.
本研究では予測ステップ数 $T = 20$ と設定し,20ステップ先までの将来位置を算出する. 
これらの将来位置群を斥力ポテンシャルとしてポテンシャル場に反映させる.
一方,静的障害物に対しては,従来手法\cite{hiraoka}に基づく静的障害物回避アルゴリズムを適用する.
すべての障害物処理が完了すると,目的地に対する引力ポテンシャルと,静的または動的障害物から生じる斥力ポテンシャルを統合し,全体のポテンシャル場を構築する.
ロボットは,このポテンシャル場の勾配方向に沿って逐次移動し,最終的に目的地へ到達する.
    % \newpage
\section{シミュレーション結果と考察}
3章で示したSTEPに従い,図\ref{fig:map}に示すマップの駐車開始位置から駐車目標位置までの経路生成を行い,シミュレーションした結果を示す.
\subsection{シミュレーションの条件}
本論文のシミュレーションではトヨタ自動車  ヤリス\cite{model_car}をモデルに用いた,その車両のパラメータの詳細を表\ref{table:parameter}に示す.
シミュレーションにはショッピングモールなどでよく使われる駐車場のサイズ$3.0[\mathrm{\,m}] \times 5.0[\mathrm{\,m}]$,道路幅を$4.4[\mathrm{\, m}]$に設定したマップを使用した.また,駐車開始位置座標は$(1300,720)$,$(1300,745)$,$(1300,660)$の3パターンで検証した.

\begin{figure}[tp]
  \centering
  \includegraphics[width=0.8\linewidth]{figure/yaris.png}
  \label{fig:yaris} 
  \caption{トヨタ自動車 ヤリス\cite{model_car}}
\end{figure}

\begin{table}[btp]
 \caption{車両のパラメータ}
  \label{table:parameter}
  \centering
  \begin{tabular}{cc}
    \hline
    パラメータ  & 値 \\
    \hline
    長さ  & 394[cm] \\
    幅  & 169.5[cm]  \\
    ホイールベース  & 255[cm]   \\
    最小回転半径  &  322[cm]  \\
    \hline
  \end{tabular}
\end{table}


\subsection{シミュレーション結果}
図\ref{fig:main1}(\subref{fig:sc1}),\ref{fig:main2}(\subref{fig:scup}),\ref{fig:main3}(\subref{fig:scdown})に駐車開始位置がそれぞれ$(1300,720)$,$(1300,745)$,$(1300,660)$とした場合のシミュレーション結果を示す.図\ref{fig:main1}(\subref{fig:sim}),\ref{fig:main2}(\subref{fig:simup}),\ref{fig:main2}(\subref{fig:simdown})にはそれぞれ同じ環境における従来研究\cite{ogata}によるシミュレーション結果を示す.図中の青色の線が車両後輪軸中心の軌道,その他の色の線は車両の四隅の軌道を表している.

図\ref{fig:main1}(\subref{fig:sc1})と\ref{fig:main1}(\subref{fig:sim})を比較すると,図\ref{fig:main1}(\subref{fig:sim})は駐車開始位置と生成した経路にズレが発生しているが,図\ref{fig:main1}(\subref{fig:sc1})ではクロソイド曲線により姿勢角を必要以上に変えずに駐車開始位置まで滑らかな経路を生成出来ている.
また,図\ref{fig:main2}(\subref{fig:scup})と\ref{fig:main2}(\subref{fig:simup})を比較すると,図\ref{fig:main2}(\subref{fig:simup})では駐車開始位置から大きくずれてしまっているが,図\ref{fig:main2}(\subref{fig:scup})は上手くクロソイド曲線で駐車開始位置まで滑らかな経路を生成出来ている.
\begin{figure}[!hb] 
\begin{minipage}[b]{.5\linewidth}  
    \centering
    \includegraphics[width=1.2\linewidth]{figure/sim_clo_mid.png}
    \subcaption{提案手法}
    \label{fig:sc1} 
  \end{minipage}
  \begin{minipage}[b]{.5\linewidth} 
    \centering
    \includegraphics[width=1.2\linewidth]{figure/C_300-440_720.png}
    \subcaption{従来研究\cite{ogata}}
    \label{fig:sim} 
\end{minipage}
  \caption{駐車開始位置$(1300,720)$のシミュレーション}
  \label{fig:main1}
\end{figure}
\newpage
\begin{figure}[!ht]
 \begin{minipage}[b]{.5\linewidth}

    \centering
    \includegraphics[width=1.2\linewidth]{figure/sim_clo_up.png}
    \subcaption{提案手法}
    \label{fig:scup} 
  \end{minipage}
  \begin{minipage}[b]{.5\linewidth}
  
    \centering
    \includegraphics[width=1.2\linewidth]{figure/C_300-440_745.png}
    \subcaption{従来研究\cite{ogata}}
    \label{fig:simup} 
\end{minipage}
    \caption{駐車開始位置$(1300,745)$のシミュレーション}
    \label{fig:main2}
\end{figure}

{更に,図\ref{fig:main3}(\subref{fig:scdown})と\ref{fig:main3}(\subref{fig:simdown})を比較すると,駐車スペースから完全に出た後にシグモイド関数を用いて駐車開始位置まで滑らかな経路を生成していることがわかる.
これらの結果から,提案手法の方が従来手法\cite{ogata}よりも駐車開始位置まで滑らかな経路を生成出来ていることが確認できた.}


\begin{figure}[!b]
 \begin{minipage}[b]{.5\linewidth}
  
    \centering
    \includegraphics[width=1.2\linewidth]{figure/sim_sig_down.png}
    \subcaption{提案手法}
    \label{fig:scdown} 
  \end{minipage}
  \begin{minipage}[b]{.5\linewidth}
  
    \centering
    \includegraphics[width=1.2\linewidth]{figure/C_300-440_660.png}
    \subcaption{従来研究\cite{ogata}}
    \label{fig:simdown} 
\end{minipage}
  \caption{駐車開始位置$(1300,660)$のシミュレーション}
  \label{fig:main3}
\end{figure}



\clearpage

    % \newpage
\section{結論}

本研究では,加速度推定機能を組み込んだポテンシャル法による動的障害物回避手法を提案した.
従来研究\cite{enokida}では,動的障害物の速度情報のみを用いた回避手法を提案しているが,
加速度を持つ障害物に対しては適切な回避が困難であるという課題があった.
そこで本研究では,センサにより取得した障害物の位置情報から速度および加速度を推定し,
等加速度運動の運動方程式を用いて複数秒先の将来位置を予測する手法を提案した.
さらに,予測した将来位置群を斥力ポテンシャル場に反映させることで,
ロボットが障害物の進行経路に侵入することを防ぎ,より安全かつ効率的な経路生成を提案した.
シミュレーション結果より,
提案手法は従来研究\cite{enokida}およびセンサベースのポテンシャル法より短い到達時間と経路長を実現した.
特に,運動方向と反対に加速度をもつ動的障害物に対しても有効に機能することが確認され,
加速度を持つ障害物だけでなく,等速で移動する障害物に対しても効率的な回避を実現できることが示された.
また,動的障害物と静的障害物が同時に存在する複雑な環境下においても,
提案手法はそれぞれに対して最適な回避行動を選択することで,
目的地への到達に成功した.

以上の結果から,加速度推定機能を組み込むことで,動的障害物の運動状態をより正確に予測でき,
ロボットが余分な回避行動を取ることなく効率的な経路を選択できることが確認された.
これにより,加速度を持つ動的障害物に対するポテンシャル法の適用範囲が拡大され,
より実用的な自動配送ロボットの実現に貢献できると考えられる.
今後の課題として,より複雑な運動パターンを持つ障害物への対応や,
複数の動的障害物が同時に存在する環境における性能評価,
実機実験による検証などが挙げられる.

    \newpage
\addcontentsline{toc}{section}{参考文献}
\begin{thebibliography}{99}

  \bibitem{meti_robot} 経済産業省,自動配送ロボットの将来像を取りまとめました,
  \url{https://www.meti.go.jp/press/2024/02/20250226002/20250226002.html}
  %
  \bibitem{idou}青柳 誠司, 佐藤 伸仁, 山本 恭輝, 高橋 智一, 鈴木 昌人, "移動ロボットの移動
  障害物回避に関するファジィルールの学習 ポテンシャル法, 強化学習法との比較,"
  システム制御情報学会論文誌,Vol.34,No.8,pp.209–218,2021.
  
  \bibitem{neural-network}Ngangbam Herojit Singh and Khelchandra Thongam,
  "Neural network-based approaches for mobile robot navigation in static and moving obstacles environments,"
  Intelligent Service Robotics 12.1 (2019): 55–67. Web. \url{https://doi.org/10.1007/s11370-018-0260-2}.
  
  \bibitem{potential}Xing. Yang, Wei. Yang, Huijuan. Zhang, Hao. Chang, Chin-Yin. Chen, and Shuangchi. Zhang, "A new method for robot path planning based artificial potential field," 
  2016 IEEE 11th Conference on Industrial Electronics and Applications (ICIEA),
  pp. 1294-1299, 2016.
  
  \bibitem{kasuyama}粕山 剛輝,”壁沿い走行を組み込んだ仮想ゴールポイントに基づくポテンシャル法による停留と障害物回避,”東京都市大学2021年度卒業論文(2021)

  \bibitem{hiraoka} 平岡 翔,”ポテンシャル法における停留問題の回避,および効率的な経路生成手法,”東京都市大学2022年度卒業論文(2022)
  %

  \bibitem{koike} 小池 基也,”A*探索法を組み込んだ静止障害物に対するポテンシャル法の回避方法,”東京都市大学2023年度卒業論文(2023)
  %------%
  %
  \bibitem{enokida} 榎田 日和,”速度推定機能を組み込んだポテンシャル法による動的障害物の回避,”東京都市大学2024年度卒業論文(2024)

  \bibitem{potential-field}瑠城 祐亮,江口 和樹,岩崎 聡,山内 由章,中田 昌宏,
  ”ポテンシャル法によるロボット製品の障害物回避技術の開発,”
  新製品・新技術特集 三菱重工技法,Vol.51, No.1, pp.40–45, 2014.

\end{thebibliography}

    \clearpage
	\section*{謝辞}
	本研究を進めるにあたり,ご多忙にも関わらず丁寧にご指導,ご鞭撻のほどをいただきました大屋 英稔 教授,星 義克 講師に深く感謝申し上げます.
	また,悩んでいるところを見逃さず声をかけていただいた大学院生の先輩方,日頃から同じ研究室の仲間として相談に応じてくださった自動制御研究室の皆様にもお礼を申し上げます.
	この1年間での研究生活を通じて,様々な経験を積むことができ大きく成長することができました.今後の人生でもこの経験を活かしていきたいと思います.ありがとうございました.
\end{document}
