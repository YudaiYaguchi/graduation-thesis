\newpage
\section{従来研究}
\subsection{ポテンシャル場の勾配}
従来研究\cite{enokida}では,障害物から発生する斥力ポテンシャル場,目的地から発生する引力ポテンシャル場,および全体のポテンシャル場を次式のように定めている.

\begin{flushleft} 
\begin{itemize}
  \item 障害物座標の斥力ポテンシャル関数 $P_{ob}(x_r, y_r)$
  \begin{equation}
    P_{ob}(x_r, y_r) \triangleq \frac{1}{\sqrt{(x_r - x_{ob})^2 + (y_r - y_{ob})^2}}
  \end{equation}

  \item 目的地座標の引力ポテンシャル関数 $P_{ds}(x_r, y_r)$
  \begin{equation}
    P_{ds}(x_r, y_r) \triangleq - \frac{1}{\sqrt{(x_r - x_{ds})^2 + (y_r - y_{ds})^2}}
  \end{equation}

  \item 全体のポテンシャル場 $P(x_r, y_r)$
  \begin{equation}
    P(x_r, y_r) \triangleq \sum \omega_{ob} P_{ob} + \omega_{ds} P_{ds}
  \end{equation}
\end{itemize}
\end{flushleft}

\begin{tabular}{ll}
  $(x_r, y_r)$           & ロボットの座標 \\
  $(x_{ob}, y_{ob})$     & 障害物の座標\\
  $(x_{ds}, y_{ds})$     & 目的地の座標\\
  $P_d(x_r,y_r)$         & 引力ポテンシャル関数の重み \\
  $P_o(x_r,y_r)$         & 斥力ポテンシャル関数の重み \\
\end{tabular}

\vspace{1em} % 段落間のスペースを調整

ここで,$(x_r, y_r)$ はロボットの座標を示し,$(x_{ob}, y_{ob})$,$(x_{ds}, y_{ds})$ はそれぞれ障害物の座標と目的地の座標を表している.
また,(3) 式の $\omega_{ob}$,$\omega_{ds}$ はどちらも正の実数の重み付けパラメータであり,
それぞれ引力ポテンシャル関数 $P_d(x_r, y_r)$ と斥力ポテンシャル関数 $P_o(x_r, y_r)$ の重みとなっている.
ただし,$\omega_{ob}$ を大きくしてしまうと障害物の斥力から受ける影響が強まり,
目的地への到着が難しくなる可能性があるため $\omega_{ob} < \omega_{ds}$ とする.
ロボットの動く方向は,ポテンシャル場の勾配によって決められ,
$x$ 方向,$y$ 方向における勾配は次式のように求められる.

% x方向の勾配
\noindent\textbf{x方向の勾配}\vspace{-20pt}

\begin{align}
\frac{\partial P_{ob}(x_r, y_r)}{\partial x} 
&= - \frac{x_r - x_{ob}}{\{(x_r - x_{ob})^2 + (y_r - y_{ob})^2\}\sqrt{(x_r - x_{ob})^2 + (y_r - y_{ob})^2}} \\[24pt] 
\frac{\partial P_{ds}(x_r, y_r)}{\partial x} 
&= \frac{x_r - x_{ds}}{\{(x_r - x_{ds})^2 + (y_r - y_{ds})^2\}\sqrt{(x_r - x_{ds})^2 + (y_r - y_{ds})^2}} \\[24pt]
\frac{\partial P(x_r, y_r)}{\partial x}
&= \sum \omega_o \frac{\partial P_{ob}(x_r, y_r)}{\partial x} + \omega_d \frac{\partial P_{ds}(x_r, y_r)}{\partial x} 
\end{align}

% y方向の勾配
\noindent\textbf{y方向の勾配}
\begin{align}
\frac{\partial P_{ob}(x_r, y_r)}{\partial y} 
&= - \frac{y_r - y_{ob}}{\{(x_r - x_{ob})^2 + (y_r - y_{ob})^2\}\sqrt{(x_r - x_{ob})^2 + (y_r - y_{ob})^2}}  \\[24pt]
\frac{\partial P_{ds}(x_r, y_r)}{\partial y} 
&= \frac{y_r - y_{ds}}{\{(x_r - x_{ds})^2 + (y_r - y_{ds})^2\}\sqrt{(x_r - x_{ds})^2 + (y_r - y_{ds})^2}}  \\[24pt]
\frac{\partial P(x_r, y_r)}{\partial y}
&= \sum \omega_o \frac{\partial P_{ob}(x_r, y_r)}{\partial y} + \omega_d \frac{\partial P_{ds}(x_r, y_r)}{\partial y}
\end{align}

よって,ポテンシャル場内のロボット位置$(x_r, y_r)$における勾配$\nabla p(x_r, y_r)$は,次式のように表される.

\begin{equation}
  \nabla p(x_r, y_r) \triangleq \left( \frac{\partial P(x_r,y_r)}{\partial x}, \frac{\partial P(x_r,y)}{\partial y} \right)
  \label{eq:gradient}
\end{equation}

地点$(x_r, y_r)$にいるロボットは,式(\ref{eq:gradient})で表される勾配ベクトルに従って移動することで,目的地へと向かう.