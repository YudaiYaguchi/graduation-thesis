\newpage
\section{序論}
\subsection{研究背景}

近年,自動走行ロボットの研究が盛んに行われている.
特に物流業界では自動配送ロボットの実用化に向けた開発や法律の改正等が行われている\cite{meti_robot}.これにより,
トラック運転手の長時間労働是正や物流分野の人手不足の緩和が期待される.

自動配送ロボットの社会実装には未知環境での安全な経路生成が不可欠であり,
その手法の一つとして大局的経路計画やニューラルネットワークを用いた方法がある.
大局的経路計画では曲がり角付近で障害物に接近しすぎる問題がある\cite{idou}.
また,ニューラルネットワークは学習に多くのデータと時間を要する.
一方,ポテンシャル法は計算量が少ないシンプルなアルゴリズムのためリアルタイム性が優れている手法である.
ポテンシャル法とは目的地から引力,障害物から斥力をそれぞれ受けるポテンシャル場を設定し,
ロボットがポテンシャル場の勾配に従って移動することで目的地へ到達させる手法である.図\ref{fig:potential}にポテンシャル場の例を示す.
しかし,ポテンシャル法には停留という問題がある.
障害物の斥力と目的地の引力が釣り合い,勾配が限りなく0に近い状態に陥り目的地にたどり着けなくなる問題である.

従来研究\cite{hriaoka,koike}では,ポテンシャル場の勾配ベクトルに沿ってロボットを移動させることが考えられている.
特に従来研究\cite{koike}では,停留した際に障害物の向こう側に仮ゴールをA*探索法を用いて設定することで,「停留問題」を解決することに成功しており,静的障害物の回避に有効であることが示されている.
さらに,壁沿い走行に加えて回り込み走行を組み込むことで,より柔軟に停留を回避する手法も提案されている.
一方,従来研究\cite{enokida}では,ポテンシャル法に速度推定機能を組み込むことで,動的障害物の回避に成功している.
この手法により,障害物の速度や進行方向を考慮した経路生成が可能となり,従来の静的障害物に対する回避手法を発展させる形で,動的環境下における自律移動の実現が図られている.
しかしながら,この手法では障害物の速度のみを考慮しているため,仮ゴールの設定位置が不適切になることがあり,
結果として生成される経路が遠回りになり,最短経路で目標地点へ到達することが困難になる場合がある.
さらに,障害物の加速度を考慮していないため,急加速や急減速を行う障害物に対しては,適切に回避動作を行うことが難しい.

\vspace{10pt} % 図の前にスペースを入れる\begin{figure}[htbp]
\begin{figure}[htbp]
  \centering
  \includegraphics[width=.8\linewidth]{figure/potential.png}
  \setlength{\abovecaptionskip}{5pt} % キャプション上
  \setlength{\belowcaptionskip}{5pt} % キャプション下
  \caption{ポテンシャル場の例\cite{enokida}}
  \label{fig:potential}
\end{figure}

\subsection{研究目的}
本研究では,従来研究\cite{enokida}を基に,ポテンシャル法に障害物の加速度予測機能を組み込んだ動的障害物回避手法を提案する.
提案手法では,障害物の位置変化から速度および加速度を算出し,障害物の将来位置を予測する.
これにより,障害物の運動傾向を考慮したより精度の高い回避行動を実現することを目的とする.
また,障害物の将来位置を複数ステップ先まで予測し,それらを斥力ポテンシャル場に反映させることで,
ロボットが障害物の進行経路に侵入することを防ぎ,より安全かつ安定した経路生成を実現する.
最後に,シミュレーションを通じて提案手法の有効性を検証する.
