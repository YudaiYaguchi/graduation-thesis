\newpage
\section{はじめに}
\subsection{研究背景}
近年,自動車の増加に伴って駐車スペースが少なくなっており,狭い土地に駐車場を作るケースが増えている.そのため,駐車の難易度が上がってしまい,運転や駐車が苦手なドライバーによって毎年多くの事故が発生している.自動運転技術の開発は快適性のみではなく,こうした交通事故を防ぐための1つの技術として,盛んに行われている.自動運転技術や運転時支援機能は社会の実用段階にまで達しており,自動車会社各社の市販車にもカメラやセンサーを搭載し障害物を常に検出することで,例えばトヨタ自動車株式会社ならば,パーキングサポートブレーキ(衝突被害軽減ブレーキ)やアドバンストパーク(自動駐車システム)などの自動運転レベル2の運転時支援機能が安全装備として市販車に多く搭載されている\cite{toyota}.
自動駐車システムのアドバンストパーク\cite{toyota}では,駐車する目標位置の真横に車を移動させて停止した状態から自動駐車システムを稼働させる.自動駐車システムが稼働すると,車両に搭載されているカメラから駐車場の白線を認識し,駐車可能な位置の候補が画面に表示される(図\ref{fig:AP}).表示された図の画面から車両の止める位置をドライバーが選択すると自動で車両が動き,図\ref{fig:AP2}の赤い矢印のように前進してから駐車目標位置まで後退する.また,自動運転中に障害物に接触しないようにカメラやセンサで周囲を監視し,障害物を検出したら自動でブレーキ制御を行う機能もある.
\begin{figure}[htbp]
  
  \centering
  \includegraphics[width=.6\linewidth]{figure/car_parking2.png}
  \caption{アドバンストパーク\cite{toyota}の画面}
  \label{fig:AP}
\end{figure}

\begin{figure}[htbp]
  
  \centering
  \includegraphics[width=.6\linewidth]{figure/car_parking.png}
  \caption{アドバンストパーク\cite{toyota}の経路と使用条件}
  \label{fig:AP2}
\end{figure}
実装されている現状のアドバンストパーク\cite{toyota}では図\ref{fig:AP2}のように,駐車開始位置が駐車目標位置の真横に固定されていることや駐車スペースの幅が$3[\mathrm{\, m}]$以上必要であることや,駐車スペース前の道路の幅が$5[\mathrm{\, m}]$以上必要であるといった使用条件がある.このように決められた広さ以上の駐車スペースが必要になるため適用できる環境が限定的になってしまう.しかし,このアドバンストパーク\cite{toyota}のように現在実装されている自動駐車システムでは対応できないような狭い環境の方が自動駐車システムを利用する価値がある.特に日本は狭い駐車場が多いため,より狭い環境でも自動駐車を行えるシステムの研究が行われているが,自動駐車システムが発展し,無人で駐車可能になれば駐車スペースを最小限にすることにより,駐車場の収容台数を増やすことができると期待される.

現在実装されている自動駐車技術よりもより狭い環境やより最適な経路で自動駐車行うための研究はこれまでにも行われている.例えば,RRT(rapid-exploring random tree)を用いる手法\cite{RRT1}\cite{RRT2} ,クロソイド曲線を用いる手法\cite{clothoid}\cite{clothoid2}や最小回転半径を用いる手法\cite{minR}\cite{minR2}などが挙げられる.RRTを用いる手法とは目標地点に近づくまで空間全体の探索を行うことで経路生成を行う手法で,複雑な環境でも経路生成が可能である.RRTを用いた手法\cite{RRT1}\cite{RRT2}では,目標地点から逆向きに経路探索を行うことでより少ないステップ数の探索で経路生成を行っている.しかし,経路が複雑になることや狭い環境では経路探索時間が長くなってしまう問題がある.クロソイド曲線を用いる手法\cite{clothoid}は,曲率が連続的に変化する滑らかな経路を生成することができる.また,近似クロソイド曲線を用いることで計算量を抑えた経路生成を行っている.また,クロソイド曲線を用いた別の手法\cite{clothoid2}では,狭い環境で縦列駐車を行うことを可能にしている.最小回転半径を用いた手法\cite{minR}ではシンプルな駐車経路を生成しているが,切り返しが考慮されていないことや決められた駐車開始位置からの駐車経路になってしまうといった課題がある.そこで文献\cite{ogata}では,切り返しを考慮したシンプルな駐車経路でかなり狭い環境での駐車経路の生成が可能となっている.しかし,生成した直線の経路から円弧の経路に進入する際に停止してその場でハンドルを最大まで切る必要があるため,ステアリングやタイヤに大きな負担がかかってしまうといった課題がある.

\subsection{研究目的}
本論文では,円弧と直線を用いた経路生成手法\cite{ogata} に,クロソイド曲線とシグモイド関数を導入し,滑らかな経路生成手法を提案する.クロソイド曲線を緩和曲線として導入することで,走行しながらハンドルを切ることが可能になり,ステアリングへの負担を軽減することを目的としている.ここで,緩和曲線とは,直線の経路から円弧の経路に進入するとき急な曲率の変化が起きるのを防ぐために直線と円弧の経路を繋ぐように生成する経路を意味している.また,駐車目標位置から駐車開始位置に向かって逆向きに経路を生成するため,本来の駐車開始位置と生成した経路の最終地点の位置がずれてしまうことがある.そこで,シグモイド関数を導入することにより,本来の駐車開始位置とのずれを補うように滑らかな駐車経路を生成できるようにすることを目的としている.


