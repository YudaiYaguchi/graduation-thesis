\newpage
\section{はじめに}
\subsection{研究背景}

近年,自動走行ロボットの研究が盛んに行われている.
特に物流業界では,自動配送ロボットの実用化に向けた開発や法律の改正が進められている\cite{meti_robot}.これにより,
トラック運転手の長時間労働の是正や物流分野の人手不足の緩和が期待される.
このようなロボットの社会実装を実現するためには,未知環境下や未知障害物の存在下において,安全かつ安定した経路を生成する必要がある.
特に,動的障害物を考慮した経路生成を行う場合は障害物の検出を行い,その移動を適切に予測することが重要となる.

移動ロボットにおける障害物回避のための経路生成手法として,
大局的経路計画やニューラルネットワーク,時空間RRT,ポテンシャル法を用いた方法がある.
大局的経路計画は,ロボットが地図全体などの環境情報をすでに把握している場合において,
スタートからゴールまでの最適な経路をあらかじめ計算して求める手法である.
ただし,この手法では静的環境下では非常に有効だが,動的障害物が存在する場合には
障害物の移動の度に経路を再計算する必要があり,計算コストが高くなるという欠点がある\cite{idou}.
次にニューラルネットワークを用いた方法では,多層パーセプトロン(MLP)などのモデルを使用し,静的および動的障害物のある環境下で,
ロボットの経路と速度を効率的に生成する手法が提案されている\cite{neural-network}.
このアプローチは,一度ネットワークの学習が完了すれば動的な環境においても少ない計算時間で経路を生成できるという利点を持つ.
しかし,ニューラルネットワークの一般的な欠点として,学習に多くのデータと時間を要するという課題がある.
この他にも,RRT(Rapidly-exploring Random Tree)を二次元平面から時間軸を含む空間へ拡張した時空間RRTが提案されている\cite{StRRT}.
これは位置だけでなく時間情報を考慮して探索を行うため,他ロボットとの衝突回避や譲り合いといった動的状況に対応した軌道生成が可能である.
しかし,サンプリングベース手法の一般的な課題として探索空間が高次元になるにつれて計算量が増大し,リアルタイム処理が難しい場面も存在する.
一方,ポテンシャル法\cite{potential}では,目的地からの引力および障害物からの斥力によって構成されるポテンシャル場を生成し,
ロボットがそのポテンシャル場の勾配にしたがって移動することで,目的地への経路を導出する手法である.
図\ref{fig:potential}にポテンシャル場の例を示す.
ポテンシャル法は計算量が少ないシンプルなアルゴリズムのためリアルタイム性が優れている手法である.
しかし,ポテンシャル法には「停留問題」がある.
「停留問題」とは障害物の斥力と目的地の引力が釣り合い,勾配が限りなく0に近い状態に陥り目的地にたどり着けなくなる問題である\cite{potential}.
\begin{figure}[t!]  \centering
  \includegraphics[width=.8\linewidth]{figure/potential.png}
  \setlength{\abovecaptionskip}{5pt} % キャプション上
  \setlength{\belowcaptionskip}{5pt} % キャプション下
  \caption{ポテンシャル場の例\cite{potential-field}}
  \label{fig:potential}
\end{figure}

従来研究\cite{kasuyama}~\cite{enokida}では,ポテンシャル場の勾配ベクトルに沿ってロボットを移動させることが考えられている.
従来研究\cite{kasuyama}では,センサが2点同時に障害物を検出した場合にその2点を結ぶ直線の傾きを算出し,その延長線を障害物の進行方向の予測線分として用いることで,
その予測線分に沿った壁沿い走行を行わせる手法を提案している.
また,予測線分と目的地方向との角度関係を用いて方向決定をしている.
一方,従来研究\cite{hiraoka}では,壁沿い走行後に斥力ポテンシャルの影響で大きく膨らんだ経路が生成される問題に着目し,
障害物の裏側へ回り込む円弧状の走行を組み合わせることでより効率的な経路を得る手法を提案している.
さらに,回避後も障害物を過度に認識し続けることで発生する冗長な経路を抑制するために,
障害物の重要度を示す重みパラメータを状況に応じて調整する手法を提案し,適切な重み設定により安全性を損なわずに経路効率を改善している.
従来研究\cite{koike}では,停留した際に障害物の向こう側に仮ゴールをA*探索法を用いて設定することで,「停留問題」を解決することに成功しており,静的障害物の回避に有効であることが示されている.
加えて,文献\cite{koike}では壁沿い走行に回り込み走行を組み込むことで,より柔軟に停留を回避する手法も提案されている.
一方,従来研究\cite{enokida}では,ポテンシャル法に速度推定機能を組み込むことで,動的障害物の回避に成功している.
この手法では,動的障害物の回避においてその速度情報を基に仮ゴールの設定を行っている.
そのため,障害物の速度や進行方向を考慮した経路生成が可能となり,従来研究\cite{koike}の静的障害物に対する回避手法を組み合わせることによって,
静的または動的環境下における自律移動を実現している.
しかし,障害物が加速や減速を伴って移動する場合には障害物の速度のみを用いた仮ゴール設定では将来位置を正確に推定できない場合がある.
速度情報はその時点におけるサンプリングした運動状態を表すに過ぎず,加速度による変化を考慮していない.
その結果,障害物が減速して向きを変えたり,障害物が加速して接近する状況では経路が冗長になる.
この誤差により,ロボットは障害物の進行方向を誤って解釈し,生成される経路が冗長になり目標地点まで効率的に移動できなくなる場合がある.

\subsection{研究目的}
本研究では,従来研究\cite{hiraoka}を基に,ポテンシャル法に障害物の加速度推定機能を組み込んだ動的障害物回避手法を提案する.
提案手法では,障害物の位置変化から速度および加速度を算出し,障害物の将来位置を予測する.
これにより,障害物の将来位置を考慮したより精度の高い回避行動を実現することを目的とする.
また障害物の将来位置を複数秒先まで予測し,それらを斥力ポテンシャル場に反映させることによって,
ロボットが障害物の進行経路に侵入することを回避し,より安全かつ効率的な経路の生成を実現する.
最後に従来研究\cite{enokida}と通常のポテンシャル法と提案する手法を到達時間および経路長の観点から比較し,その有効性を検証する.
