\newpage
\section{序論}
\subsection{研究背景}

近年,自動走行ロボットの研究が盛んに行われている.
特に物流業界では,自動配送ロボットの実用化に向けた開発や法律の改正が進められている\cite{meti_robot}.これにより,
トラック運転手の長時間労働の是正や物流分野の人手不足の緩和が期待される.
このような背景から,物流業界では労働力不足を補う手段として,自動配送ロボットの社会実装が強く期待されている.
そこで社会実装を実現するために,未知環境下や未知障害物の存在下において,安全かつ安定した経路を生成する必要がある.
特に,動的障害物を考慮した経路生成を行う場合は,障害物の検出を行い,その移動を適切に予測することが重要となる.

移動ロボットにおける障害物回避のための経路生成手法として,
大局的経路計画やニューラルネットワーク,ポテンシャル法を用いた方法がある.
大局的経路計画では,ロボットが地図全体などの環境情報をすでに把握している場合において,
スタートからゴールまでの最適な経路をあらかじめ計算して求める手法である.
この手法では,静的環境下ではでは非常に有効だが,動的障害物が存在する場合には,
障害物の移動のたびに経路を再計算する必要があり,計算コストが高くなるという欠点がある\cite{idou}.
また,ニューラルネットワークは学習に多くのデータと時間を要する\cite{neural-network}.
これに対し,ポテンシャル法\cite{potential}は,目的地からの引力および障害物からの斥力によって構成されるポテンシャル場を生成し,
ロボットがそのポテンシャル場の勾配に従って移動することで,目的地への経路を導出する手法である.
図\ref{fig:potential}にポテンシャル場の例を示す.
ポテンシャル法は計算量が少ないシンプルなアルゴリズムのためリアルタイム性が優れている手法である.
しかし,ポテンシャル法には「停留問題」がある.
「停留問題」とは障害物の斥力と目的地の引力が釣り合い,勾配が限りなく0に近い状態に陥り目的地にたどり着けなくなる問題である\cite{potential}.

従来研究\cite{kasuyama}~\cite{enokida}では,ポテンシャル場の勾配ベクトルに沿ってロボットを移動させることが考えられている.
特に従来研究\cite{koike}では,停留した際に障害物の向こう側に仮ゴールをA*探索法を用いて設定することで,「停留問題」を解決することに成功しており,静的障害物の回避に有効であることが示されている.
さらに,壁沿い走行に加えて回り込み走行を組み込むことで,より柔軟に停留を回避する手法も提案されている.
一方,従来研究\cite{enokida}では,ポテンシャル法に速度推定機能を組み込むことで,動的障害物の回避に成功している.
この手法では,動的障害物の回避においてその速度情報を基に仮ゴールの設定を行っている.
そのため,障害物の速度や進行方向を考慮した経路生成が可能となり,従来の静的障害物に対する回避手法を発展させた形で,動的環境下における自律移動を実現している.
しかし,加速度をもつ障害物に対しては,速度のみを基にした仮ゴールの設定では適切な回避が困難となるという問題がある.
その結果,生成される経路が遠回りになったり,目標地点までの最短経路で進行できない場合がある.
また,急加速や急減速を行う障害物の挙動を十分に反映できず,スムーズな回避動作を実現することが難しい.

\vspace{10pt} % 図の前にスペースを入れる\begin{figure}[htbp]
\begin{figure}[t!]  \centering
  \includegraphics[width=.8\linewidth]{figure/potential.png}
  \setlength{\abovecaptionskip}{5pt} % キャプション上
  \setlength{\belowcaptionskip}{5pt} % キャプション下
  \caption{ポテンシャル場の例\cite{potential-field}}
  \label{fig:potential}
\end{figure}

\subsection{研究目的}
本研究では,従来研究\cite{enokida}を基に,ポテンシャル法に障害物の加速度予測機能を組み込んだ動的障害物回避手法を提案する.
提案手法では,障害物の位置変化から速度および加速度を算出し,障害物の将来位置を予測する.
これにより,障害物の将来位置を考慮したより精度の高い回避行動を実現することを目的とする.
また,障害物の将来位置を複数ステップ先まで予測し,それらを斥力ポテンシャル場に反映させることで,
ロボットが障害物の進行経路に侵入することを防ぎ,より安全かつ安定した経路生成を実現する.
最後に,シミュレーションを通じて提案手法の有効性を検証する.
