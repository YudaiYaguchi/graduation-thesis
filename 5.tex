\newpage
\section{結論}

本研究では,加速度推定機能を組み込んだポテンシャル法による動的障害物回避手法を提案した.
従来研究\cite{enokida}では,動的障害物の速度情報のみを用いた回避手法を提案しているが,
加速度を持つ障害物に対しては適切な回避が困難であるという課題があった.
そこで本研究では,センサにより取得した障害物の位置情報から速度および加速度を推定し,
等加速度運動の運動方程式を用いて複数秒先の将来位置を予測する手法を提案した.
さらに,予測した将来位置群を斥力ポテンシャル場に反映させることで,
ロボットが障害物の進行経路に侵入することを防ぎ,より安全かつ効率的な経路生成を提案した.
シミュレーション結果より,
提案手法は従来研究\cite{enokida}およびセンサベースのポテンシャル法より短い到達時間と経路長を実現した.
特に,運動方向と反対に加速度をもつ動的障害物に対しても有効に機能することが確認され,
加速度を持つ障害物だけでなく,等速で移動する障害物に対しても効率的な回避を実現できることが示された.
また,動的障害物と静的障害物が同時に存在する複雑な環境下においても,
提案手法はそれぞれに対して最適な回避行動を選択することで,
目的地への到達に成功した.

以上の結果から,加速度推定機能を組み込むことで,動的障害物の運動状態をより正確に予測でき,
ロボットが余分な回避行動を取ることなく効率的な経路を選択できることが確認された.
これにより,加速度を持つ動的障害物に対するポテンシャル法の適用範囲が拡大され,
より実用的な自動配送ロボットの実現に貢献できると考えられる.
今後の課題として,より複雑な運動パターンを持つ障害物への対応や,
複数の動的障害物が同時に存在する環境における性能評価,
実機実験による検証などが挙げられる.
