\newpage
\section{結論}
本論文では,従来手法\cite{ogata}の円弧と直線を用いた経路生成にクロソイド曲線とシグモイド関数を導入することで,滑らかな経路を生成する手法を提案した.従来手法\cite{ogata}では最小回転半径により姿勢角を大きく変えることが可能な非常にシンプルな経路を生成することができるが,ステアリングを大きく切る必要がある.生成した経路を走行させることを考えると経路が滑らかな方が望ましいので,本研究ではクロソイド曲線とシグモイド関数を用いることでステアリングを大きく切らずによりスムーズな駐車ができる経路を生成する手法を提案した.また,現在実装されている自動運転技術\cite{toyota}では所望の駐車位置の真横に停車する必要があるため,クロソイド曲線とシグモイド関数を導入することにより,駐車開始位置によらず適切な経路生成を行うことを可能とした.また,シミュレーションを行うことにより,クロソイド曲線とシグモイド関数を導入することにより滑らかな経路生成を行うできることが確認できた.

今後の課題としては,クロソイドパラメータ$A$の決定方法について,より効率的な手法を検討する必要がある.






